%!TEX root = ../template.tex
%%%%%%%%%%%%%%%%%%%%%%%%%%%%%%%%%%%%%%%%%%%%%%%%%%%%%%%%%%%%%%%%%%%%
%% abstrac-pt.tex
%% NOVA thesis document file
%%
%% Abstract in Portuguese
%%%%%%%%%%%%%%%%%%%%%%%%%%%%%%%%%%%%%%%%%%%%%%%%%%%%%%%%%%%%%%%%%%%%

% O modelo de computação centralizado proposto pelo paradigma da Computação na Nuvem diverge do modelo das aplicações para a Internet das Coisas e para aplicações móveis. Dado que a maioria da produção e requisição de dados é feita por dispositivos que se encontram distantes dos centros de dados, ao armazenar dados e executar computações predominantemente em centros de dados, a computação na nuvem pode originar custos de infrastrutura adicionais e aumentar a latência para os utilizadores e fornecedores de serviços. Para mitigar estas limitações, surgiu um novo paradigma: Computação na Periferia. Este paradigma propõe executar computações, e potencialmente armazenar dados, em dispositivos fora dos centros de dados, reduzindo custos e criando um novo leque de possibilidades para efetuar computações distribuídas. 

% Neste trabalho expõe-se a criação de uma nova solução de monitorização e disseminação de informação descentralizada, desenhada para executar em sistemas de larga escala principalmente compostos por dispositivos com ligações de dados limitadas, como os que se encontram na periferia da rede. Esta solução, através do uso de uma rede em àrvore, providencia um leque de operações para coletar e processar, de uma forma descentralizada e eficiente, informação relativa a dispositivos (ou processos) a executar num dado sistema. Resultados em ambientes de emulação?? mostram que esta solução, quando empregue no contexto de disseminação de informação, apresenta pontos de saturação mais tarde do que alternativas no estado da arte. No contexto de monitorização, o sistema apresenta maior tolerância a falhas e, dependendo da configuração das alternativas e tamanho da rede, maior capacidade de coleção e agregação de métricas.

O modelo centralizado de computação utilizado no paradigma da Computação na Nuvem apresenta limitações no contexto de aplicações no domínio da Internet das Coisas e aplicações móveis. Neste tipo de aplicações, os dados são produzidos e consumidos maioritariamente por dispositivos que se encontram na periferia da rede. Desta forma, transportar estes dados de e para os centros de dados impõe uma carga excessiva nas infraestruturas de rede que ligam os dispositivos aos centros de dados, aumentando a latência de respostas e diminuindo a qualidade de serviço para os utilizadores. 

Para combater estas limitações, surgiu o paradigma da Computação na Periferia, este paradigma propõe a execução de computações, e potencialmente armazenamento de dados, em dispositivos fora dos centros de dados, mais perto dos clientes, reduzindo custos e criando um novo leque de possibilidades para efetuar computações distribuídas mais próximas dos dispositivos que produzem e consomem os dados.

Contudo, gerir e supervisionar a execução desses dispositivos levanta obstáculos não equacionados pela Computação na Nuvem, como a escala destes sistemas, ou a variabilidade na conectividade e na capacidade de computação dos dispositivos que os compõem. O estudo da literatura revela que ferramentas populares para gerir e supervisionar aplicações e dispositivos possuem limitações para a sua escalabilidade, como por exemplo, pontos de falha centralizados, ou requerem a configuração manual de cada dispositivo.

Nesta dissertação, propõem-se uma nova solução de monitorização e disseminação de informação descentralizada. Esta solução oferece operações que permitem recolher informação sobre o estado do sistema, de modo a ser utilizada por soluções (também descentralizadas) que gerem aplicações especializadas para executar na periferia da rede. A nossa solução foi avaliada em redes emuladas de várias dimensões com um máximo de 750 nós, no contexto de disseminação e de monitorização de informação. Os nossos resultados mostram que o nosso sistema consegue ser mais robusto ao mesmo tempo que é mais escalável quando comparado com o estado da arte.

\begin{keywords}
    Computação na periferia, Computação distribuída, Gestão de recursos, Monitorização, Gestão de topologias de redes
\end{keywords}

% desenhada para executar em sistemas de larga escala principalmente compostos por dispositivos com ligações de dados com capacidade limitada, como os que se encontram na periferia da rede.

% Esta solução baseia-se numa topologia descentralizada em árvore estabelecida entre dispositivos da periferia e da nuvem, que é utilizada para eficientemente disseminar, coletar, e processar informação relativa aos dispositivos (ou processos) em execução neste ambiente híbrido.


% como também recolher e agregar métricas sobre a operação e execução de componentes aplicacionais, de forma descentralizada.
%, relativas ao funcionamento de componentes de aplicações hospedadas nestes dispositivos.
% Estas métricas, por sua vez, auxiliam a tomada de decisão relativa à migração, replicação ou delegação (de porções) dos componentes aplicacionais, permitindo assim a adaptação autonómica do sistema.
% No entanto, à medida que nos distanciamos dos centros de dados, a capacidade de computação e armazenamento dos dispositivos tende a ser limitada. Tendo isto, surge a necessidade de partilhar recursos entre dispositivos na periferia, de modo a executar computações sofisticadas que outrora seriam impossíveis com um único dispositivo destes.

% O estudo do estado da arte revelou que as soluções existentes para a gestão e localização de recursos são normalmente especializadas para ambientes na Nuvem, onde os dispositivos têm capacidade de computação e armazenamento semelhantes, algo que não é adequado para ambientes dinâmicos e heterogéneos como a periferia do sistema. 



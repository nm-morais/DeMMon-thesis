%!TEX root = ../template.tex
%%%%%%%%%%%%%%%%%%%%%%%%%%%%%%%%%%%%%%%%%%%%%%%%%%%%%%%%%%%%%%%%%%%%
%% abstrac-en.tex
%% NOVA thesis document file
%%
%% Abstract in English([^%]*)
%%%%%%%%%%%%%%%%%%%%%%%%%%%%%%%%%%%%%%%%%%%%%%%%%%%%%%%%%%%%%%%%%%%%

\typeout{NT FILE abstrac-en.tex}

The centralized model proposed by the Cloud computing paradigm mismatches the decentralized nature of mobile and IoT applications, given the fact that most of the data production and consumption is performed by end-user devices outside of the Data Center (DC). As the number of these devices grows, and given the need to transport data to and from DCs for computation, application providers incur additional infrastructure costs, and end-users incur delays when performing operations. 

These reasons have led us into a post-cloud era, where a new computing paradigm arose: Edge Computing. Edge Computing takes into account the broad spectrum of devices residing outside of the DC, closer to the clients, as potential 
targets for computations, potentially reducing infrastructure costs, improving the quality of service (QoS) for end-users and allowing new interaction paradigms between users and applications. 

Managing and monitoring the execution of these devices raises new challenges previously unaddressed by Cloud computing, given the scale of these systems and the devices' (potentially) unreliable data connections and heterogenous computational power. The study of the state-of-the-art has revealed that existing resource monitoring and management solutions require manual configuration and have centralized components, which we believe do not scale for larger-scale systems. 

In this work, we address these limitations by presenting a novel Decentralized Management and Monitoring (``DeMMon'') system, targeted for edge settings. DeMMon provides primitives to ease the development of tools that manage computational resources that support edge-enabled applications, decomposed in components, through decentralized actions, taking advantage of partial knowledge of the system. Our solution was evaluated to amount to its benefits regarding information dissemination and monitoring capabilities across a set of realistic emulated scenarios of up to 750 nodes with variable failure rates. The results show the validity of our approach and that it can outperform state-of-the-art solutions regarding scalability and reliability. 

% . 
% , there is the need for them to effectively share resources and coordinate to accomplish tasks otherwise impossible for a single device. 

\begin{keywords}
  Edge Computing, Resource Management, Resource Monitoring, Topology Management, Middleware
\end{keywords} 

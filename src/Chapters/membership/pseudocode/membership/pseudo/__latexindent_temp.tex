% JOIN -----

% Types :

% Node {
%     ID \Comment{ string attributed by parent \asdupon[entering the syst]m}
%     measuredLatency \Comment{ the measured latency to the peer}
%     parentIP \Comment{ the IP of the parent of the Node}
%     nrChildren \Comment{ the number of children}
%     replied \Comment{ wether the node replied to joinMessage}
%     IP \Comment{ the node IP}
%     children: [] {
%         ID  \Comment{ the children id of the peer}
%         nrChildren  \Comment{ the nr of children of the child}
%         IP \Comment{ the node IP}
%     }
% }

\begin{algorithm}
% \setstretch{0.85}
\begin{algorithmic}[1]

\caption{Join Protocol (part 2)}
\asdupon[Init(landmarks : map{IP}:Node, selfIP)]
        \State joinTimeouts \asdassign \{\}
        \State bestPeerLastLevel \asdassign \{\}
        \State landmarks \asdassign []
        \State isLandmark \asdassign selfIP \asdin landmarks
        \If{isLandmark}
            \State self \asdassign landmarks[selfIP]
            \For{landmark in landmarks}
                \State siblings \asdassign siblings + landmark
                \State redialUntilSuccess(landmark)
                \State measureUntilSuccess(landmark)
            \EndFor
        \Else 
            \State progressToNextLevel(landmarkIps)
        \EndIf
    \asdend
    
\asdupon[JoinTimeoutTimer(L)]
        \If{(L in Landmarks)}
            \State rejoinLater()
        \Else
            \State delete(nodesToContact[L])
        \EndIf
    \asdend
    
\asdupon[receive(Join<>,sender)]
    \State sendMessageSideChannel(JoinReply<self.parent, self.node, self.children>, sender)
\asdend
    
\asdupon[receive JoinReply(<parentIP, node, children>, sender)]
        \State contactedNode \asdassign nodesToContact[node.IP]
        \State contactedNode.id \asdassign node.ID
        \State contactedNode.parentIP \asdassign parentIP
        \State contactedNode.nrChildren \asdassign len(children)
        \State contactedNode.replied \asdassign true
        \State contactedNode.children \asdassign children
    \asdend
   
\asdupon[NodeMeasured(node, latency)]
        \If{node $\in$ landmarkIPs}
            \State self.coordinates[landmarkIPs.indexOf(node)] = measuredLatency
        \EndIf
        \State nodesToContact[node].measuredLatency = latency
        \State nodesToContact[node].measured = true
    \asdend
        
\asdupon[NodeMeasuringFailed(node)]
    \State delete(nodesToContact, node)
\asdend

\asdprocedure[progressToNextLevel(nodeIPs)]
    \State nodesToContact \asdassign \{\}
    \State parentID \asdassign  nil
    \If{bestPeerLastLevel != nil}
        \State parentID \asdassign  bestPeerLastLevel.ID
    \EndIf
    \For{p in nodeIPs}
        \State nodesToContact[p] = Node \{IP: p, replied:false,measured: false\}
        \State MeasureNodeOnce(p) 
        \State sendMessageSideChannel(JoinMessage<>, p)
        \State t \asdassign setupTimer(JoinTimeoutTimer(p))
        \State joinTimeouts[p] = t
    \EndFor
\asdend

\end{algorithmic}
\end{algorithm}

\begin{algorithm}
% \setstretch{0.85}
\begin{algorithmic}[1]
\caption{Join Protocol (part 3)}
    
\asdupon[(forall n $\in$ nodesToContact -> n.measured \&\& n.replied)]
        \If{ len(nodesToContact) == 0}
            \If{bestPeerLastLevel == nil} \Comment{has not gotten past landmarks}
                \State rejToinLater()
                \State return
            \EndIf
        \EndIf
        \State contactedNodes.appendAll(nodesToContact)
        \For{node in sortedByLatency(nodesToContact)}
            \If{bestPeerLastLevel != nil}
                \If{bestPeerLastLevel.measuredLatency $\le$ node.measuredLatency} 
                    \State joinAsChild(bestPeerLastLevel)
                    \State return
                \EndIf
            \EndIf
            \If{(\asdnotin{node.IP}{landmarks}) \&\& node.nrChildren == 0}
                \State continue \Comment{ check if node has enough children to become joiner's parent (unless its a landmark)}
            \EndIf
            \State bestPeerLastLevel = node
            \State progressToNextLevel([c.IP for c in node.children])
            \State return
            \If{bestPeerLastLevel.parentIP == nil}
                \State bestPeerLastLevel = contactedNodes[bestPeerLastLevel.parentIP]
                \State joinAsChild(bestPeerLastLevel)       
                \State return
            \Else
                \State rejoinLater()
                \State return
            \EndIf
        \EndFor
    \asdend
   
    \asdprocedure[joinAsChild(p)]
        \State joinTimeoutTimerID = setupTimer(JoinRequestTimeout<p>)
        \State sendMessageSideChannel(JoinRequest<>, p.IP)
    \asdend

    \asdupon[JoinRequestTimeout(p)]
        \If{p.parentIP != nil}
            \State bestPeerLastLevel = contactedNodes[bestPeerLastLevel.parentIP]
            \State joinAsChild(bestPeerLastLevel)
        \Else
            \State rejoinLater()
        \EndIf
    \asdend

    \asdupon[receive(JoinRequest<>, sender)]
        \State generatedID \asdassign addChildren(sender) \Comment{ measures node and sets up periodic exchanges of information }

        \State sendMessageSideChannel(JoinRequestReply<generatedID,self,children>, p.IP)
    \asdend
        
    \asdupon[receive(JoinRequestReply<attributedId,parent,siblings>, sender)]
        \State addParent(sender) \Comment{ measures node and sets up periodic exchanges of information}
        \State addSiblings(siblings) \Comment{ measures nodes and sets up connections}
        \State self.id = parent.ID + "/" + attributedId
        \State cancelTimer(joinTimeoutTimerID)
    \asdend
\end{algorithmic}
\end{algorithm}

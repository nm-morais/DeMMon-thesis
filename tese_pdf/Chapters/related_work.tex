\chapter{Related Work} \label{cha:related_work}

\section{Context}

The following chapter provides context about to the challenge we attempt to solve. We present the sub-challenges identified towards performing management of microservices in the edge of the network. 

First, \textbf{microservice networking}: microservices need to cooperate towards solving applicational tasks, as such, peers in the system need to be integrated in an efficient abstraction layer (e.g. overlay) that allows them to find each other and communicate. This raises an old challenge in P2P computing: how can peers organize themselves in the network such that they can find resources (either other peers, services or even computing power) in an efficient way? It is important to notice that the edge environment is composed by lots of sub-networks composed of heterogenous devices concurrently entering and leaving the network, which is a hard scenario in \textbf{resource location systems}, particularly the underlying \textbf{topology management} challenge, because the overlay needs to adapt to the underlying network to remain efficient. 

Given this, section \ref{sec:topology_management} of this chapter provides context about \textbf{topology management}, particularly the main categories of topologies, how they can be evaluated, and a discussion about their applicability in edge environments. Second section, \ref{sec:res_location} studies \textbf{resource location architectures}, which leverage on the different types of topologies studied in \ref{sec:topology_management} to index resources in the system. For each type of architecture we discuss their differences and present popular implementations in the state of the art.

The next challenge is \textbf{maintaining quality of service} of microservices. When attempting to maintain quality of service we need to perform \textbf{monitoring} of device and service status, particularly of devices in the edge environment. Section \ref{sec:res_monitoring} covers popular techniques used in tracking device and service status, we particularly study which metrics to collect to aid in pinpointing causes of QoS degradation, and how to perform failure detection in edge systems, in the same section we discuss popular implementations of monitoring systems. 

Then, when federating and tracking massive amounts of monitoring data, transmitting and storing it becomes a limitation in the scalability of the system. To circumvent this, that data needs to be \textbf{aggregated}. Section \ref{sec:aggregation} studies aggregation, aggregation consists in the process of combining several numeric values into one single representative. We discuss the different types of aggregation, and how they can be applied towards maintaining quality of service.

Lastly, section \ref{sec:offloading_computation} studies how the aggregated results can be used to perform microservice management and deploymen. Namely how to orchestrate microservices such that the computation is offloaded to the Edge. We discuss popular paradigms such as Fog Computing and Osmotic Computing and how they compare to Edge Computing. Lastly, we discuss approaches towards implementing elastic computing in Edge environments.

\section{Topology Management} \label{sec:topology_management} 
% resource management in the DC
% -------------------
% Resource Management
% -------------------

Resource location systems are one of the most common applications of the P2P paradigm \cite{leitaoPHDthesis}, in a resource location system, a participant provided with a resource descriptor is able to query other peers and obtain an answer to the location (or absence) of that resource in the system within a reasonable amount of time. To do so, resource location systems employ search strategies, which depend on : (1) the structure the an overlay network (structured or unstructured). (2) on the characteristics of the resources to search (e.g. if there are many copies of it or not), and (3) on the desired results (e.g. if a single copy of a resource satisfies the query, or multiple are required). 

In the context of resource management, if a peer wishes to offload computations to other peers, it must employ an efficient search strategy to find nearby available resources (e.g., storage capacity, computing power, among others) in order to offload computations. In this section, we cover resource location and discovery, starting with the taxonomy of querying techniques for P2P systems, followed by the study of how resources can be stored or indexed and looked up throughout the topologies studied in the previous section.

\subsection{Querying techniques}

Querying techniques consist of how peers describe the resources they need, these, according \cite{leitaoPHDthesis}, may be classified as: \textbf{(1)~Exact Match queries}, these specify the resource to search by the value of a unique attribute (i.e., an identifier, commonly the hash of the value of the resource); the second querying methodology type is \textbf{(2)~keyword queries}, that employ one or more keywords (or tags) combined with logical operators to describe resources (e.g. "pop", "rock", "pop and rock" ...); next, \textbf{(3)~range queries} retrieve all resources whose value is contained within a given interval (e.g. "movies with 100 to 300 minutes of duration"); finally, \textbf{(4)~arbitrary queries} aim to find a set of nodes or resources that satisfy one or more arbitrary conditions (e.g. looking for a set of resources encoded in a certain format).

Provided with a way of describing their resource needs, peers need strategies to index and retrieve the resources in the system, there are three popular techniques: \textbf{centralized}, \textbf{distributed over an unstructured overlay}, or \textbf{distributed over a structured overlay}.

\subsection{Centralized Resource Location}

\textbf{Centralized resource location} relies on one (or a group of) centralized peers that index all existing resources. This type of architecture greatly reduces the complexity of systems, as peers only need to contact a subset of nodes to locate resources. 

It is important to notice that in a centralized architecture, while the indexation of resources is centralized, the resource access may still be distributed (e.g. a centralized server provides the addresses of peers who have the files, and files are obtained in a pure P2P fashion), a system which employs this architecture with success is BitTorrent \cite{cohen2003incentives}.

Although centralized architectures are widely used nowadays, they lack the necessary scalability to index the large number of dynamic resources we intend to manage, and have limited fault tolerance to failures, making them unsuited for edge environments. 

%However, there are many ways that a hybrid architecture can be applied to Edge computing: since the failure rate of a single data center (DC) is low, if we assume a system composed by multiple DCs, they may act as a reliable failover for whenever edge devices are partitioned of fail. 

\subsection{Resource Location on Unstructured Overlays}

When employing an unstructured overlay for resource location, the resources are scattered throughout all peers in the system, consequently, peers need to employ distributed search strategies to find the intended resources. This is accomplished through disseminating messages containing these queries throughout the overlay. The dissemination of these messages can follow multiple strategies, we now cover there two popular approaches: \textbf{flooding} and \textbf{random walks} \cite{leitaoPHDthesis}. 

\textbf{Flooding} consists of peers eagerly forwarding queries to others in the system as soon as they receive them for the first time, the objective of flooding is to contact multiple distinct peers that may have the queried resource. One approach is \textbf{complete flooding}, which consists in contacting every node in the system, this guarantees that if the resource exists, it will be found. However, complete flooding is not scalable and incurs significant message redundancy. \textbf{Flooding with limited horizon} minimizes the message overhead by attaching a TTL to messages that limits the number of times a message can be retransmitted. However, there is a trade-off for efficiency: flooding with limited horizon does not guarantee that all resources will be found. 

\textbf{Random Walks} are a dissemination strategy that attempts to minimize the communication overhead that is associated with flooding. A random walk consists of a message with a TTL that is randomly forwarded one peer at a time throughout the network. Random walks may also attempt to bias their path towards peers that are more likely to have answers to the query \cite{1022239}, this technique is commonly reffered to in the literature as a \textbf{random guided walk}. A common approach to bias random walks is to use bloom filters \cite{5751342}, which are space-efficient probabilistic data structures that allow the creation of imprecise distributed indexes for resources.

First generation of decentralized resource location systems relied on unstructured overlays (such as Gnutella \cite{gnutella_gtk}) and employed simple broadcasts with limited horizon to query other peers in the system. However, as the size of the system grew, simple flooding techniques lacked the required scalability for satisfying the rising number of queries, which triggered the emergence of new techniques to reduce the number of messages per query, called \textbf{super-peers}. 

\textbf{Super-peers} are peers which are assigned special roles in the system (often chosen in function of their capacity or stability). In the case of resource location systems, super-peers disseminate queries throughout the system. This technique is at the core of solutions such as Gia \cite{Chawathe2003}, employed towards effectively reducing the number of peers that have to disseminate queries on the second version of Gnutella \cite{gnutella_gtk}. 

\textbf{SOSP-Net} \cite{garbacki2007optimizing} (Self-Organizing Super-Peer Network) proposes a resource location system composed by regular peers and super-peers that effectively employs feedback concerning previous queries to improve the overlay network. Weak peers maintain links to super-peers which are biased based on the success of previous queries, and super-peers bias the routing of queries by taking into account the semantic content of each query. 

However, even with super-peers, one problem that still remains in these systems is finding very rare resources, which requires flooding the entire overlay. To circumvent this, the third generation of resource location systems rely on Distributed Hash Tables to ensure that even rare resources in the system can be found within a limited number of communication steps.

\subsection{Resource Location on Distributed Hash Tables}

Resource location on structured overlays is often done by relying on the applicational routing capabilities of distributed Hash Tables (DHTs). In a DHT, peers use hash functions to generate node identifiers (IDS) often uniformly distributed over the ID space. Then, by employing the same hash function to generate resource IDs, and assigning a portion of the ID space to each node, peers are able to map resources to the responsible peers in a bounded number of steps, which makes them very suitable for (\textbf{exact match queries}) \cite{leitaoPHDthesis}. 

One particular type of DHT that is commonly employed in small-sized resource location systems is the One-Hop Distributed Hash Table (DHT), nodes in a one-hop DHT have full membership of the system, and can locally map resources to known peers, thus performing lookups in O(1) time and message complexity. Facebook's Cassandra \cite{lakshman2010cassandra} and Amazon's Dynamo \cite{decandia2007dynamo} are widely used implementations of one-hop DHTs. 

There are two popular techniques for storing resources in a DHT, the first approach is to store the resources locally and publish the location of the resource in the DHT. This way, the node responsible for the resource's key only stores the locations of other nodes in the system and the resource may be replicated among distinct nodes composing the system. The second technique consists of transferring the resource to the responsible node in the DHT, although fewer nodes must keep the same value. It is, however, important to mention that this way the resources are not replicated, provided that with consistent hashing, all nodes with the same resource will publish the resource in the same location of the DHT.

\subsection{Discussion}

As mentioned previously, we believe centralized resource location systems are unsuited for edge environments, given that as previously mentioned in section \ref{sec:context}, for our goal, centralizing the computation (for example in data-centers) will eventually lead to a bottleneck for the system scalability. Furthermore, these types of systems are plagued with a single point of failure, making them unsuitable for volatile environments. \todo {improve}

Unstructured resource location systems are attractive for systems that perform queries in search for resources with multiple copies or for range queries, however, this approach is inefficient when performing exact match queries, as a finding the exact resource in an unstructured resource location system requires flooding the entire system with messages.

Conversely, distributed hash tables are specially tailored towards exact match queries, but are less robust to churn and are subject to low-capacity nodes being a bottleneck in routing procedures. 

% In the context of the proposed solution, given that the resources we intend to manage are present in all nodes (e.g., computing power, memory, among others), we believe unstructured resource location is better suited for our needs. For example, if an edge device wishes to find nearby computing resources to offload a certain task, it may employ a random walk. On the other hand, if a peer wishes to find a larger set of computing resources to deploy multiple application components, it may employ flooding techniques. 

%Hybrid approaches between  simultaneously ensure load balancing properties and address  problems related to churn and 

%\subsection{Hybrid approaches}

%\textbf{Curiata} \& \textbf{Build One Get One Free}

% TODO falar de surrogate routing
%\textcolor{red}{surrogate routing}

%\paragraph{\textbf{Viceroy} }

%\paragraph{\textbf{Koala} }

% TODO falar de otimizacoes fixes observadas:
% lazyness a montar a rede (usar mensagens de servicos)
% manter peers antigos (churn inversamente proporcional a uptime)
% formar grupos para reduzir routing (increased background communication)
% ao usar prefix routing consegue-se logb(n) routing
% Xor-distance vs numeric distance (unidirectionality)
% Pedidos assincronos para fazer queries mais rapidas
% usar um algoritmo para dar "feed" ao outro (gossip + dht)




\section{Resource Location and Discovery} \label{sec:res_location} 
% resource management in the DC
% -------------------
% Resource Management
% -------------------

Resource location systems are one of the most common applications of the P2P paradigm \cite{leitaoPHDthesis}, in a resource location system, a participant provided with a resource descriptor is able to query other peers and obtain an answer to the location (or absence) of that resource in the system within a reasonable amount of time. To do so, resource location systems employ search strategies, which depend on : (1) the structure the an overlay network (structured or unstructured). (2) on the characteristics of the resources to search (e.g. if there are many copies of it or not), and (3) on the desired results (e.g. if a single copy of a resource satisfies the query, or multiple are required). 

In the context of resource management, if a peer wishes to offload computations to other peers, it must employ an efficient search strategy to find nearby available resources (e.g., storage capacity, computing power, among others) in order to offload computations. In this section, we cover resource location and discovery, starting with the taxonomy of querying techniques for P2P systems, followed by the study of how resources can be stored or indexed and looked up throughout the topologies studied in the previous section.

\subsection{Querying techniques}

Querying techniques consist of how peers describe the resources they need, these, according \cite{leitaoPHDthesis}, may be classified as: \textbf{(1)~Exact Match queries}, these specify the resource to search by the value of a unique attribute (i.e., an identifier, commonly the hash of the value of the resource); the second querying methodology type is \textbf{(2)~keyword queries}, that employ one or more keywords (or tags) combined with logical operators to describe resources (e.g. "pop", "rock", "pop and rock" ...); next, \textbf{(3)~range queries} retrieve all resources whose value is contained within a given interval (e.g. "movies with 100 to 300 minutes of duration"); finally, \textbf{(4)~arbitrary queries} aim to find a set of nodes or resources that satisfy one or more arbitrary conditions (e.g. looking for a set of resources encoded in a certain format).

Provided with a way of describing their resource needs, peers need strategies to index and retrieve the resources in the system, there are three popular techniques: \textbf{centralized}, \textbf{distributed over an unstructured overlay}, or \textbf{distributed over a structured overlay}.

\subsection{Centralized Resource Location}

\textbf{Centralized resource location} relies on one (or a group of) centralized peers that index all existing resources. This type of architecture greatly reduces the complexity of systems, as peers only need to contact a subset of nodes to locate resources. 

It is important to notice that in a centralized architecture, while the indexation of resources is centralized, the resource access may still be distributed (e.g. a centralized server provides the addresses of peers who have the files, and files are obtained in a pure P2P fashion), a system which employs this architecture with success is BitTorrent \cite{cohen2003incentives}.

Although centralized architectures are widely used nowadays, they lack the necessary scalability to index the large number of dynamic resources we intend to manage, and have limited fault tolerance to failures, making them unsuited for edge environments. 

%However, there are many ways that a hybrid architecture can be applied to Edge computing: since the failure rate of a single data center (DC) is low, if we assume a system composed by multiple DCs, they may act as a reliable failover for whenever edge devices are partitioned of fail. 

\subsection{Resource Location on Unstructured Overlays}

When employing an unstructured overlay for resource location, the resources are scattered throughout all peers in the system, consequently, peers need to employ distributed search strategies to find the intended resources. This is accomplished through disseminating messages containing these queries throughout the overlay. The dissemination of these messages can follow multiple strategies, we now cover there two popular approaches: \textbf{flooding} and \textbf{random walks} \cite{leitaoPHDthesis}. 

\textbf{Flooding} consists of peers eagerly forwarding queries to others in the system as soon as they receive them for the first time, the objective of flooding is to contact multiple distinct peers that may have the queried resource. One approach is \textbf{complete flooding}, which consists in contacting every node in the system, this guarantees that if the resource exists, it will be found. However, complete flooding is not scalable and incurs significant message redundancy. \textbf{Flooding with limited horizon} minimizes the message overhead by attaching a TTL to messages that limits the number of times a message can be retransmitted. However, there is a trade-off for efficiency: flooding with limited horizon does not guarantee that all resources will be found. 

\textbf{Random Walks} are a dissemination strategy that attempts to minimize the communication overhead that is associated with flooding. A random walk consists of a message with a TTL that is randomly forwarded one peer at a time throughout the network. Random walks may also attempt to bias their path towards peers that are more likely to have answers to the query \cite{1022239}, this technique is commonly reffered to in the literature as a \textbf{random guided walk}. A common approach to bias random walks is to use bloom filters \cite{5751342}, which are space-efficient probabilistic data structures that allow the creation of imprecise distributed indexes for resources.

First generation of decentralized resource location systems relied on unstructured overlays (such as Gnutella \cite{gnutella_gtk}) and employed simple broadcasts with limited horizon to query other peers in the system. However, as the size of the system grew, simple flooding techniques lacked the required scalability for satisfying the rising number of queries, which triggered the emergence of new techniques to reduce the number of messages per query, called \textbf{super-peers}. 

\textbf{Super-peers} are peers which are assigned special roles in the system (often chosen in function of their capacity or stability). In the case of resource location systems, super-peers disseminate queries throughout the system. This technique is at the core of solutions such as Gia \cite{Chawathe2003}, employed towards effectively reducing the number of peers that have to disseminate queries on the second version of Gnutella \cite{gnutella_gtk}. 

\textbf{SOSP-Net} \cite{garbacki2007optimizing} (Self-Organizing Super-Peer Network) proposes a resource location system composed by regular peers and super-peers that effectively employs feedback concerning previous queries to improve the overlay network. Weak peers maintain links to super-peers which are biased based on the success of previous queries, and super-peers bias the routing of queries by taking into account the semantic content of each query. 

However, even with super-peers, one problem that still remains in these systems is finding very rare resources, which requires flooding the entire overlay. To circumvent this, the third generation of resource location systems rely on Distributed Hash Tables to ensure that even rare resources in the system can be found within a limited number of communication steps.

\subsection{Resource Location on Distributed Hash Tables}

Resource location on structured overlays is often done by relying on the applicational routing capabilities of distributed Hash Tables (DHTs). In a DHT, peers use hash functions to generate node identifiers (IDS) often uniformly distributed over the ID space. Then, by employing the same hash function to generate resource IDs, and assigning a portion of the ID space to each node, peers are able to map resources to the responsible peers in a bounded number of steps, which makes them very suitable for (\textbf{exact match queries}) \cite{leitaoPHDthesis}. 

One particular type of DHT that is commonly employed in small-sized resource location systems is the One-Hop Distributed Hash Table (DHT), nodes in a one-hop DHT have full membership of the system, and can locally map resources to known peers, thus performing lookups in O(1) time and message complexity. Facebook's Cassandra \cite{lakshman2010cassandra} and Amazon's Dynamo \cite{decandia2007dynamo} are widely used implementations of one-hop DHTs. 

There are two popular techniques for storing resources in a DHT, the first approach is to store the resources locally and publish the location of the resource in the DHT. This way, the node responsible for the resource's key only stores the locations of other nodes in the system and the resource may be replicated among distinct nodes composing the system. The second technique consists of transferring the resource to the responsible node in the DHT, although fewer nodes must keep the same value. It is, however, important to mention that this way the resources are not replicated, provided that with consistent hashing, all nodes with the same resource will publish the resource in the same location of the DHT.

\subsection{Discussion}

As mentioned previously, we believe centralized resource location systems are unsuited for edge environments, given that as previously mentioned in section \ref{sec:context}, for our goal, centralizing the computation (for example in data-centers) will eventually lead to a bottleneck for the system scalability. Furthermore, these types of systems are plagued with a single point of failure, making them unsuitable for volatile environments. \todo {improve}

Unstructured resource location systems are attractive for systems that perform queries in search for resources with multiple copies or for range queries, however, this approach is inefficient when performing exact match queries, as a finding the exact resource in an unstructured resource location system requires flooding the entire system with messages.

Conversely, distributed hash tables are specially tailored towards exact match queries, but are less robust to churn and are subject to low-capacity nodes being a bottleneck in routing procedures. 

% In the context of the proposed solution, given that the resources we intend to manage are present in all nodes (e.g., computing power, memory, among others), we believe unstructured resource location is better suited for our needs. For example, if an edge device wishes to find nearby computing resources to offload a certain task, it may employ a random walk. On the other hand, if a peer wishes to find a larger set of computing resources to deploy multiple application components, it may employ flooding techniques. 

%Hybrid approaches between  simultaneously ensure load balancing properties and address  problems related to churn and 

%\subsection{Hybrid approaches}

%\textbf{Curiata} \& \textbf{Build One Get One Free}

% TODO falar de surrogate routing
%\textcolor{red}{surrogate routing}

%\paragraph{\textbf{Viceroy} }

%\paragraph{\textbf{Koala} }

% TODO falar de otimizacoes fixes observadas:
% lazyness a montar a rede (usar mensagens de servicos)
% manter peers antigos (churn inversamente proporcional a uptime)
% formar grupos para reduzir routing (increased background communication)
% ao usar prefix routing consegue-se logb(n) routing
% Xor-distance vs numeric distance (unidirectionality)
% Pedidos assincronos para fazer queries mais rapidas
% usar um algoritmo para dar "feed" ao outro (gossip + dht)




\section{Monitoring} \label{sec:res_monitoring} 

\subsection{Device monitoring}

\subsection{Service monitoring}

\subsection{Relevant monitoring services}

\subsection{discussion}

\section{Aggregation} \label{sec:aggregation} 
Aggregation is an essential building block on modern distributed systems, it  enables the determination of important system wide properties in a decentralized manner \cite{DBLP:journals/corr/abs-1110-0725}. Aggregation consists in computing an aggregation function over a set of input values where each node has one input value. Common aggregation functions consist in sum, count, average, min, max, where each function presents different properties which must be considered.

Towards deploying services in edge devices, aggregation may be used to monitor the device and service state (e.g. computing the average latency of the closest available service that meets a certain criteria; counting nearby available computing resources that can be used to offload services, or identify hotspots by aggregating the average system load in certain areas). Given this, it is important to understand the challenges of aggregating values in a highly decentralized manner. There are two properties of aggregation functions: \textit{decomposable functions} and \textit{duplicate sensitive} functions.

\subsection{Properties of aggregation functions}

For some aggregation functions, we may need to involve all elements in the multiset, however, for memory and bandwidth issues, it is impractical to perform a centralized computation, hence, the aim is to employ \textit{in-transit computation}. In order to enable this, it is required that the aggregation function is \textbf{decomposable}. Intuitively, a decomposable aggregation function is one where a function may be composed defined as a composition of other functions. Decomposable functions may \textit{self-decomposable}, which intuitively means that the aggregated value is the same for all possible combinations of all sub-multisets partitioned in the multiset. This happens whenever the applied function is commutative and associative (e.g. min, max, sum, count). A canonical example of a decomposable function that is not self-decomposable is average, which consists in the sum of all pairs divided by the count of peers that contributed to the aggregation.

The second property of aggregation is \textbf{duplicate sensitiveness}, and it is related to wether a given value occurs several times in a multiset. Depending on the aggregation function used, the presence of repeated values may influence the result, it is said that a function is \textbf{duplicate sensitive} if the result of the aggregation function is influenced by the repeated values (e.g. SUM), conversely, if the aggregation function is \textbf{duplicate insensitive} it can be successfully repeated any number of times to the same multiset without affecting the result (e.g. MIN and MAX).

Table \ref{table:aggregation_functions} classifies popular aggregation functions in function of decomposability and duplicate sensitiveness as found in \cite{DBLP:journals/corr/abs-1110-0725}:

\begin{table}[]
    \begin{tabular}{|l|l|l|l|}
    \hline
                          & \multicolumn{2}{l|}{Decomposable} & Non-Decomposable  \\ \hline
                          & Self-decomposable    &                             &  \\ \hline
    Duplicate insensitive & Min, Max             & Range     & Distinct Count    \\ \hline
    Duplicate sensitive   & Sum, Count           & Average   & Median, Mode     \\ \hline
    \end{tabular}
    \caption{popular aggregation functions in function of decomposability and duplicate sensitiveness}
    \label{table:aggregation_functions}
\end{table}

Building on the concepts of duplicate sensitiveness and decomposability, we show that aggregation functions present their own particularities which dictate their applicability in particular scenarios. For example, a Min or Max function may be easier to implement with a simpler algorithm, while Sum, Count and Average require extra considerations. This presents a limitation towards calculating exact aggregations in large scale systems, to circumvent this, some systems do not require obtaining exact aggregated values to perform near optimally  (e.g. estimating the system size in order to select the optimal fanout for a gossip system only requires an estimation of the magnitude of the system). 

\subsection{Aggregation techniques}

Following, we present the studied categories of aggregation techniques: Hierarchical, Averaging, Sketches (hash or min-k based), Digests, Deterministic and Sampling. In each technique, we discuss its applicability in the edge environment.

\subsubsection{Hierarchical}

\textbf{Hierarchical} approaches leverage directly on the decomposability of aggregation functions. Aggregations from this class depend on the existence of a hierarchical communication structure, (e.g. a spanning tree) with one root (sink node). Aggregations take place by splitting inputs into groups and aggregating values bottom-up in the hierarchy. Commonly, hierarchical aggregation systems have nodes whose roles are \textit{aggregators} or \textit{forwarders}, intuitively, aggregators compute the aggregation functions forward results to forwarders who transfer results to upper levels in the hierarchy. In the absence of faults, the correct final result is obtained in the sink node. Many systems employ hierarchical approaches to aggregation, namely TAG \cite{}, DAG \cite{}, among others. Hierarchical approaches, due to taking advantage of device heterogeneity, are attractive in edge environments. However, due to the low computational power of devices, not all nodes may be  able to handle the additional overhead of maintaining the hierarchical topology.

\textbf{Averaging} aggregation consists in the continuous computation and exchanging of partial averages data among all active nodes in the aggregation process. In this type of systems, after a few rounds, all nodes usually converge to the correct value with high accuracy, as shown in \cite{gossip_aggregation}. This type of aggregation is attractive for gossip protocols, where nodes may employ varied gossip techniques to continuously share and update their values with random neighbors. Algorithms from this category are also attractive to use in edge environments, because they are accurate while employing random unstructured overlays, which retain their fault-tolerance and resilience to churn.

\textbf{Sketches} are fixed-size data structures that hold a \textit{sketch} of all network values. Multiple sketches are usually forwarded throughout the system, and nodes who forward sketches apply (usually commutative and associative) operations to update and merge them. \textcolor{red}{functioning and edge discussion}

% completar functionamento

\textbf{Digests} are an aggregation technique that gathers a representation of all system values, it supports complex aggregation functions such as Median and Mode. In short, algorithms employ a fixed-size data structures commonly composed of a set of values and associated counters) which compacts the data distribution (e.g. into a histogram). \textcolor{red}{edge discussion}

\textbf{Counting} algorithms target the same aggregation function: Count, algorithms from this class usually employ some randomized procedure to achieve a probabilistic approximation of the population size.

\subsection{Relevant aggregation protocols}

In this subsection we will analyze relevant aggregation protocolsthat ilustrate some techniques discussed above.

\textbf{TAG: Tiny AGgregation}\cite{Madden2002} is a service for aggregation in low-power, distributed, wireless sensor networks. TAG distributes queries in the network in a time and power-efficient manner by employing a hierarchical aggregation pattern. For each aggregation procedure, there is a \textit{root} nodes which broadcasts a message to start the tree-building process, each message contains two fields: a level and a an ID. Whenever a node without an assigned level receives a tree-building message, it assigns its own level as the message level plus one, and its own parent as the message sender. Then, it reassigns the level and ID to its own and forwards the message to other nodes. Then, whenever a node wishes to send a message to the root, it simply forwards the message bottom-up in the tree. The formed topology allows the computation of Count, Maximum, Minimum, Sum and Average. It is important to notice that the formed tree will be unbalanced as a function of the underlay latency and processing time.

\textbf{DECA} \cite{Artigas2006} \textcolor{red}{//TODO}

\textbf{Astrolabe} \cite{Renesse2003} \textcolor{red}{//TODO}

\textbf{SingleTree \cite{} and MultipleTree \cite{}} \textcolor{red}{//TODO}

\subsection{Discussion}

\section{Offloading computation to the Edge} \label{sec:offloading_computation} \subsubsection{Decentralizing clouds}

\subsubsection{Fog Computing}

\subsubsection{Edge Computing}

\subsubsection{Osmotic Computing}



\chapter{Related Work} \label{cha:related_work}

The following chapter provides context about the related work done towards solving the identified challenges. First section introduces related work about resource location systems, mainly studies most popular types of resource location architectures. For each type of architecture we present popular implementations in the state of the art and discuss their applicability in Edge Computing Environments. 

Second section addresses resource and service management, which is paramount towards maintaining QoS, especially in Edge devices which have limited capabilities. We examine popular system implementations and discuss their advantages and disadvantages. Third section studies aggregation, which is essential for handling the large flow of data originating from performing resource management. This section studies different types of aggregation protocols with an emphasis on protocols that leverage on hierarchical topologies to perform aggregation.

Fourth section discusses what type of resources are commonly collected towards service deployment. We study how to perform queries over large networks in order to find suitable devices that own enough resources for the service deployments. 

Finally, we address how computation can be offloaded to the Edge, discuss popular paradigms such as Fog Computing and Osmotic Computing and how they differ from Edge Computing. Lastly, we discuss  recent approaches towards implementing elastic computing in Edge environments.

\section{Resource Location} \label{sec:res_location} 
% resource management in the DC
% -------------------
% Resource Management
% -------------------

Resource location systems are one of the most common applications of the P2P paradigm \cite{leitaoPHDthesis}, in a resource location system, a participant provided with a resource descriptor is able to query other peers and obtain an answer to the location (or absence) of that resource in the system within a reasonable amount of time. To do so, resource location systems employ search strategies, which depend on : (1) the structure the an overlay network (structured or unstructured). (2) on the characteristics of the resources to search (e.g. if there are many copies of it or not), and (3) on the desired results (e.g. if a single copy of a resource satisfies the query, or multiple are required). 

In the context of resource management, if a peer wishes to offload computations to other peers, it must employ an efficient search strategy to find nearby available resources (e.g., storage capacity, computing power, among others) in order to offload computations. In this section, we cover resource location and discovery, starting with the taxonomy of querying techniques for P2P systems, followed by the study of how resources can be stored or indexed and looked up throughout the topologies studied in the previous section.

\subsection{Querying techniques}

Querying techniques consist of how peers describe the resources they need, these, according \cite{leitaoPHDthesis}, may be classified as: \textbf{(1)~Exact Match queries}, these specify the resource to search by the value of a unique attribute (i.e., an identifier, commonly the hash of the value of the resource); the second querying methodology type is \textbf{(2)~keyword queries}, that employ one or more keywords (or tags) combined with logical operators to describe resources (e.g. "pop", "rock", "pop and rock" ...); next, \textbf{(3)~range queries} retrieve all resources whose value is contained within a given interval (e.g. "movies with 100 to 300 minutes of duration"); finally, \textbf{(4)~arbitrary queries} aim to find a set of nodes or resources that satisfy one or more arbitrary conditions (e.g. looking for a set of resources encoded in a certain format).

Provided with a way of describing their resource needs, peers need strategies to index and retrieve the resources in the system, there are three popular techniques: \textbf{centralized}, \textbf{distributed over an unstructured overlay}, or \textbf{distributed over a structured overlay}.

\subsection{Centralized Resource Location}

\textbf{Centralized resource location} relies on one (or a group of) centralized peers that index all existing resources. This type of architecture greatly reduces the complexity of systems, as peers only need to contact a subset of nodes to locate resources. 

It is important to notice that in a centralized architecture, while the indexation of resources is centralized, the resource access may still be distributed (e.g. a centralized server provides the addresses of peers who have the files, and files are obtained in a pure P2P fashion), a system which employs this architecture with success is BitTorrent \cite{cohen2003incentives}.

Although centralized architectures are widely used nowadays, they lack the necessary scalability to index the large number of dynamic resources we intend to manage, and have limited fault tolerance to failures, making them unsuited for edge environments. 

%However, there are many ways that a hybrid architecture can be applied to Edge computing: since the failure rate of a single data center (DC) is low, if we assume a system composed by multiple DCs, they may act as a reliable failover for whenever edge devices are partitioned of fail. 

\subsection{Resource Location on Unstructured Overlays}

When employing an unstructured overlay for resource location, the resources are scattered throughout all peers in the system, consequently, peers need to employ distributed search strategies to find the intended resources. This is accomplished through disseminating messages containing these queries throughout the overlay. The dissemination of these messages can follow multiple strategies, we now cover there two popular approaches: \textbf{flooding} and \textbf{random walks} \cite{leitaoPHDthesis}. 

\textbf{Flooding} consists of peers eagerly forwarding queries to others in the system as soon as they receive them for the first time, the objective of flooding is to contact multiple distinct peers that may have the queried resource. One approach is \textbf{complete flooding}, which consists in contacting every node in the system, this guarantees that if the resource exists, it will be found. However, complete flooding is not scalable and incurs significant message redundancy. \textbf{Flooding with limited horizon} minimizes the message overhead by attaching a TTL to messages that limits the number of times a message can be retransmitted. However, there is a trade-off for efficiency: flooding with limited horizon does not guarantee that all resources will be found. 

\textbf{Random Walks} are a dissemination strategy that attempts to minimize the communication overhead that is associated with flooding. A random walk consists of a message with a TTL that is randomly forwarded one peer at a time throughout the network. Random walks may also attempt to bias their path towards peers that are more likely to have answers to the query \cite{1022239}, this technique is commonly reffered to in the literature as a \textbf{random guided walk}. A common approach to bias random walks is to use bloom filters \cite{5751342}, which are space-efficient probabilistic data structures that allow the creation of imprecise distributed indexes for resources.

First generation of decentralized resource location systems relied on unstructured overlays (such as Gnutella \cite{gnutella_gtk}) and employed simple broadcasts with limited horizon to query other peers in the system. However, as the size of the system grew, simple flooding techniques lacked the required scalability for satisfying the rising number of queries, which triggered the emergence of new techniques to reduce the number of messages per query, called \textbf{super-peers}. 

\textbf{Super-peers} are peers which are assigned special roles in the system (often chosen in function of their capacity or stability). In the case of resource location systems, super-peers disseminate queries throughout the system. This technique is at the core of solutions such as Gia \cite{Chawathe2003}, employed towards effectively reducing the number of peers that have to disseminate queries on the second version of Gnutella \cite{gnutella_gtk}. 

\textbf{SOSP-Net} \cite{garbacki2007optimizing} (Self-Organizing Super-Peer Network) proposes a resource location system composed by regular peers and super-peers that effectively employs feedback concerning previous queries to improve the overlay network. Weak peers maintain links to super-peers which are biased based on the success of previous queries, and super-peers bias the routing of queries by taking into account the semantic content of each query. 

However, even with super-peers, one problem that still remains in these systems is finding very rare resources, which requires flooding the entire overlay. To circumvent this, the third generation of resource location systems rely on Distributed Hash Tables to ensure that even rare resources in the system can be found within a limited number of communication steps.

\subsection{Resource Location on Distributed Hash Tables}

Resource location on structured overlays is often done by relying on the applicational routing capabilities of distributed Hash Tables (DHTs). In a DHT, peers use hash functions to generate node identifiers (IDS) often uniformly distributed over the ID space. Then, by employing the same hash function to generate resource IDs, and assigning a portion of the ID space to each node, peers are able to map resources to the responsible peers in a bounded number of steps, which makes them very suitable for (\textbf{exact match queries}) \cite{leitaoPHDthesis}. 

One particular type of DHT that is commonly employed in small-sized resource location systems is the One-Hop Distributed Hash Table (DHT), nodes in a one-hop DHT have full membership of the system, and can locally map resources to known peers, thus performing lookups in O(1) time and message complexity. Facebook's Cassandra \cite{lakshman2010cassandra} and Amazon's Dynamo \cite{decandia2007dynamo} are widely used implementations of one-hop DHTs. 

There are two popular techniques for storing resources in a DHT, the first approach is to store the resources locally and publish the location of the resource in the DHT. This way, the node responsible for the resource's key only stores the locations of other nodes in the system and the resource may be replicated among distinct nodes composing the system. The second technique consists of transferring the resource to the responsible node in the DHT, although fewer nodes must keep the same value. It is, however, important to mention that this way the resources are not replicated, provided that with consistent hashing, all nodes with the same resource will publish the resource in the same location of the DHT.

\subsection{Discussion}

As mentioned previously, we believe centralized resource location systems are unsuited for edge environments, given that as previously mentioned in section \ref{sec:context}, for our goal, centralizing the computation (for example in data-centers) will eventually lead to a bottleneck for the system scalability. Furthermore, these types of systems are plagued with a single point of failure, making them unsuitable for volatile environments. \todo {improve}

Unstructured resource location systems are attractive for systems that perform queries in search for resources with multiple copies or for range queries, however, this approach is inefficient when performing exact match queries, as a finding the exact resource in an unstructured resource location system requires flooding the entire system with messages.

Conversely, distributed hash tables are specially tailored towards exact match queries, but are less robust to churn and are subject to low-capacity nodes being a bottleneck in routing procedures. 

% In the context of the proposed solution, given that the resources we intend to manage are present in all nodes (e.g., computing power, memory, among others), we believe unstructured resource location is better suited for our needs. For example, if an edge device wishes to find nearby computing resources to offload a certain task, it may employ a random walk. On the other hand, if a peer wishes to find a larger set of computing resources to deploy multiple application components, it may employ flooding techniques. 

%Hybrid approaches between  simultaneously ensure load balancing properties and address  problems related to churn and 

%\subsection{Hybrid approaches}

%\textbf{Curiata} \& \textbf{Build One Get One Free}

% TODO falar de surrogate routing
%\textcolor{red}{surrogate routing}

%\paragraph{\textbf{Viceroy} }

%\paragraph{\textbf{Koala} }

% TODO falar de otimizacoes fixes observadas:
% lazyness a montar a rede (usar mensagens de servicos)
% manter peers antigos (churn inversamente proporcional a uptime)
% formar grupos para reduzir routing (increased background communication)
% ao usar prefix routing consegue-se logb(n) routing
% Xor-distance vs numeric distance (unidirectionality)
% Pedidos assincronos para fazer queries mais rapidas
% usar um algoritmo para dar "feed" ao outro (gossip + dht)




\section{Resource and Service Management} \label{sec:res_management} 
In this section, we study resource management in the context of edge environments. Resource management consists in providing resources (e.g. computing power, memory, among others) to tenants (i.e. applications, frameworks, among others), such that these can perform their computations. In this section, we cover aspects of resource management solutions and study popular solutions in the literature.

\subsection{Resource Management Taxonomy}

A resource management system aims at controlling the distribution of resources among tenants. We may classify resource management architectures according to their \textit{control} and \textit{tenancy}.

\subsubsection{Tenancy}

The term tenancy in resource management refers to whether or not underlying hardware resources are shared among entities \cite{Hong2019}.

\textbf{Single tenancy} refers to an architecture in which a single instance of a software application and supporting infrastructure serves one customer. In single-tenancy architectures, a customer (tenant) has nearly full control over the customization of software and infrastructure.

\textbf{Multi-tenancy} consists of tenants sharing multiple resources across multiple processes and machines. This approach has clear advantages, as sharing the infrastructure leads to lower costs (e.g. electricity), and companies of all sizes like to share infrastructure in order to achieve lower operational costs.

However, providing performance guarantees and isolation in multi-tenant systems is extremely hard, resource management systems must avoid mismatching the resource allocation, as tenant-generated requests compete with each other and with the system generated tasks. Furthermore, tenant workload can change in unpredictable ways depending on the input workload, the workload of other tenants in the system, and the underlying topology.

\subsubsection{Control}

Control refers to how resource management systems allocates tasks to available resources, there are two alternatives towards performing resource allocations: either \textit{centralized} or \textit{decentralized}.

\textbf{Centralized control} consists in a centralized component with a global view of the state of the system making all decisions regarding resource allocations. Intuitively, given that a centralized component generates manages all the resources in the system, this component can easily enforce policies to achieve the desired performance guarantees or fairness goals by identifying and only throttling the tenants or system activities responsible for resource bottlenecks \cite{verma2015large}.

\textbf{Decentralized control} architectures are defined by having the decision-making process regarding resource allocations distributed across multiple components \cite{Hong2019}. This topic has yet not been subject to much research, although it is of extreme relevance towards edge environments. For example, if the system is globally distributed, it may take too long for a centralized controller to identify hotspots in a certain zone and load-balance them.

One of the key challenges in distributed resource management is ensuring that the components which perform resource assignments do not conflict with each other. Additionally, in a multi-tenant decentralized resource management system, tenants may request resources to different resource controllers in the system, and if they do not coordinate themselves, the application may be provisioned with too many (or too little) resources.

\subsection{Resource Management Systems}

\textbf{Mesos} \cite{hindman2011mesos} is a multi-tenant centralized resource sharing platform that attempts to provide fine-grained resource sharing within a data centre. The tenants for this platform are frameworks such as HDFS \cite{borthakur2008hdfs}, MapReduce \cite{dean2008mapreduce}, among others, which in turn support multiple applications running within a DC. In short, the Mesos resource sharing system consists of a \textit{master} process which manages \textit{slave} daemons running on each cluster node. In order to achieve fault-tolerance for the master component, Mesos employs Zookeeper \cite{hunt2010zookeeper} to maintain replicas, elect a new master, and transfer state to a new master in case the active master fails.

The master implements fine-grained sharing of resources across frameworks by employing \textit{resource offers}, which consist of lists containing free resources distributed among slaves. The master makes decisions about how many resources to offer to each framework, and the decision-making process is based on an arbitrary organizational policy, such as fair sharing or priority. Each framework that wishes to use Mesos must implement a \textit{scheduler} and an \textit{executor}. The scheduler registers with the Mesos master to receive resource offers, and the executor is the process that is launched on slave nodes to run the framework's tasks.

A limitation of the Mesos resource sharing platform is that it is not scalable, given the central component issuing resource allocations (the original authors mention the system scales up to 50000 slave daemons on 99 physical machines), which is not enough for an edge environment. Furthermore, the resource offer model forces frameworks to employ a specific programming model based on schedulers and executors, which we believe to be too restrictive.

\textbf{Yarn} (Yet Another Resource Negotiator) \cite{Vavilapalli2013ApacheHY} is a centralized multi-tenant resource sharing platform that decouples the programming model from the resource management infrastructure and delegates many scheduling functions to per-application components. The architecture of YARN is composed by: a per-cluster Resource Manager (RM), multiple Application Masters (AM), and Node Managers (NM). The Resource Manager (RM) tracks resource usage and node liveness, enforces allocation invariants and arbitrates contention among tenants.

Application Masters (AM) run arbitrary user code, their duties in the system consist of managing the lifecycle aspects, including dynamically increasing and decreasing resource consumption, managing the flow of execution, and handling faults. Node Managers (NM) are worker daemons, whose responsibilities consist of managing container dependencies, monitoring their execution, and providing a set of services for them.

AMs send resource requests to the RM, containing the number of containers to request, the resources per container, locality preferences, and a priority level within the application. These requests are designed to capture the needs of applications while at the same time removing application concerns (such as task dependencies) from the scheduler. Because the RM is in charge of processing and scheduling all task distributions for each request made by AMs, it is effectively a \textit{monolithic} scheduler. By consequence, there is a unique point of failure, which makes this system inadequate for large scale edge environments.

\textbf{Omega} \cite{41684} is a scheduler designed for grid computing systems composed by schedulers and workers. Each scheduler receives large amounts of jobs composed by either one or many tasks that have to be scheduled among workers. Contrary to YARN, which is monolithic, OMEGA uses multiple schedulers per cluster, each with a shared global view of the cluster state.

Schedulers make task placement decisions according to their view of the cluster state and their scheduling policy. If two or more schedulers attempt to schedule a task to the the same worker (i.e., generating a conflict), the worker first tries to accommodate both tasks, if it cant, it rejects the least important one.

One advantage of OMEGA in relation to MESOS is that MESOS resource attributions ``lock'' the resources to the corresponding framework,  which means that only one framework is examining a resource at a time. While it achieves higher throughput in allocation operations, its main limitations are that: (1) in case the grid becomes overloaded, resource allocations can potentially start interfering with each other; (2) scheduling policies are harder to ensure; and finally, (3) all schedulers must have global knowledge of the system.

\textbf{Edge NOde Resource Management} \cite{wang2017enorm} (ENORM) is framework aimed at employing edge resources towards applications by provisioning and auto-scaling edge node resources. ENORM proposes a three-tier architecture: (1) the Cloud tier, where application servers are hosted; (2) the middle tier, where the edge nodes are situated; and (3) the bottom tier, where user devices (e.g. smartphones, wearables, gadgets) are situated.

To enable the use of edge nodes, ENORM deploys a cloud server manager on each application server, which communicates with potential edge nodes, requesting computing services. Using these computing resources, it deploys partitioned servers on the edge nodes. Edge nodes are maintained in a global view.

ENORM authors tested the designed system using an online game inspired on Pokemon GO (iPokemon)\cite{pokemonGo}. The ENORM framework partitions the game server and sends user data to each edge node containing information regarding the users within that geographical location. Users from the relevant geographical zone then connect to the edge server and are serviced by a geographically closer edge node as if they were connected to the data centre. Limitations from this framework are the large size of the required information to perform the deployments, and similarly to previous solutions, the lack of fault-tolerance and scalability, from employing a centralized component to perform monitoring and management of resources.

\textbf{FogTorch} \cite{Brogi2017} is a service deployment framework aimed at determining eligible deployments for an application over a given Fog infrastructure, modeled by: (1) Cloud Data Centers, denoted by their location and software capabilities; (2) Fog Nodes, that consist of tuples containing: the location, hardware, the software capabilities, and the things directly reachable from the fog node; (3) Things, which are represented by a tuple denoting the thing (sensor or actuator) location and its type; (4) QoS profiles, that are sets of QoS profiles composed by the latency and bandwidth of a communication link. (5) Applications, which are composed of independent sets of components, each with a set of requirements regarding QoS profiles, hardware and software capabilities, and things. Then, authors model service deployments as restrictions over the system model and employ a greedy heuristic, which reduces the search space of devices constituting options for these service deployments.

FogTorch is also the base for \textbf{FogTorchPI} \cite{brogi2017best}, which is a solution that employs the system model of FogTorch, however instead of a greedy approach, it uses Monte Carlo simulations to calculate the best possible deployment configurations.

These solutions provide a comprehensive system model which models many different types of application requirements, however,similarly to FogTorch, it requires an updated global view of the system, which requires collecting a large amount of information to a central entity, limiting system scalability.

\subsection{Discussion}

Although resource management systems have been present for many years, these are often tailored towards small scale environments composed by homogenous devices in stable environments, which contrast with the edge of the network, where devices are extremely numerous, operate on a decentralized fashion, and are highly heterogenous.

We argue that a centralized controller is not the ideal solution for an edge environment, given the fact that as the number of devices in the system increases, so does the number of resources to track, and the harder it is for a centralized component to have an up-to-date global view of the system.

Due to their low capacity, devices at the edge of the network are very susceptible to workload changes, for example, a 5G tower that is hosting services cannot handle a drastic increase in the number of users it is serving. In this scenario, we argue that in order to maintain pre-established performance criteria, devices must autonomously make resource management decisions such as scaling an allocation horizontally or vertically in order to quickly meet the demands of users/tenants.

% \subsubsection{Scheduling}

% There are many approaches towards scheduling resources among edge nodes: \textbf{Brute force} proposes exhaustively exploring all potential targets combinations towards offloading tasks (including the Cloud, the Edge and other user devices) and picking the one which provides the minimum execution time. This technique intuitively is not scalable enough to be applied in practice.

% \textbf{Greedy} heuristics focus on minimizing the time it takes for the task to be completed on a mobile device. FogTorch \cite{Brogi2017} employs a greedy heuristic by which reduces the search space of devices that constitute options for service deployments. 

% \textbf{Simulated Annealing} employs a search space based on the utilization of edge and cloud nodes, total costs, and the completion time of the task to find the optimal solution.

% \subsubsection{Offloading}

%In this section we study offloading, which is technique used by edge-enabled applications to fully take advantage of edge nodes.

%Offloading is a technique in which a server, application and the associated data are transferred onto another node in the network \cite{Hong2019}. There are two variants for offloading: either from the user device to the edge, or from the cloud to the edge. 

%Offloading from user device to the edge enhances computing in mobile nodes by employing edge nodes which are usually only one or two hops away. While offloading from the Cloud to the edge has the potential to reduce bandwidth consumption and improve QoS of edge-enabled applications. 

%\textbf{Server offloading} is a technique in which servers are offloaded to the edge via either replication or partitioning. \textbf{Replication} consists in offloading the full server state(e.g. a database or an  application server), while \textbf{partitioning} consists in offloading only a portion of the server state. 

%The portion of the server to offload must take into account a set of parameters such as latency, functionality, energy efficiency, geographical distribution, among others. 



\section{Aggregation} \label{sec:aggregation} 
Aggregation is an essential building block on modern distributed systems, it  enables the determination of important system wide properties in a decentralized manner \cite{DBLP:journals/corr/abs-1110-0725}. Aggregation consists in computing an aggregation function over a set of input values where each node has one input value. Common aggregation functions consist in sum, count, average, min, max, where each function presents different properties which must be considered.

Towards deploying services in edge devices, aggregation may be used to monitor the device and service state (e.g. computing the average latency of the closest available service that meets a certain criteria; counting nearby available computing resources that can be used to offload services, or identify hotspots by aggregating the average system load in certain areas). Given this, it is important to understand the challenges of aggregating values in a highly decentralized manner. There are two properties of aggregation functions: \textit{decomposable functions} and \textit{duplicate sensitive} functions.

\subsection{Properties of aggregation functions}

For some aggregation functions, we may need to involve all elements in the multiset, however, for memory and bandwidth issues, it is impractical to perform a centralized computation, hence, the aim is to employ \textit{in-transit computation}. In order to enable this, it is required that the aggregation function is \textbf{decomposable}. Intuitively, a decomposable aggregation function is one where a function may be composed defined as a composition of other functions. Decomposable functions may \textit{self-decomposable}, which intuitively means that the aggregated value is the same for all possible combinations of all sub-multisets partitioned in the multiset. This happens whenever the applied function is commutative and associative (e.g. min, max, sum, count). A canonical example of a decomposable function that is not self-decomposable is average, which consists in the sum of all pairs divided by the count of peers that contributed to the aggregation.

The second property of aggregation is \textbf{duplicate sensitiveness}, and it is related to wether a given value occurs several times in a multiset. Depending on the aggregation function used, the presence of repeated values may influence the result, it is said that a function is \textbf{duplicate sensitive} if the result of the aggregation function is influenced by the repeated values (e.g. SUM), conversely, if the aggregation function is \textbf{duplicate insensitive} it can be successfully repeated any number of times to the same multiset without affecting the result (e.g. MIN and MAX).

Table \ref{table:aggregation_functions} classifies popular aggregation functions in function of decomposability and duplicate sensitiveness as found in \cite{DBLP:journals/corr/abs-1110-0725}:

\begin{table}[]
    \begin{tabular}{|l|l|l|l|}
    \hline
                          & \multicolumn{2}{l|}{Decomposable} & Non-Decomposable  \\ \hline
                          & Self-decomposable    &                             &  \\ \hline
    Duplicate insensitive & Min, Max             & Range     & Distinct Count    \\ \hline
    Duplicate sensitive   & Sum, Count           & Average   & Median, Mode     \\ \hline
    \end{tabular}
    \caption{popular aggregation functions in function of decomposability and duplicate sensitiveness}
    \label{table:aggregation_functions}
\end{table}

Building on the concepts of duplicate sensitiveness and decomposability, we show that aggregation functions present their own particularities which dictate their applicability in particular scenarios. For example, a Min or Max function may be easier to implement with a simpler algorithm, while Sum, Count and Average require extra considerations. This presents a limitation towards calculating exact aggregations in large scale systems, to circumvent this, some systems do not require obtaining exact aggregated values to perform near optimally  (e.g. estimating the system size in order to select the optimal fanout for a gossip system only requires an estimation of the magnitude of the system). 

\subsection{Aggregation techniques}

Following, we present the studied categories of aggregation techniques: Hierarchical, Averaging, Sketches (hash or min-k based), Digests, Deterministic and Sampling. In each technique, we discuss its applicability in the edge environment.

\subsubsection{Hierarchical}

\textbf{Hierarchical} approaches leverage directly on the decomposability of aggregation functions. Aggregations from this class depend on the existence of a hierarchical communication structure, (e.g. a spanning tree) with one root (sink node). Aggregations take place by splitting inputs into groups and aggregating values bottom-up in the hierarchy. Commonly, hierarchical aggregation systems have nodes whose roles are \textit{aggregators} or \textit{forwarders}, intuitively, aggregators compute the aggregation functions forward results to forwarders who transfer results to upper levels in the hierarchy. In the absence of faults, the correct final result is obtained in the sink node. Many systems employ hierarchical approaches to aggregation, namely TAG \cite{}, DAG \cite{}, among others. Hierarchical approaches, due to taking advantage of device heterogeneity, are attractive in edge environments. However, due to the low computational power of devices, not all nodes may be  able to handle the additional overhead of maintaining the hierarchical topology.

\textbf{Averaging} aggregation consists in the continuous computation and exchanging of partial averages data among all active nodes in the aggregation process. In this type of systems, after a few rounds, all nodes usually converge to the correct value with high accuracy, as shown in \cite{gossip_aggregation}. This type of aggregation is attractive for gossip protocols, where nodes may employ varied gossip techniques to continuously share and update their values with random neighbors. Algorithms from this category are also attractive to use in edge environments, because they are accurate while employing random unstructured overlays, which retain their fault-tolerance and resilience to churn.

\textbf{Sketches} are fixed-size data structures that hold a \textit{sketch} of all network values. Multiple sketches are usually forwarded throughout the system, and nodes who forward sketches apply (usually commutative and associative) operations to update and merge them. \textcolor{red}{functioning and edge discussion}

% completar functionamento

\textbf{Digests} are an aggregation technique that gathers a representation of all system values, it supports complex aggregation functions such as Median and Mode. In short, algorithms employ a fixed-size data structures commonly composed of a set of values and associated counters) which compacts the data distribution (e.g. into a histogram). \textcolor{red}{edge discussion}

\textbf{Counting} algorithms target the same aggregation function: Count, algorithms from this class usually employ some randomized procedure to achieve a probabilistic approximation of the population size.

\subsection{Relevant aggregation protocols}

In this subsection we will analyze relevant aggregation protocolsthat ilustrate some techniques discussed above.

\textbf{TAG: Tiny AGgregation}\cite{Madden2002} is a service for aggregation in low-power, distributed, wireless sensor networks. TAG distributes queries in the network in a time and power-efficient manner by employing a hierarchical aggregation pattern. For each aggregation procedure, there is a \textit{root} nodes which broadcasts a message to start the tree-building process, each message contains two fields: a level and a an ID. Whenever a node without an assigned level receives a tree-building message, it assigns its own level as the message level plus one, and its own parent as the message sender. Then, it reassigns the level and ID to its own and forwards the message to other nodes. Then, whenever a node wishes to send a message to the root, it simply forwards the message bottom-up in the tree. The formed topology allows the computation of Count, Maximum, Minimum, Sum and Average. It is important to notice that the formed tree will be unbalanced as a function of the underlay latency and processing time.

\textbf{DECA} \cite{Artigas2006} \textcolor{red}{//TODO}

\textbf{Astrolabe} \cite{Renesse2003} \textcolor{red}{//TODO}

\textbf{SingleTree \cite{} and MultipleTree \cite{}} \textcolor{red}{//TODO}

\subsection{Discussion}

\section{Resource Discovery} \label{sec:res_discovery} \input{Chapters/related_work/resource_discovery.tex}

\section{Offloading computation to the Edge} \label{sec:offloading_computation} \subsubsection{Decentralizing clouds}

\subsubsection{Fog Computing}

\subsubsection{Edge Computing}

\subsubsection{Osmotic Computing}



\chapter{Related Work} \label{cha:related_work}

\section{Context}

The following chapter provides context about to the challenge we attempt to solve. We present the sub-challenges identified towards performing management of microservices in the edge of the network. 

First, \textbf{microservice networking}: microservices need to cooperate towards solving applicational tasks, as such, peers in the system need to be integrated in an efficient abstraction layer (e.g. overlay) that allows them to find each other and communicate. This raises an old challenge in P2P computing: how can peers organize themselves in the network such that they can find resources (either other peers, services or even computing power) in an efficient way? It is important to notice that the edge environment is composed by lots of sub-networks composed of heterogenous devices concurrently entering and leaving the network, which is a hard scenario in \textbf{resource location systems}, particularly the underlying \textbf{topology management} challenge, because the overlay needs to adapt to the underlying network to remain efficient. 

Given this, section \ref{sec:topology_management} of this chapter provides context about \textbf{topology management}, particularly the main categories of topologies, how they can be evaluated, and a discussion about their applicability in edge environments. Second section, \ref{sec:res_location} studies \textbf{resource location architectures}, which leverage on the different types of topologies studied in \ref{sec:topology_management} to index resources in the system. For each type of architecture we discuss their differences and present popular implementations in the state of the art.

The next challenge is \textbf{maintaining quality of service} of microservices. When attempting to maintain quality of service we need to perform \textbf{monitoring} of device and service status, particularly of devices in the edge environment. Section \ref{sec:res_monitoring} covers popular techniques used in tracking device and service status, we particularly study which metrics to collect to aid in pinpointing causes of QoS degradation, and how to perform failure detection in edge systems, in the same section we discuss popular implementations of monitoring systems. 

Then, when federating and tracking massive amounts of monitoring data, transmitting and storing it becomes a limitation in the scalability of the system. To circumvent this, that data needs to be \textbf{aggregated}. Section \ref{sec:aggregation} studies aggregation, aggregation consists in the process of combining several numeric values into one single representative. We discuss the different types of aggregation, and how they can be applied towards maintaining quality of service.

Lastly, section \ref{sec:offloading_computation} studies how the aggregated results can be used to perform microservice management and deploymen. Namely how to orchestrate microservices such that the computation is offloaded to the Edge. We discuss popular paradigms such as Fog Computing and Osmotic Computing and how they compare to Edge Computing. Lastly, we discuss approaches towards implementing elastic computing in Edge environments.

\section{Topology Management} \label{sec:topology_management} 
% resource management in the DC
% -------------------
% Resource Management
% -------------------

Resource location systems are one of the most common applications of the P2P paradigm \cite{leitaoPHDthesis}, in a resource location system, a participant provided with a resource descriptor is able to query other peers and obtain an answer to the location (or absence) of that resource in the system within a reasonable amount of time. To do so, resource location systems employ search strategies, which depend on : (1) the structure the an overlay network (structured or unstructured). (2) on the characteristics of the resources to search (e.g. if there are many copies of it or not), and (3) on the desired results (e.g. if a single copy of a resource satisfies the query, or multiple are required). 

In the context of resource management, if a peer wishes to offload computations to other peers, it must employ an efficient search strategy to find nearby available resources (e.g., storage capacity, computing power, among others) in order to offload computations. In this section, we cover resource location and discovery, starting with the taxonomy of querying techniques for P2P systems, followed by the study of how resources can be stored or indexed and looked up throughout the topologies studied in the previous section.

\subsection{Querying techniques}

Querying techniques consist of how peers describe the resources they need, these, according \cite{leitaoPHDthesis}, may be classified as: \textbf{(1)~Exact Match queries}, these specify the resource to search by the value of a unique attribute (i.e., an identifier, commonly the hash of the value of the resource); the second querying methodology type is \textbf{(2)~keyword queries}, that employ one or more keywords (or tags) combined with logical operators to describe resources (e.g. "pop", "rock", "pop and rock" ...); next, \textbf{(3)~range queries} retrieve all resources whose value is contained within a given interval (e.g. "movies with 100 to 300 minutes of duration"); finally, \textbf{(4)~arbitrary queries} aim to find a set of nodes or resources that satisfy one or more arbitrary conditions (e.g. looking for a set of resources encoded in a certain format).

Provided with a way of describing their resource needs, peers need strategies to index and retrieve the resources in the system, there are three popular techniques: \textbf{centralized}, \textbf{distributed over an unstructured overlay}, or \textbf{distributed over a structured overlay}.

\subsection{Centralized Resource Location}

\textbf{Centralized resource location} relies on one (or a group of) centralized peers that index all existing resources. This type of architecture greatly reduces the complexity of systems, as peers only need to contact a subset of nodes to locate resources. 

It is important to notice that in a centralized architecture, while the indexation of resources is centralized, the resource access may still be distributed (e.g. a centralized server provides the addresses of peers who have the files, and files are obtained in a pure P2P fashion), a system which employs this architecture with success is BitTorrent \cite{cohen2003incentives}.

Although centralized architectures are widely used nowadays, they lack the necessary scalability to index the large number of dynamic resources we intend to manage, and have limited fault tolerance to failures, making them unsuited for edge environments. 

%However, there are many ways that a hybrid architecture can be applied to Edge computing: since the failure rate of a single data center (DC) is low, if we assume a system composed by multiple DCs, they may act as a reliable failover for whenever edge devices are partitioned of fail. 

\subsection{Resource Location on Unstructured Overlays}

When employing an unstructured overlay for resource location, the resources are scattered throughout all peers in the system, consequently, peers need to employ distributed search strategies to find the intended resources. This is accomplished through disseminating messages containing these queries throughout the overlay. The dissemination of these messages can follow multiple strategies, we now cover there two popular approaches: \textbf{flooding} and \textbf{random walks} \cite{leitaoPHDthesis}. 

\textbf{Flooding} consists of peers eagerly forwarding queries to others in the system as soon as they receive them for the first time, the objective of flooding is to contact multiple distinct peers that may have the queried resource. One approach is \textbf{complete flooding}, which consists in contacting every node in the system, this guarantees that if the resource exists, it will be found. However, complete flooding is not scalable and incurs significant message redundancy. \textbf{Flooding with limited horizon} minimizes the message overhead by attaching a TTL to messages that limits the number of times a message can be retransmitted. However, there is a trade-off for efficiency: flooding with limited horizon does not guarantee that all resources will be found. 

\textbf{Random Walks} are a dissemination strategy that attempts to minimize the communication overhead that is associated with flooding. A random walk consists of a message with a TTL that is randomly forwarded one peer at a time throughout the network. Random walks may also attempt to bias their path towards peers that are more likely to have answers to the query \cite{1022239}, this technique is commonly reffered to in the literature as a \textbf{random guided walk}. A common approach to bias random walks is to use bloom filters \cite{5751342}, which are space-efficient probabilistic data structures that allow the creation of imprecise distributed indexes for resources.

First generation of decentralized resource location systems relied on unstructured overlays (such as Gnutella \cite{gnutella_gtk}) and employed simple broadcasts with limited horizon to query other peers in the system. However, as the size of the system grew, simple flooding techniques lacked the required scalability for satisfying the rising number of queries, which triggered the emergence of new techniques to reduce the number of messages per query, called \textbf{super-peers}. 

\textbf{Super-peers} are peers which are assigned special roles in the system (often chosen in function of their capacity or stability). In the case of resource location systems, super-peers disseminate queries throughout the system. This technique is at the core of solutions such as Gia \cite{Chawathe2003}, employed towards effectively reducing the number of peers that have to disseminate queries on the second version of Gnutella \cite{gnutella_gtk}. 

\textbf{SOSP-Net} \cite{garbacki2007optimizing} (Self-Organizing Super-Peer Network) proposes a resource location system composed by regular peers and super-peers that effectively employs feedback concerning previous queries to improve the overlay network. Weak peers maintain links to super-peers which are biased based on the success of previous queries, and super-peers bias the routing of queries by taking into account the semantic content of each query. 

However, even with super-peers, one problem that still remains in these systems is finding very rare resources, which requires flooding the entire overlay. To circumvent this, the third generation of resource location systems rely on Distributed Hash Tables to ensure that even rare resources in the system can be found within a limited number of communication steps.

\subsection{Resource Location on Distributed Hash Tables}

Resource location on structured overlays is often done by relying on the applicational routing capabilities of distributed Hash Tables (DHTs). In a DHT, peers use hash functions to generate node identifiers (IDS) often uniformly distributed over the ID space. Then, by employing the same hash function to generate resource IDs, and assigning a portion of the ID space to each node, peers are able to map resources to the responsible peers in a bounded number of steps, which makes them very suitable for (\textbf{exact match queries}) \cite{leitaoPHDthesis}. 

One particular type of DHT that is commonly employed in small-sized resource location systems is the One-Hop Distributed Hash Table (DHT), nodes in a one-hop DHT have full membership of the system, and can locally map resources to known peers, thus performing lookups in O(1) time and message complexity. Facebook's Cassandra \cite{lakshman2010cassandra} and Amazon's Dynamo \cite{decandia2007dynamo} are widely used implementations of one-hop DHTs. 

There are two popular techniques for storing resources in a DHT, the first approach is to store the resources locally and publish the location of the resource in the DHT. This way, the node responsible for the resource's key only stores the locations of other nodes in the system and the resource may be replicated among distinct nodes composing the system. The second technique consists of transferring the resource to the responsible node in the DHT, although fewer nodes must keep the same value. It is, however, important to mention that this way the resources are not replicated, provided that with consistent hashing, all nodes with the same resource will publish the resource in the same location of the DHT.

\subsection{Discussion}

As mentioned previously, we believe centralized resource location systems are unsuited for edge environments, given that as previously mentioned in section \ref{sec:context}, for our goal, centralizing the computation (for example in data-centers) will eventually lead to a bottleneck for the system scalability. Furthermore, these types of systems are plagued with a single point of failure, making them unsuitable for volatile environments. \todo {improve}

Unstructured resource location systems are attractive for systems that perform queries in search for resources with multiple copies or for range queries, however, this approach is inefficient when performing exact match queries, as a finding the exact resource in an unstructured resource location system requires flooding the entire system with messages.

Conversely, distributed hash tables are specially tailored towards exact match queries, but are less robust to churn and are subject to low-capacity nodes being a bottleneck in routing procedures. 

% In the context of the proposed solution, given that the resources we intend to manage are present in all nodes (e.g., computing power, memory, among others), we believe unstructured resource location is better suited for our needs. For example, if an edge device wishes to find nearby computing resources to offload a certain task, it may employ a random walk. On the other hand, if a peer wishes to find a larger set of computing resources to deploy multiple application components, it may employ flooding techniques. 

%Hybrid approaches between  simultaneously ensure load balancing properties and address  problems related to churn and 

%\subsection{Hybrid approaches}

%\textbf{Curiata} \& \textbf{Build One Get One Free}

% TODO falar de surrogate routing
%\textcolor{red}{surrogate routing}

%\paragraph{\textbf{Viceroy} }

%\paragraph{\textbf{Koala} }

% TODO falar de otimizacoes fixes observadas:
% lazyness a montar a rede (usar mensagens de servicos)
% manter peers antigos (churn inversamente proporcional a uptime)
% formar grupos para reduzir routing (increased background communication)
% ao usar prefix routing consegue-se logb(n) routing
% Xor-distance vs numeric distance (unidirectionality)
% Pedidos assincronos para fazer queries mais rapidas
% usar um algoritmo para dar "feed" ao outro (gossip + dht)




\section{Resource Location and Discovery} \label{sec:res_location} 
% resource management in the DC
% -------------------
% Resource Management
% -------------------

Resource location systems are one of the most common applications of the P2P paradigm \cite{leitaoPHDthesis}, in a resource location system, a participant provided with a resource descriptor is able to query other peers and obtain an answer to the location (or absence) of that resource in the system within a reasonable amount of time. To do so, resource location systems employ search strategies, which depend on : (1) the structure the an overlay network (structured or unstructured). (2) on the characteristics of the resources to search (e.g. if there are many copies of it or not), and (3) on the desired results (e.g. if a single copy of a resource satisfies the query, or multiple are required). 

In the context of resource management, if a peer wishes to offload computations to other peers, it must employ an efficient search strategy to find nearby available resources (e.g., storage capacity, computing power, among others) in order to offload computations. In this section, we cover resource location and discovery, starting with the taxonomy of querying techniques for P2P systems, followed by the study of how resources can be stored or indexed and looked up throughout the topologies studied in the previous section.

\subsection{Querying techniques}

Querying techniques consist of how peers describe the resources they need, these, according \cite{leitaoPHDthesis}, may be classified as: \textbf{(1)~Exact Match queries}, these specify the resource to search by the value of a unique attribute (i.e., an identifier, commonly the hash of the value of the resource); the second querying methodology type is \textbf{(2)~keyword queries}, that employ one or more keywords (or tags) combined with logical operators to describe resources (e.g. "pop", "rock", "pop and rock" ...); next, \textbf{(3)~range queries} retrieve all resources whose value is contained within a given interval (e.g. "movies with 100 to 300 minutes of duration"); finally, \textbf{(4)~arbitrary queries} aim to find a set of nodes or resources that satisfy one or more arbitrary conditions (e.g. looking for a set of resources encoded in a certain format).

Provided with a way of describing their resource needs, peers need strategies to index and retrieve the resources in the system, there are three popular techniques: \textbf{centralized}, \textbf{distributed over an unstructured overlay}, or \textbf{distributed over a structured overlay}.

\subsection{Centralized Resource Location}

\textbf{Centralized resource location} relies on one (or a group of) centralized peers that index all existing resources. This type of architecture greatly reduces the complexity of systems, as peers only need to contact a subset of nodes to locate resources. 

It is important to notice that in a centralized architecture, while the indexation of resources is centralized, the resource access may still be distributed (e.g. a centralized server provides the addresses of peers who have the files, and files are obtained in a pure P2P fashion), a system which employs this architecture with success is BitTorrent \cite{cohen2003incentives}.

Although centralized architectures are widely used nowadays, they lack the necessary scalability to index the large number of dynamic resources we intend to manage, and have limited fault tolerance to failures, making them unsuited for edge environments. 

%However, there are many ways that a hybrid architecture can be applied to Edge computing: since the failure rate of a single data center (DC) is low, if we assume a system composed by multiple DCs, they may act as a reliable failover for whenever edge devices are partitioned of fail. 

\subsection{Resource Location on Unstructured Overlays}

When employing an unstructured overlay for resource location, the resources are scattered throughout all peers in the system, consequently, peers need to employ distributed search strategies to find the intended resources. This is accomplished through disseminating messages containing these queries throughout the overlay. The dissemination of these messages can follow multiple strategies, we now cover there two popular approaches: \textbf{flooding} and \textbf{random walks} \cite{leitaoPHDthesis}. 

\textbf{Flooding} consists of peers eagerly forwarding queries to others in the system as soon as they receive them for the first time, the objective of flooding is to contact multiple distinct peers that may have the queried resource. One approach is \textbf{complete flooding}, which consists in contacting every node in the system, this guarantees that if the resource exists, it will be found. However, complete flooding is not scalable and incurs significant message redundancy. \textbf{Flooding with limited horizon} minimizes the message overhead by attaching a TTL to messages that limits the number of times a message can be retransmitted. However, there is a trade-off for efficiency: flooding with limited horizon does not guarantee that all resources will be found. 

\textbf{Random Walks} are a dissemination strategy that attempts to minimize the communication overhead that is associated with flooding. A random walk consists of a message with a TTL that is randomly forwarded one peer at a time throughout the network. Random walks may also attempt to bias their path towards peers that are more likely to have answers to the query \cite{1022239}, this technique is commonly reffered to in the literature as a \textbf{random guided walk}. A common approach to bias random walks is to use bloom filters \cite{5751342}, which are space-efficient probabilistic data structures that allow the creation of imprecise distributed indexes for resources.

First generation of decentralized resource location systems relied on unstructured overlays (such as Gnutella \cite{gnutella_gtk}) and employed simple broadcasts with limited horizon to query other peers in the system. However, as the size of the system grew, simple flooding techniques lacked the required scalability for satisfying the rising number of queries, which triggered the emergence of new techniques to reduce the number of messages per query, called \textbf{super-peers}. 

\textbf{Super-peers} are peers which are assigned special roles in the system (often chosen in function of their capacity or stability). In the case of resource location systems, super-peers disseminate queries throughout the system. This technique is at the core of solutions such as Gia \cite{Chawathe2003}, employed towards effectively reducing the number of peers that have to disseminate queries on the second version of Gnutella \cite{gnutella_gtk}. 

\textbf{SOSP-Net} \cite{garbacki2007optimizing} (Self-Organizing Super-Peer Network) proposes a resource location system composed by regular peers and super-peers that effectively employs feedback concerning previous queries to improve the overlay network. Weak peers maintain links to super-peers which are biased based on the success of previous queries, and super-peers bias the routing of queries by taking into account the semantic content of each query. 

However, even with super-peers, one problem that still remains in these systems is finding very rare resources, which requires flooding the entire overlay. To circumvent this, the third generation of resource location systems rely on Distributed Hash Tables to ensure that even rare resources in the system can be found within a limited number of communication steps.

\subsection{Resource Location on Distributed Hash Tables}

Resource location on structured overlays is often done by relying on the applicational routing capabilities of distributed Hash Tables (DHTs). In a DHT, peers use hash functions to generate node identifiers (IDS) often uniformly distributed over the ID space. Then, by employing the same hash function to generate resource IDs, and assigning a portion of the ID space to each node, peers are able to map resources to the responsible peers in a bounded number of steps, which makes them very suitable for (\textbf{exact match queries}) \cite{leitaoPHDthesis}. 

One particular type of DHT that is commonly employed in small-sized resource location systems is the One-Hop Distributed Hash Table (DHT), nodes in a one-hop DHT have full membership of the system, and can locally map resources to known peers, thus performing lookups in O(1) time and message complexity. Facebook's Cassandra \cite{lakshman2010cassandra} and Amazon's Dynamo \cite{decandia2007dynamo} are widely used implementations of one-hop DHTs. 

There are two popular techniques for storing resources in a DHT, the first approach is to store the resources locally and publish the location of the resource in the DHT. This way, the node responsible for the resource's key only stores the locations of other nodes in the system and the resource may be replicated among distinct nodes composing the system. The second technique consists of transferring the resource to the responsible node in the DHT, although fewer nodes must keep the same value. It is, however, important to mention that this way the resources are not replicated, provided that with consistent hashing, all nodes with the same resource will publish the resource in the same location of the DHT.

\subsection{Discussion}

As mentioned previously, we believe centralized resource location systems are unsuited for edge environments, given that as previously mentioned in section \ref{sec:context}, for our goal, centralizing the computation (for example in data-centers) will eventually lead to a bottleneck for the system scalability. Furthermore, these types of systems are plagued with a single point of failure, making them unsuitable for volatile environments. \todo {improve}

Unstructured resource location systems are attractive for systems that perform queries in search for resources with multiple copies or for range queries, however, this approach is inefficient when performing exact match queries, as a finding the exact resource in an unstructured resource location system requires flooding the entire system with messages.

Conversely, distributed hash tables are specially tailored towards exact match queries, but are less robust to churn and are subject to low-capacity nodes being a bottleneck in routing procedures. 

% In the context of the proposed solution, given that the resources we intend to manage are present in all nodes (e.g., computing power, memory, among others), we believe unstructured resource location is better suited for our needs. For example, if an edge device wishes to find nearby computing resources to offload a certain task, it may employ a random walk. On the other hand, if a peer wishes to find a larger set of computing resources to deploy multiple application components, it may employ flooding techniques. 

%Hybrid approaches between  simultaneously ensure load balancing properties and address  problems related to churn and 

%\subsection{Hybrid approaches}

%\textbf{Curiata} \& \textbf{Build One Get One Free}

% TODO falar de surrogate routing
%\textcolor{red}{surrogate routing}

%\paragraph{\textbf{Viceroy} }

%\paragraph{\textbf{Koala} }

% TODO falar de otimizacoes fixes observadas:
% lazyness a montar a rede (usar mensagens de servicos)
% manter peers antigos (churn inversamente proporcional a uptime)
% formar grupos para reduzir routing (increased background communication)
% ao usar prefix routing consegue-se logb(n) routing
% Xor-distance vs numeric distance (unidirectionality)
% Pedidos assincronos para fazer queries mais rapidas
% usar um algoritmo para dar "feed" ao outro (gossip + dht)




\section{Monitoring} \label{sec:res_monitoring} 
% \subsection{Device Monitoring}

% A particularly hard problem in resource monitoring is fault detection, given the need to ensure each component is monitored by at least one non-faulty component, even in the face of joins, leaves, and failures of both nodes as well as network infrastructure. Most fault-detectors rely on heartbeats, which consist of a peer sending a message periodically to another peer in order to signal that it is functioning correctly.

% \textcite{leitao2008large} proposes a decentralized device monitoring system by employing Hyparview \cite{Hyparview} as a decentralized monitoring fault detector, given the fixed number of active connections, which ensures overlay connectivity, each peer will have at least another non-faulty component monitoring it through the active TCP connection. 

% In addition to tracking device health, it is paramount to collect metrics regarding the operation of the device, such as: \textbf{(1)~Network~related~metrics}: devices need to be interconnected across an underlying infrastructure which is continuously changing. This raises concerns about the network link quality between devices across the system, especially if they are running time-critical services. Given this, it is paramount to track network related metrics such as bandwidth, latency and link status. \textbf{(2)~Memory related metrics:} either related to volatile memory or persistent memory, it is important to track the amount of free and used memory. \textbf{(3) CPU metrics}: the utilization of the CPU (e.g., user, sys, idle, wait).

% \subsection{Container Monitoring}

% As previously mentioned, containers are the solution which incurs less overhead when it comes to sharing resources in the same node, given this, we now study tools which monitor the status of containers and the applications executing inside them. 

% \textbf{Docker} \cite{docker} has a built tool called \textbf{Docker Stats} \cite{docker_stats} which provides a live data stream of metrics related to running containers. It provides information about the network I/O, CPU and memory usage, among others. 

% \textbf{Container Advisor} \cite{cAdvisor} (cAdvisor) is a service which analyzes and exposes both resource usage and performance data from running containers. The information it collects consists of resource isolation parameters, historical resource usage and network statistics. cAdvisor includes native support for Docker containers and supports a wide variety of other container implementations.

% \textbf{Agentless System Crawler}  (ASC) \cite{cloudviz_2019} is a monitoring tool with support for containers that collects monitoring information including performance metrics, system state, and configuration information. It provides the ability to build two types of plugins: function  plugins for on-the-fly data aggregation or analysis, and output plugins for target monitoring and analytics endpoints.

% There are many other tools that offer the ability to continuously collect metrics about running services/, however,


In this section, we will cover \textbf{resource monitoring}, which consists in tracking the state of certain aspects of a system, such as the device status, the capacity of links between devices, the status of available resources within a given geographical zone, among others, which is paramount for making effective management decisions regarding task allocations and managing the overlay network. However, if every node were to continuously collect, store and process the metrics of other nodes, the amount of communication and processing needed to do this would quickly overload the system. Consequently, there is the need to reduce the size of the data through a process called \textit{aggregation}.

\subsection{Aggregation}

Aggregation consists in the determination of important, system-wide properties and it is an essential building block towards monitoring distributed systems \cite{akosThesis} \cite{DBLP:journals/corr/abs-1110-0725}. This technique can be employed, for example, towards computing the average of available computing resources in a certain part of the network or towards identifying application hotspots by aggregating the average resource usage in certain areas, among many other uses. There are two properties of aggregation functions: \textit{decomposability} and \textit{duplicate sensitiveness}.

\subsubsection*{Decomposability}

A decomposable aggregation function is one where a function may be defined as a composition of other functions. Decomposable functions may be \textbf{self-decomposable}, where the aggregated value is the same for all possible combinations of all sub-multisets partitioned in the multiset. This happens whenever the applied function is commutative and associative (e.g. min, max, sum, count). A canonical example of a decomposable function that is not self-decomposable is average, which consists of the sum of all pairs divided by the count of peers that contributed to the aggregation. For non-decomposable aggregations, we need to involve all elements in the multiset. These are less desirable to perform in a large scale system, as the number of input values is large, and gathering all the input values may incur additional networking costs.

%  As our focus is on \textit{decentralized aggregation}, which is only possible to do if the aggregation function is \textbf{decomposable}. 

\subsubsection*{Duplicate sensitiveness}

The second property of aggregation is \textbf{duplicate sensitiveness}, and it is related to whether a given value can or cannot occur several times in a multiset, as depending on the aggregation function, the presence of repeated values may influence the result. It is said that a function is \textbf{duplicate sensitive} if the result of the aggregation function is influenced by the repeated values (e.g. SUM). Conversely, if the aggregation function is \textbf{duplicate insensitive}, it can be successfully repeated any number of times to the same multiset without affecting the result (e.g. MIN and MAX).

Table \ref{table:aggregation_functions} classifies popular aggregation functions in function of decomposability and duplicate sensitiveness as found in \cite{DBLP:journals/corr/abs-1110-0725}.

\begin{table}[]
    \centering
    \resizebox{0.60\linewidth}{!}{%
    \begin{tabular}{|c|c|c|c|}
    \hline
                          & \multicolumn{2}{c|}{Decomposable} & Non-Decomposable  \\ \hline
                          & Self-decomposable    &                             &  \\ \hline
    Duplicate insensitive & Min, Max             & Range     & Distinct Count    \\ \hline
    Duplicate sensitive   & Sum, Count           & Average   & Median, Mode     \\ \hline
    \end{tabular}}
    \caption{Decomposability and duplicate sensitiveness of aggregation functions}
    \label{table:aggregation_functions}
\end{table}

%Building on the concepts of duplicate sensitiveness and decomposability, we show that aggregation functions present their own particularities which dictate their applicability in particular scenarios. For example, a Min or Max function may be easier to implement with a simpler algorithm, while Sum, Count and Average require extra considerations. 

%This presents a limitation towards calculating exact aggregations in large scale systems, to circumvent this, some systems do not require obtaining exact aggregated values to perform near optimally (e.g. estimating the system size in order to select the optimal fanout for a gossip system only requires an estimation of the magnitude of the system). 

\subsection{Aggregation techniques}

In the following subsection, we provide context about the taxonomy of aggregation techniques:

\subsubsection*{Hierarchical aggregation}

\textbf{Tree-based} approaches leverage directly on the decomposability of aggregation functions. Aggregations from this class depend on the existence of a hierarchical communication structure (e.g. a spanning tree) with one root ( also called the sink node). Aggregations take place by splitting inputs into groups and aggregating values bottom-up in the hierarchy.  Tree-based architectures also allow efficient multi-tree aggregation, which consists in the calculation of an aggregation result through the exchange of partial averages data among all active nodes in the aggregation process \cite{akosThesis}. 

%Commonly, tree-based systems have nodes whose roles are \textit{aggregators} or \textit{forwarders}, intuitively, aggregators compute the aggregation functions and forward results to forwarders who then retransmit the results to upper levels in the hierarchy. In the absence of faults, the correct final result is obtained in the sink node.

\textbf{Cluster-based} techniques rely on clustering the nodes in the network according to a certain criterion (e.g. latency, energy efficiency). Then, within each cluster, a representative is responsible for local aggregation and for transmitting the results to other representatives. 

Hierarchical approaches, due to taking advantage of device heterogeneity, are attractive in edge environments. However, due to the low computational power of devices, not all nodes may be able to handle the additional overhead of maintaining the hierarchical topology, furthermore, there are additinal concerns regarding failures when compared to ad-hoc aggregation.


\subsubsection*{Ad-hoc aggregation}

Ad-hoc aggregation consists of a class of aggregation algorithms that calculate aggregations through periodic, randomized exchanges of messages. These types of algorithms allow an estimation of an aggregated value high accuracy while employing unstructured overlays \cite{gossip_aggregation}, consequently, these retain the fault tolerancee and resilience to churn from these overlays.

%\textbf{Sketches} are fixed-size data structures that hold a \textit{sketch} of all network values. Multiple sketches are usually forwarded throughout the system, and nodes who forward sketches apply (usually commutative and associative) operations to update and merge them.

%\textbf{Digests} are an aggregation technique which gathers a representation of all system values, it supports complex aggregation functions such as Median and Mode. In short, algorithms employ a fixed-size data structures commonly composed of a set of values and associated counters) which compacts the data distribution (e.g. into a histogram).

%\textbf{Counting} algorithms target the same aggregation function: Count, algorithms from this class usually employ some randomized procedure to achieve a probabilistic approximation of the population size.

%\subsection{Relevant aggregation protocols}

%In this subsection we will analyze relevant aggregation protocols that illustrate some techniques discussed above.

%\subsubsection{TAG: Tiny AGgregation}

%\textbf{TAG: Tiny AGgregation}\cite{Madden2002} is a service for aggregation in low-power, distributed, wireless sensor networks. TAG distributes queries in the network in a time and power-efficient manner by employing a hierarchical aggregation pattern. For each aggregation procedure, there is a \textit{root} nodes which broadcasts a message to start the tree-building process, each message contains two fields: a level and a an ID. Whenever a node without an assigned level receives a tree-building message, it assigns its own level as the message level plus one, and its own parent as the message sender. Then, it reassigns the level and ID to its own and forwards the message to other nodes. Then, whenever a node wishes to send a message to the root, it simply forwards the message bottom-up in the tree. The formed topology allows the computation of Count, Maximum, Minimum, Sum and Average. It is important to notice that the formed tree will be unbalanced as a function of the underlay latency and processing time.

%\subsubsection{SingleTree} 

%\textbf{SingleTree} \cite{} \textcolor{red}{//TODO}

%\subsubsection{MultipleTree} 

%\textbf{MultipleTree} \cite{} \textcolor{red}{//TODO}

%\subsubsection{DECA} \textbf{DECA} \cite{Artigas2006} \textcolor{red}{//TODO}

\subsection{Monitoring systems}

Provided with this overview of aggregation techniques, we now discuss popular monitoring systems in the literature. For each system, we discuss what we believe to be their advantages and drawbacks as solutions for edge settings.

\textbf{Astrolabe} \cite{Renesse2003} is a distributed information management platform that aims at monitoring the dynamically changing state of a collection of distributed resources. It introduces a hierarchical architecture defined by zones, where a zone is recursively defined to be either a host or a set of non-overlapping zones. Each zone (minus the root zone) has a local identifier, which is unique within the zone where it is contained. Zones are globally identified by their \textit{zone name}, which consists of the concatenation of all zone identifiers within the path from the root to the zone in question.

Associated with each zone there is a Management Information Base (MIB) containing attributes relative to that zone. These attributes are not directly writable, instead, they are generated by aggregation functions contained in special entries in the MIB. Leaf zones are the excepted from these restrictions, instead containing \textit{virtual child zones} which are directly writable by devices within that virtual child zone.

The aggregation functions which produce the MIBs are contained in \textit{aggregation function certificates} (AFCs). These contain a user-programmable SQL function, a timestamp and a digital signature. In addition to the function code, AFCs may contain other information, such as an \textit{Information Request AFC}, that specifies which information to retrieve from each participating host, and how to summarize the retrieved information. Alternatively, we may have a \textit{configuration AFC}, used for specifying runtime parameters that applications may use for dynamic configuration.

Astrolabe employs gossip exchanges to update the MIBs, which provides an eventual consistency model: if updates cease to exist for a long enough time, all the elements of the system converge towards the same state. This is achieved by employing a gossip algorithm that selects another agent at random and exchanges zone state with it. If the agents are within the same zone, they exchange information relative to their zone. Conversely, if agents are in different zones, they exchange information relative to the zone which is their least common ancestor.

Not all nodes gossip information, within each zone, a node is elected (the authors do not specify how) to perform gossip on behalf of that zone. Additionally, nodes can represent nodes from other zones, in this case, nodes run one instance of the gossip protocol per represented zone, where the maximum number of zones a node can represent is bounded by the number of levels in the Astrolabe tree.

An agents' zone is defined by its system administrator, which is a potential limitation towards scalability, given that configuration errors have the potential of heavily raising system latency and reducing traffic locality. Additionally, the authors state that the size of gossip messages scales with the branching factor, often exceeding the maximum size of a UDP packet. Other limitations which arise from using Astrolabe are the high memory requirements per participant due to the high degree of replication, and the potential points of failure of the representatives of zones.

\textbf{Ganglia} \cite{massie2004ganglia} is a distributed monitoring system for high performance computing systems, namely clusters and grids. In short, Ganglia groups nodes in clusters, in each cluster, there are representative cluster nodes that federate devices and aggregate internal cluster state. Then, representatives aggregate information in a tree of point-to-point connections.

Ganglia relies on IP multicast to perform intra-cluster aggregation, it is mainly designed to monitor infrastructure monitoring data about machines in a high-performance computing cluster. Given this, its applicability is limited towards edge environments: (1) clusters are assumed to be in stable environments, which contrasts with the edge environment; (2) it relies on IP multicast, which has been proven not to hold in a number of cases; (3) has no mechanism to prevent network congestion; finally, (4) it requires manual configuration of the tree structure.

\textbf{SDIMS} \cite{SDIMS} (Scalable Distributed Information Management System) proposes a combination of techniques employed in Astrolabe \cite{Renesse2003} and distributed hash tables (in this case, Pastry \cite{rowstron2001pastry}). It is based on an abstraction that exposes the underlying \textbf{aggregation trees} provided by a DHT such as Pastry. 

Given a key $k$, an \textbf{aggregation tree} is defined by the union of the routing paths from all nodes to the node responsible for key $k$, where each routing step along the path to $k$ corresponds to a level in the aggregation tree. \textbf{Aggregation functions} are associated with an attribute type and a name and rooted at \textit{hash(attribute type, attribute name)}, which results in different attributes with the same function being aggregated along trees rooted in different parts of the DHT, enabling load-balancing.

This achieves communication and memory efficiency when compared to gossip-based approaches, because MIBs have a lesser degree of replication. However, as each node belongs to every aggregation tree, this could potentially hinder scalability in edge settings, given that low-capacity nodes may become overloaded if they are intermediate aggregation points in all aggregation trees. 

\textbf{Prometheus} \cite{prometheus} is an open-source monitoring and alerting toolkit originally built for recording any purely numeric time series. We believe this tool is one of the most popular tools in the state-of-the-art in regard to querying and collecting multi-dimensional data collections. This solution uses a ``pull'' technique to aggregate metrics, which means it scrapes targets periodically to obtain its metric values. To do so, it requires a configuration file that dictates many aspects of its behaviour, such as the targets for scraping metric values, the periodicity at which to perform this scrape, how long to retain metrics in the database, among other aspects. Furthermore, Prometheus also allows the configuration of alarms that trigger (configurable) actions whenever a given criterion is met. 

Finally, Prometheus also allows federation, which consists of a server scraping selected time-series from another Prometheus server. Federation is split in two categories, \textit{hierarchical federation} and \textit{cross-service federation}. In \textit{hierarchical federation}, Prometheus servers are organized into a topology resembling a tree, where each server aggregates aggregated time-series data from a larger number of subordinated servers. Alternatively,  \textit{cross-service federation} enables scraping selected data from another service's Prometheus server to enable alerting and queries against both datasets within a single server. 

\subsection{Discussion}

After the study of the literature related to monitoring systems, we believe there is a lack of monitoring systems targeted towards edge settings, as popular existing solutions often have centralized points of failure, rely on manual configuration or depend on techniques such as IP multicast, which make them unsuited for large-scale dynamic systems such as the ones found in edge environments.

Furthermore, we argue that large-scale monitoring systems purely based on distributed hash tables \cite{SDIMS} are unsuitable for edge environments, provided these assume all nodes have an equal capacity, which we believe to mismatch the heterogeneity of edge environments. Other  alternatives that better align with our objectives, such as Astrolabe \cite{Renesse2003} (given it can be configured with device heterogeneity in mind), require heavy amounts of message exchanges to keep information up-to-date and require manual configuration of the hierarchical tree, which is also be undesirable, provided the dynamicity of these environments. 

%\subsection{End-to-end link monitoring}

%Given that the edge infrastructure envisions cooperation from all devices in the path from the origin of the data to the DC, devices need to be interconnected across an underlying infrastructure which is continuously changing. This raises concerns about the network quality of links between devices across the system, especially if they are running time-critical services. 

%It is paramount to analyze how to monitor and improve link quality, for providing traffic locality, latency, among others. According to the literature \cite{}, the most popular metrics to analyze are:

%\begin{enumerate}

    %\item Network throughput, which is the average rate of successful data transfer through a network connection.
    
    %\item Latency, which consists in how long a packet takes to travel across a link from one endpoint to another
    
    %\item Packet loss, which consists in how many packets are lost when traveling towards their destination.
%\end{enumerate}

\section{Aggregation} \label{sec:aggregation} 
A basic requirement towards enabling service deployment in the edge is aggregation. Aggregation consists in 

Efficient aggregation protocols can be employ

\subsection{Types of aggregation}

\subsection{Relevant aggregation protocols}


\section{Offloading computation to the Edge} \label{sec:offloading_computation} \subsubsection{Decentralizing clouds}

\subsubsection{Fog Computing}

\subsubsection{Edge Computing}

\subsubsection{Osmotic Computing}



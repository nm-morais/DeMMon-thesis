%!TEX root = ../template.tex
%%%%%%%%%%%%%%%%%%%%%%%%%%%%%%%%%%%%%%%%%%%%%%%%%%%%%%%%%%%%%%%%%%%%
%% chapter2.tex
%% NOVA thesis document file
%%
%% Chapter with the template manual
%%%%%%%%%%%%%%%%%%%%%%%%%%%%%%%%%%%%%%%%%%%%%%%%%%%%%%%%%%%%%%%%%%%%
\chapter{ThesisDIFCTNL User's Manual}
\label{cha:users_manual}

% ================
% = Introduction =
% ================
\section{Introduction} % (fold)
\label{sec:introduction}


% section introduction (end)

% ====================
% = Folder Structure =
% ====================
\section{Folder Structure} % (fold)
\label{sec:folder_structure}

\noindent
\begin{tabularx}{\linewidth}{>{\ttfamily}r@{~/~}>{\ttfamily}l>{\itshape}l>{\upshape}X}
ul     & ist    & folder  & 
The folder for the \href{http://www.tecnico.ulisboa.pt}{\emph{Instituto Superior Técnico}} of the \emph{University of Lisbon}.
\\
\end{tabularx}

% section folder_structure (end)

% ===================
% = Package options =
% ===================
\section{\novathesisclass\ Class Options} % (fold)
\label{sec:package_options}

The \novathesis\ class can be customized with the options listed below.

\newcommand{\classoption}[3]{\textbf{#1=OPT}\qquad #2\\\qquad\emph{#3}\\}

\noindent
\begin{ctabular}{@{}p{\linewidth}@{}}
  \toprule
  \classoption{docdegree}%
    {phd(*), phdplan, phdprop, msc, mscplan, bsc}%
    {The type of the document: PhD Thesis (default), PhD Plan, PhD Proposal, MSc Disseration, MSc Plan, BSc Report}
    \midrule
  
    \bottomrule
\end{ctabular}

\section{Additional considerations about the class options} % (fold)
\label{sec:additional_considerations}

In this section we will provide some additional considerations about some of the customizations available as class options.

\subsection{The main language} % (fold)
\label{sub:the_main_language}

The choice of the main language with the option “\texttt{lang=OPT}” affects:

\subsection{Class of Text} % (fold)
\label{sub:class_of_text}

You must choose the class of text for the document. The available options are:

\begin{enumerate}
	\item \textbf{bsc} --- BSc graduation report.
	\item \textbf{*mscplan} --- Preparation of MSc dissertation. This is a preliminary report graduate students at DI-FCT-NOVA must prepare to conclude the first semester of the two-semesters MSc work. The files specified by \verb!\dedicatoryfile! and \verb!\acknowledgmentsfile! are ignored, even if present, for this class of document.
	\item \textbf{msc} --- MSc dissertation.
	\item \textbf{phdprop} ---  Proposal for a PhD work. The files specified by \verb!\dedicatoryfile! and \verb!\acknowledgmentsfile! are ignored, even if present, for this class of document.
	\item \textbf{prepphd} ---  Preparation of a PhD thesis. This is a preliminary report PhD students at DI-FCT-NOVA must prepare before the end of the third semester of PhD work. The files specified by \verb!\dedicatoryfile! and \verb!\acknowledgmentsfile! are ignored, even if present, for this class of document.
	\item \textbf{phd} --- PhD dissertation.
\end{enumerate}
% subsection class_of_text (end)

% ============
% = Printing =
% ============
\subsection{Printing} % (fold)
\label{sub:printing}

You must choose how your document will be printed. The available options are:
\begin{enumerate}
	\item \textbf{oneside} --- Single side page printing.
	\item \textbf{*twoside} --- Double sided page printing.
\end{enumerate}
% subsection printing (end)

% =============
% = Font Size =
% =============
\subsection{Font Size} % (fold)
\label{ssec:font_size}

You must select the encoding for your text. The available options are:
\begin{enumerate}
	\item \textbf{11pt} --- Eleven (11) points font size.
	\item \textbf{*12pt} --- Twelve (12) points font size. You should really stick to 12pt\ldots
\end{enumerate}
% subsection font_size (end)

% =================
% = Text encoding =
% =================
\subsection{Text Encoding} % (fold)
\label{ssec:text_encoding}

You must choose the font size for your document. The available options are:
\begin{enumerate}
	\item \textbf{latin1} --- Use Latin-1 (\href{http://en.wikipedia.org/wiki/ISO/IEC_8859-1}{ISO 8859-1}) encoding.  Most probably you should use this option if you use Windows;
	\item \textbf{utf8} --- Use \href{http://en.wikipedia.org/wiki/UTF-8}{UTF8} encoding.    Most probably you should use this option if you are not using Windows.
\end{enumerate}
% subsection font_size (end)

% ============
% = Examples =
% ============
\subsection{Examples} % (fold)
\label{ssec:examples}

Let's have a look at a couple of examples:

\begin{itemize}
	\item Preparation of PhD thesis, in portuguese, with 11pt size and to be printed single sided (I wonder why one would do this!)\\
	\verb!\documentclass[prepphd,pt,11pt,oneside,latin1]{thesisdifct-nova}!
	\item MSc dissertation, in english, with 12pt size and to be printed double sided\\
	\verb!\documentclass[msc,en,12pt,twoside,utf8]{thesisdifct-nova}!
\end{itemize}
% subsection examples (end)

\section{How to Write Using \LaTeX} % (fold)
\label{sec:how_to_write_using_latex}

Please have a look at Chapter~\ref{cha:a_short_latex_tutorial_with_examples}, where you may find many examples of \LaTeX constructs, such as Sectioning, inserting Figures and Tables, writing Equations, Theorems and algorithms, exhibit code listings, etc.

% section how_to_write_using_latex (end)



\section{Exmaple glossary and acronyms}
%
% \todo[inline]{A a note in a line by itself.}
%
This is the first occurrence of an abbreviation: \gls{abbrev}. And now the second occurrence of the same abbreviation: \gls{abbrev}. And a new acronym with capital letter: \Gls{xpt} and reused \gls{xpt}.  Let's also use a few other acronyms such as \gls{aaa}, \gls{aab}, \gls{aba}, \gls{bbb} and \gls{xpt}.


Lets add the term ``\gls{computer}'' to the glossary!
%
% Please note that
% \begin{center}
%   \textbf{\large this package and template are not official for FCT/NOVA}.
% \end{center}

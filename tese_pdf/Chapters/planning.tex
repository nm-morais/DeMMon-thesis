%!TEX root = ../template.tex
%%%%%%%%%%%%%%%%%%%%%%%%%%%%%%%%%%%%%%%%%%%%%%%%%%%%%%%%%%%%%%%%%%%%
%% chapter4.tex
%% NOVA thesis document file
%%
%% Chapter with lots of dummy text
%%%%%%%%%%%%%%%%%%%%%%%%%%%%%%%%%%%%%%%%%%%%%%%%%%%%%%%%%%%%%%%%%%%%
\chapter{Planning} \label{cha:planning}

In this Chapter we start by briefly pr

%%%%%%%%%%%%%%%%%%%%%%%%%%%%%%%%%%%%%%%%%%%%%%%%%%%%%%%%%%%%%%%%%%
\section{Proposed Solution}
\label{cha:proposed_sol}

To achieve this, we propose to create a new novel algorithm which employs a hierarchical topology that resembles the device distribution of the Edge Infrastructure. This topology is created by assigning a level to each device and leveraging on gossip mechanisms to build a structure resembling a FAT-tree 

The levels of the tree will be determined by \textcolor{red}{...undecided...} and will the tree be used to employ efficient aggregation and search algorithms. Each level of the tree will be composed by many devices that form groups among themselves, the topology of the groups \textcolor{red}{...undecided...} 

The purpose of this algorithm is to allow:

\begin{enumerate} 
    \item Efficient resource monitoring to deploy services on.
    \item Offloading computation from the cloud to the Edge and vice-versa through elastic management of deployed services.
    \item Efficient service discovery
    \item Federate large amount of heterogeneous devices and use heterogeneity as an advantage for building the topology.
\end{enumerate}

We plan to research existing protocols (both for topology management and aggregation) and enumerate their trade-offs along with how they behave across different environments. Then, employ a combination of different techniques in a unique way to monitor the topology.

% \cite{DBLP:journals/corr/abs-1805-06989}

\section{Scheduling}

%!TEX root = ../template.tex
%%%%%%%%%%%%%%%%%%%%%%%%%%%%%%%%%%%%%%%%%%%%%%%%%%%%%%%%%%%%%%%%%%%%
%% abstrac-en.tex
%% NOVA thesis document file
%%
%% Abstract in English
%%%%%%%%%%%%%%%%%%%%%%%%%%%%%%%%%%%%%%%%%%%%%%%%%%%%%%%%%%%%%%%%%%%%

% What's the problem?
% Why is it interesting?
% What's the solution?
% What follows from the solution?

The centralized model proposed by the Cloud computing paradigm mismatches the decentralized nature of mobile and IoT applications, given the fact that most of data production and consumption is performed by devices outside of the data center (DC). Transporting the data to and from the data center increases the infrastructure costs for service providers, the latency for the end users, and also raises security and privacy concerns. 

The aforementioned limitations have lead us into a post-cloud era where a new computing paradigm arose: Edge Computing. Edge Computing takes into account the broad spectrum of devices residing outside of the data center as potential targets for computations, however, as edge devices tend to have restricted capacity and computational power, there is the need for edge devices to effectively share resources in order to accomplish tasks which otherwise would be impossible for a single edge device. 

The study of the art revealed that there is a lack of resource sharing systems for edge environments, as existing ones are commonly tailored for homogenous devices integrated in stable environments, which contrasts with the edge environment. In this work we propose to address this problem by presenting an infrastructure tailored for resource sharing in edge devices. This infrastructure federates large numbers of devices and continuously collects and aggregates information regarding the operation of devices and applications in a decentralized manner. This information in turn will be used towards allowing edge-enabled applications to adapt to changes in the environment either by offloading tasks, replicating the application, or migrating towards more suitable devices. 
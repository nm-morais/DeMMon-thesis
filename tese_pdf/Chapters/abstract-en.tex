%!TEX root = ../template.tex
%%%%%%%%%%%%%%%%%%%%%%%%%%%%%%%%%%%%%%%%%%%%%%%%%%%%%%%%%%%%%%%%%%%%
%% abstrac-en.tex
%% NOVA thesis document file
%%
%% Abstract in English
%%%%%%%%%%%%%%%%%%%%%%%%%%%%%%%%%%%%%%%%%%%%%%%%%%%%%%%%%%%%%%%%%%%%

% What's the problem?
% Why is it interesting?
% What's the solution?
% What follows from the solution?

The centralized model proposed by the Cloud computing paradigm mismatches the decentralized nature of mobile and IoT applications, given the fact that most of data production and consumption is performed by devices outside of the data center. Serving data from and performing most of computations on cloud data centers increases the infrastructure costs for service providers and the latency for end users, while also raising security and privacy concerns. 

The aforementioned limitations have led us into a post-cloud era where a new computing paradigm arose: Edge Computing. Edge Computing takes into account the broad spectrum of devices residing outside of the data center as potential targets for computations. However, as edge devices tend to have heterogenous capacity and computational power, there is the need for them to effectively share resources and coordinate to accomplish tasks which would otherwise be impossible for a single edge device. 

The study of the state of the art has revealed that existing resource tracking and sharing solutions are commonly tailored for homogenous devices deployed on a single stable environment, which are inadequate for dynamic edge environments. In this work, we propose to address these limitations by presenting a novel solution for resource tracking and sharing in edge settings. This solution aims to federate large numbers of devices and continuously collect and aggregate information regarding their operation and
execution of deployed applicational components in a decentralized manner. This will enable edge-enabled applications, decomposed in components, to adapt to runtime environmental changes by either, offloading tasks, replicating, or migrating components.

%This information in turn will be used towards allowing edge-enabled components to adapt to changes in the environment either by offloading tasks, replicating, or migrating the components to more suitable devices. 

\begin{keywords}

    Edge Computing, Resource Management, Resource Monitoring, Resource Location, Topology Management
    
\end{keywords}
% to add an extra black line

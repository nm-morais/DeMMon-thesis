%!TEX root = ../template.tex
%%%%%%%%%%%%%%%%%%%%%%%%%%%%%%%%%%%%%%%%%%%%%%%%%%%%%%%%%%%%%%%%%%%%
%% abstrac-en.tex
%% NOVA thesis document file
%%
%% Abstract in English
%%%%%%%%%%%%%%%%%%%%%%%%%%%%%%%%%%%%%%%%%%%%%%%%%%%%%%%%%%%%%%%%%%%%

% What's the problem?
% Why is it interesting?
% What's the solution?
% What follows from the solution?

The centralized model proposed by the Cloud computing paradigm mismatches the decentralized nature of mobile and IoT applications, given the fact that most of data production and consumption is performed by devices outside of the data center (DC). Serving data from and performing most of computations on cloud data centers increases the infrastructure costs for service providers and the latency for the end users, while also raising security and privacy concerns. 

The aforementioned limitations have lead us into a post-cloud era where a new computing paradigm arose: Edge Computing. Edge Computing takes into account the broad spectrum of devices residing outside of the data center as potential targets for computations, however, as edge devices tend to have restricted capacity and computational power, there is the need for them to effectively share resources in order to accomplish tasks which otherwise would be impossible for a single edge device. 

The study of the state of the art has revealed that there is a lack of resource tracking and sharing solutions that are adequate for edge environments, as existing ones are commonly tailored for homogenous devices deployed on a single stable environment, which contrasts with the edge environment. In this work we propose to address this problem by presenting a novel solution for resource tracking and sharing in edge settings. This solution federates large numbers of devices and continuously collects and aggregates information regarding their operation and of components deployed on them in a decentralized manner. This information in turn will be used towards allowing edge-enabled components to adapt to changes in the environment either by offloading tasks, replicating or or migrating the components to more suitable devices. 
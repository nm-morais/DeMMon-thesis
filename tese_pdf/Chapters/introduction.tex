%!TEX root = ../template.tex
%%%%%%%%%%%%%%%%%%%%%%%%%%%%%%%%%%%%%%%%%%%%%%%%%%%%%%%%%%%%%%%%%%%
%% chapter1.tex
%% NOVA thesis document file
%%
%% Chapter with introduciton
%%%%%%%%%%%%%%%%%%%%%%%%%%%%%%%%%%%%%%%%%%%%%%%%%%%%%%%%%%%%%%%%%%%
\newcommand{\novathesis}{\emph{novathesis}}
\newcommand{\novathesisclass}{\texttt{novathesis.cls}}


\chapter{Introduction}
\label{cha:introduction}

Nowadays, the construction and operation of large-scale data-centers in low-cost locations decreased cost in electricity,
network infrastructure and the multiplexing of hardware at scale has enabled the establishment of the Cloud Computing paradigm.
This paradigm is currently the standard for the deployment and management of services, and has proven to have massive economic 
benefits which make it very likely to remain permanent in future of the computing landscape. 
It provides the ilusion of unlimited resources available to services, and has changed the way developers,
users and businesses rationalize about  applications \cite{Armbrust:2010:VCC:1721654.1721672}.
Currently, most software present in our everyday life such as Google Apps,
Amazon, Twitter, among many others is deployed on some form of cloud service. 

Although the Internet of Things (IoT) has been present in the community since 1999, and has been already employed in many
industries, the rise popularity of mobile applications and IoT applications differs from the centralized model proposed
by the Cloud Computing paradigm. When all computations reside in the data center (DC) and far  
from the source of the data, problems arise: from the need of physical space to contain all the infrastructure,
the increasing ammount of bandwidth needed to support the information exchange from the DC to the client,
the latency in communication between the DC and the client, among others. 

The accumulation of the aforementioned factors has directed us into a post-cloud era where
 a new computing paradigm emerged, Edge Computing. 
This paradigm takes into consideration all the computing and network resources along
the path from the data source to the DC \cite{iot_journal_shi_weisong_and_cao}.

Since the data is effectively being consumed closer to where it is being produced,
decentralizing computation has many benefits for end-users and business owners:
decentralization enables new novel applications that take into account the locality of computations for functionality,
heavily reduces the ammount of data that effectively needs to be transfered to the DC
for permanent storage, .....

 % \cite{DBLP:journals/corr/abs-1805-06989}


%!TEX root = ../template.tex
%%%%%%%%%%%%%%%%%%%%%%%%%%%%%%%%%%%%%%%%%%%%%%%%%%%%%%%%%%%%%%%%%%%
%% chapter1.tex
%% NOVA thesis document file
%%
%% Chapter with introduction
%%%%%%%%%%%%%%%%%%%%%%%%%%%%%%%%%%%%%%%%%%%%%%%%%%%%%%%%%%%%%%%%%%%
\newcommand{\novathesis}{\emph{novathesis}}
\newcommand{\novathesisclass}{\texttt{novathesis.cls}}



\chapter{Introduction}
\label{cha:introduction}

\section{Context}

Nowadays, the Cloud Computing paradigm is the standard for development, deployment and management of services, most software present in our everyday life such as Google Apps, Amazon, Twitter, among many others is deployed on some form of cloud service. This paradigm has proven to have massive economic benefits that make it very likely to remain permanent in future of the computing landscape. Cloud Computing provides the illusion of unlimited computing power, which and has revolutionized the way developers, users and businesses rationalize about building and deploying applications \cite{10.1145/1721654.1721672}.

However, the rise in popularity of mobile applications and IoT applications differs from the centralized model proposed by the Cloud Computing paradigm. With recent advances in the IoT industry, it is safe to assume that in the future almost all consumer electronics will play a role in producing and consuming data. However, when computation resides in the data center (DC), far from the source of the data, problems arise: from the physical space needed to contain all the infrastructure, the increasing amount of bandwidth needed to support the information exchange from the DC to the client, the latency in communication from the client to the DC as well as the security aspects that arise from offloading data storage and computation.

The aforementioned aspects have directed us into a post-cloud era where a new computing paradigm emerged, \textbf{Edge Computing}. Edge computing takes into consideration all the computing and network resources which act as an "edge" along the path between the data source and the DC. It addresses the increasing need for supporting interaction between cloud computing systems and mobile or IoT applications \cite{iot_journal_shi_weisong_and_cao}, and allows the emergence of novel edge-enabled applications (e.g. traffic management, smart city management, mobile games, among others).

Additionally, systems that require real-time processing of data may be not be feasible with cloud computing, Google's self-driving car generates 1 Gigabyte every second \cite{datafloq}, while a Boeing 787 will create around 5 gigabytes of data per second \cite{finnegan_2013}. If this data were to be processed in real-time (e.g. towards self-driving), it would be infeasible to transport it to cloud and back. 

\section{Motivation}

When accounting for all the devices that are external to the DC, we are met by a huge increase in heterogeneity of devices: from Data Centers to private servers, desktops and mobile devices to 5G towers and ISP servers, among others. Contrary to the cloud, edge environments tend to be highly dynamic, devices have constrained computational power and their connections are often limited in capacity and reliability. 

Developing an efficient general compute platform for edge environments is still an open challenge in Edge Computing. A crucial requirement towards this is performing \textbf{resource management}, which consists of keeping track of the tasks to perform and manage the utilization of computational resources of each device. Resource management systems are extensibly applied and optimized towards managing clusters (e.g. Mesos \cite{hindman2011mesos}, Yarn \cite{Vavilapalli2013ApacheHY}), however, those solutions are tailored towards small numbers of homogenous resource-heavy devices, which is the opposite of the edge environment.

Instead, a resource monitoring solution for these environments must be capable of federating very large numbers of devices, while leveraging heterogeneity to build a hierarchical infrastructure which combines naturally with the device taxonomy. An infrastructure capable of performing the aforementioned tasks successfully has strong applicability in todays world (e.g. Cloud providers, Smart Cities, among others). 

A particularly hard task in resource management is \textbf{scheduling}, which consists in distributing tasks among nodes in the system, ideally, the task distribution must promote a balanced resource usage among nodes in the system. One of the popular solutions is transporting all the data towards a centralized point and redistribute the tasks among nodes. However, this presents a centralized point of failure and a point of contention in a large scale system. 

Alternatively, nodes with only partial knowledge of the system (ideally without sending any additional messages) must be able to autonomously offload tasks towards neighbors. However, offloading tasks of varied complexity in heterogeneous nodes is not an easy task. And the accuracy of the knowledge each peer has dictates how efficiently they can offload them such that the system  remains balanced in the face of environment changes. 

Given this, we believe that peers must integrate a robust decentralized \textbf{resource monitoring system}. \textcolor{red}{finish}

\section{Expected Contributions}

The expected contributions, as will be further detailed in section \ref{cha:proposed_sol}, derive from the aforementioned challenges:

\begin{itemize}

    \item Devise a decentralized monitoring infrastructure for edge devices which employs a topology tailored towards edge environments, and a combination of monitoring and aggregation techniques in a natural way which promotes load-balancing and networking locality.
    
    \item Evaluate the designed infrastructure through simulation (e.g. iFogSim or PeerSim) and compare the performance with similar systems.

    \item Design an experimental scenario to test the system under various scenarios. For this, we must design or adapt an existing scheduler and test the feasibility of the solution by deploying applications and collecting metrics about the overlay, the potential reduction in cost and the quality of service.
    
\end{itemize}

\section{Document structure}

This document is structured in the following manner:

\textbf{Chapter 2} focuses on the related work, first section covers edge computing in further detail, next we cover P2P systems and the different types of topology management protocols. Third section studies the different types of resource location architectures, namely how to efficiently find a specific peer in the system and common techniques towards performing efficient searches over networks composed by a large number of devices. Fourth section covers aggregation techniques and popular implementations of these systems \textcolor{red}{era para meter isto em monitoring?}. Finally, last section covers popular resource monitoring systems.

\textbf{Chapter 3} further explains the proposed contribution and proposes the work plan for the remainder of the thesis. 


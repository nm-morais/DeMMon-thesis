%!TEX root = ../template.tex
%%%%%%%%%%%%%%%%%%%%%%%%%%%%%%%%%%%%%%%%%%%%%%%%%%%%%%%%%%%%%%%%%%%
%% chapter1.tex
%% NOVA thesis document file
%%
%% Chapter with introduciton
%%%%%%%%%%%%%%%%%%%%%%%%%%%%%%%%%%%%%%%%%%%%%%%%%%%%%%%%%%%%%%%%%%%
\newcommand{\novathesis}{\emph{novathesis}}
\newcommand{\novathesisclass}{\texttt{novathesis.cls}}


\chapter{Introduction}
\label{cha:introduction}

\section{Context}

Nowadays, the Cloud Computing paradigm is the standard for development, deployment and management of services, 
it has proven to have massive economic benefits that make it very likely to remain permanent in future of the 
computing landscape. It provides the ilusion of unlimited resources available to services, and has changed the 
way developers, users and businesses rationalize about  applications \cite{Armbrust:2010:VCC:1721654.1721672}.
Currently, most software present in our everyday life such as Google Apps,
Amazon, Twitter, among many others is deployed on some form of cloud service. 

However, currently, the rise in popularity of mobile applications and IoT applications differs from 
the centralized model proposed by the Cloud Computing paradigm.
When all computations reside in the data center (DC), far from the source of the data, 
problems arise: from the physical space needed to contain all the infrastructure,
the increasingly ammount of bandwidth needed to support the information exchange from the DC to the client,
the latency in communication between to the DC, has directed us
 into a post-cloud era where a new computing paradigm emerged, Edge Computing. 

\section{Motivation}

Edge computing takes into consideration all the computing and network resources along
the path from the data source to the DC, it resolves the increasing need for supporting interaction
 between IoT applications and cloud computing systems \cite{iot_journal_shi_weisong_and_cao}.
 Since data is effectively being consumed closer to where it is being produced,
decentralizing computation has many benefits for end-users and business owners:
 novel applications that take into account the locality of computations for functionality,
 faster response times for end users, enabled through the deployment of microservices 
 that are composed and inter-connected over both edge and cloud infrastructures.

 Understanding how 

\section{Expected Contribution}



 % \cite{DBLP:journals/corr/abs-1805-06989}


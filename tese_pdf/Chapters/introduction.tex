%!TEX root = ../template.tex
%%%%%%%%%%%%%%%%%%%%%%%%%%%%%%%%%%%%%%%%%%%%%%%%%%%%%%%%%%%%%%%%%%%
%% chapter1.tex
%% NOVA thesis document file
%%
%% Chapter with introduction
%%%%%%%%%%%%%%%%%%%%%%%%%%%%%%%%%%%%%%%%%%%%%%%%%%%%%%%%%%%%%%%%%%%
\newcommand{\novathesis}{\emph{novathesis}}
\newcommand{\novathesisclass}{\texttt{novathesis.cls}}

\chapter{Introduction}
\label{cha:introduction}

\section{Motivation}

Nowadays, the Cloud Computing paradigm is the standard for development, deployment and management of services, consequently, most software present in our everyday life such as Google Apps, Amazon, Twitter, among many others is deployed on some form of cloud service. Cloud Computing provides the illusion of unlimited computing power, which revolutionized the way developers, users and businesses rationalize about building and deploying applications \cite{10.1145/1721654.1721672}.

However, the rise in popularity of mobile applications and IoT applications differs from the centralized model proposed by the Cloud Computing paradigm, with recent advances in the IoT industry, it is safe to assume that in the future almost all consumer electronics will play a role in producing and consuming data. However, when the computation resides in the data center (DC), far from the source of the data, problems arise: from the physical space needed to contain all the infrastructure, the increasing amount of bandwidth needed to support the information exchange from the DC to the client, the latency in communication from the client to the DC as well as the security aspects that arise from offloading data storage and computation have directed us into a post-cloud era where a new computing paradigm emerged, \textit{Edge Computing}.

Edge computing takes into consideration all the computing and network resources which act as an "edge"\ along the path between the data source and the DC \cite{Leitao2018}. It addresses the increasing need for supporting interaction between cloud computing systems and mobile or IoT applications \cite{iot_journal_shi_weisong_and_cao}, and allows the emergence of novel edge-enabled applications (e.g. traffic management, smart city management, mobile games, among others).

Additionally, systems that require real-time processing of data may be not be feasible with cloud computing, Google's self-driving car generates 1 Gigabyte every second \cite{datafloq}, while a Boeing 787 will create around 5 gigabytes of data per second \cite{finnegan_2013}. If this data were to be processed in real-time (e.g. towards self-driving), it would be infeasible to transport it to cloud and back. 

\section{Context}

Developing an efficient resource sharing platform for edge environments is still an open challenge in Edge Computing. A crucial requirement towards this is performing efficient \textbf{resource management}, which consists of keeping track of the tasks to perform and manage the utilization of computational resources of each device. General compute platforms are extensibly used in Cloud systems (e.g. Mesos \cite{hindman2011mesos}, Yarn \cite{Vavilapalli2013ApacheHY}), however, those solutions are tailored towards small numbers of homogenous resource-heavy devices, which mismatches the edge environment.

When accounting for all the devices that are external to the DC, we are met by a huge increase in heterogeneity of devices: from Data Centers to private servers, desktops and mobile devices to 5G towers and ISP servers, among others,
However, contrary to the cloud, edge environments tend to be highly dynamic, and devices have constrained computational power with connections that are often unreliable. 

To overcome this, a resource sharing solution for the edge must be capable of perform resource allocation schemes in order to dynamically improve the quality of service (QoS) of edge-enabled applications and maximize the resource usage of the system. An infrastructure capable of performing the aforementioned tasks successfully has very strong demand in todays world (e.g. Cloud providers, Smart Cities, among others). 

A particularly hard task towards the aforementioned system is \textbf{scheduling}, which consists in distributing tasks among nodes in the system, ideally, the task distribution must promote a balanced resource usage among nodes in the system. The most popular solution is transporting all the system information towards a centralized point and redistribute the tasks among nodes. However, this presents a centralized point of failure and a point of contention in a large scale system. 

Alternatively, nodes with only partial knowledge of the system (ideally even without sending any additional messages) must be able to autonomously offload tasks towards other devices. However, the accuracy and freshness of the \textbf{monitoring information} each peer has dictates how efficiently they can offload tasks such that the system remains balanced, and applications running on the infrastructure maintain quality of service. 

Given this, it is paramount that peers must integrate a robust decentralized \textbf{resource monitoring system} which tracks device and service metrics. This system must federate peers such that they leverage on heterogeneity to build a hierarchical infrastructure which combines naturally with the device taxonomy, and adapts to the environment changes. 

\section{Expected Contributions}

The expected contributions, as will be further detailed in chapter \ref{cha:planning}, arise from the aforementioned challenges, we will focus on creating a resource monitoring and management solution tailored for provisioning resources for edge-enabled applications. Given this, the contributions expected to arise from our work consist of: 

\begin{itemize}

    \item A novel algorithm which constructs an overlay tailored towards federating large numbers of edge devices.
    
    \item A lightweight monitoring solution which relies on the overlay structure to obtain information regarding applications’ operation and the load of edge resources, which can then be employed to perform adequate management decisions.
    
\todo[]{devo falar de algum resource management?}

    \item Design an experimental scenario for the system and collect metrics to evaluate the performance and correctness of designed components. Additionally, we may design or adapt an existing edge-enabled application to test the solution in a real world scenario by scheduling application deployments and tracking metrics about the overlay and the quality of service of the applications.
    
\end{itemize}

\section{Document structure}

This document is structured in the following manner:

\textbf{Chapter} \ref{cha:related_work} studies related work towards our goal: we begin by analyzing similar paradigms to Edge Computing, and the devices which compose these environments. Follwing, we cover popular strategies towards federating various devices in the same system, how to efficiently find a specific peer in such systems and common techniques towards performing efficient resource discovery over large numbers of devices. Lastly, we study monitoring of system resources and cover related work about resource sharing systems.

\textbf{Chapter} \ref{cha:planning} further explains the proposed contribution and proposes the work plan for the remainder of the thesis. 


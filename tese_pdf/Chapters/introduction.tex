%!TEX root = ../template.tex
%%%%%%%%%%%%%%%%%%%%%%%%%%%%%%%%%%%%%%%%%%%%%%%%%%%%%%%%%%%%%%%%%%%
%% chapter1.tex
%% NOVA thesis document file
%%
%% Chapter with introduciton
%%%%%%%%%%%%%%%%%%%%%%%%%%%%%%%%%%%%%%%%%%%%%%%%%%%%%%%%%%%%%%%%%%%
\newcommand{\novathesis}{\emph{novathesis}}
\newcommand{\novathesisclass}{\texttt{novathesis.cls}}


\chapter{Introduction}
\label{cha:introduction}

Nowadays, the Cloud Computing paradigm has been established as the standard for the deployment and management of services. 
However, when all computations reside in the data center (DC), new problems arise:
from the need of physical space to contain all the infrastructure, the increasing ammount of bandwidth needed support the 
information exchange from the DC to the client, the latency in communication between the DC and the client, among others. 
% quotation

To address the aforementioned problems, a new computational paradigm emerged: Edge Computing.
This paradigm englobates Data Centers as a subset of all devices that may be used for computation, 
and extends the computation to all the devices outside the DC.
 % quotation

 
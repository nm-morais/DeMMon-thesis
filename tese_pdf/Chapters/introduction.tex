%!TEX root = ../template.tex
%%%%%%%%%%%%%%%%%%%%%%%%%%%%%%%%%%%%%%%%%%%%%%%%%%%%%%%%%%%%%%%%%%%
%% chapter1.tex
%% NOVA thesis document file
%%
%% Chapter with introduciton
%%%%%%%%%%%%%%%%%%%%%%%%%%%%%%%%%%%%%%%%%%%%%%%%%%%%%%%%%%%%%%%%%%%
\newcommand{\novathesis}{\emph{novathesis}}
\newcommand{\novathesisclass}{\texttt{novathesis.cls}}


\chapter{Introduction}
\label{cha:introduction}

\section{Context}

Nowadays, the Cloud Computing paradigm is the standard for development, deployment and management of services, 
it has proven to have massive economic benefits that make it very likely to remain permanent in future of the 
computing landscape. It provides the ilusion of unlimited resources available to services, and has changed the 
way developers, users and businesses rationalize about  applications \cite{10.1145/1721654.1721672}.
Currently, most software present in our everyday life such as Google Apps,
Amazon, Twitter, among many others is deployed on some form of cloud service. 

However, currently, the rise in popularity of mobile applications and IoT applications differs from 
the centralized model proposed by the Cloud Computing paradigm.
When all computations reside in the data center (DC), far from the source of the data, 
problems arise: from the physical space needed to contain all the infrastructure,
the increasing ammount of bandwidth needed to support the information exchange from the DC to the client,
the latency in communication from the client to the DC as well as the security aspects that arise from offloading data storage and computation,
have directed us into a post-cloud era where a new computing paradigm emerged, Edge Computing. 

Edge computing takes into consideration all the computing and network resources that act as an "edge" along
the path between the data source and the DC and addresses the increasing need for supporting interaction between cloud 
computing systems and mobile or IoT applications \cite{iot_journal_shi_weisong_and_cao}. However, 
when accounting for all the devices that are external to the DC, we are met by a huge increase in hetereogeneity of devices: 
from Data Centers to private servers, desktops and mobile devices to 5G towers and ISP servers, among others. 

\section{Motivation}

The aforementioned hetereogeneity implies that there is a broad spectrum of
computational, storage and networking capabilities along the edge of the network that
can be leveraged upon to perform different types of computation that rely on the individual
characteristics of the devices performing the tasks, which can vary from
from generic computations to aggregation, summarization, and filtering of data. \cite{DBLP:journals/corr/abs-1805-06989}

% exemplos reais

To fully materialize the Edge Computing paradigm, tools must be developed that allow the
federation and search over a large number of heteregenous devices. These tools must leverage on efficient
topologies and aggregation protocols.

\section{Expected Contribution}



 % \cite{DBLP:journals/corr/abs-1805-06989}


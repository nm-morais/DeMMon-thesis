

In this chapter we provide context about related work towards decentralizing computation from the Cloud. First, we define how we characterize Edge Computing and discuss how it is related to edge-related paradigms. Following, we study the taxonomy of the environment and focus on which computations each device can perform.

\subsection{Edge Computing}

As previously mentioned, edge computing calls for the processing of data in the edge of the network, specifically servicing IoT devices and performing computations on behalf of cloud services. This proposes to address battery life constraints, lower bandwidth consumption (which in turn lowers infrastructure costs) as well as address data privacy and safety \cite{7488250}. 

Many approaches have already leveraged on some form of Edge computing in the past. \textbf{Cloudlets} \cite{10.1145/2307849.2307858} are an extension of the cloud computing paradigm beyond the DC, which consists of deploying resource rich computers near the vicinity of users that provide cloud functionality.

A limitation of cloudlets is that because they are specialized computers, they cannot not guarantee low-latency ubiquitous service provision, and consequently cannot satisfy QoS of large hotspots of users. Cloudlets have become a trending subject, and have been employed towards resource management, Big Data analytics, security, among others.

\textbf{Content Distribution networks} \cite{peng2004cdn} (CDNs) emerged to address the overwhelming utilization of network bandwidth and server capacity that arose with bandwidth-intensive content (e.g. streaming HD video). In short, CDNs consist of specialized high bandwidth servers strategically located at the edge of the network, these servers replicate content from a certain origin and serve it at reduced latencies, effectively decentralizing the content delivery. 

Additionally, many paradigms have emerged which propose to solve similar problems to the Edge Computing paradigm. \textbf{Fog Computing} \cite{bonomi2012fog} proposes to provide compute, storage and networking services between end devices and traditional cloud computing data centers, typically, but not exclusively located at the edge of the network. Fog computing is interchangeable with our vision of all devices acting as an "edge" contributing towards computing, however, with a bigger emphasis on providing infrastructure towards sensor networks that produce large amounts of data, and less in how to properly integrate the whole infrastructure. 

\textbf{Osmotic Computing} \cite{villari2016osmotic} envisions the automatic deployment and management of inter-connected microservices on both edge and cloud infrastructures. Osmotic computing envisions edge devices employing an orchestration technique similar to the process of "osmosis". Translated, this consists in dynamically detecting and resolving resource contention via coordinated microservice deployments, furthermore, this paradigm is focused towards ensuring and maintaining quality of service.

\textbf{Multi-access edge computing} (MEC) formerly known as mobile-edge cloud computing, is a network architecture that proposes to provide fast-interactive responses for mobile applications. It solves this by employing the network edge (e.g. base stations and access points) to provide compute resources for latency-critical mobile applications \cite{mobile_edge_cloud}. MEC is a subset of our edge computing vision, although with a higher focus on communications technology and how to offload the computation from mobile to the cloud and not vice-versa. 

\subsubsection{Edge Environment Taxonomy} \label{subsec:edge_taxonomy}

Similar to \cite{Leitao2018}, we classify edge device according to 3 main attributes: \textbf{capacity} refers to computational, storage and connectivity capabilities of the device,  \textbf{availability} consists in the probability of a device being reachable, and finally, \textbf{domain} characterizes the way in which a device may be employed towards applications in short, if the device can support the whole \textit{applicational domain} or only the activities of a single user (\textit{user domain}) .

\begin{table}[!htb]
    \caption{Taxonomy of the edge environment}
    \begin{minipage}{.45\linewidth}
        \centering
        \resizebox{\columnwidth}{!}{%
        \begin{tabular}{|l|l|l|l|}
            \hline
            Level & Category & Availability & Capacity \\ \hline
            L0 & Cloud Data Centers & High & High \\ \hline
            L1 & ISP, Edge \& Private DCs & High & High \\ \hline
            L2 & 5G Towers & High & Medium \\ \hline
            L3 & Networking devices & High & Low \\ \hline
        \end{tabular}}
    \end{minipage} %
    \begin{minipage}{.45\linewidth}
        \centering
        \resizebox{\columnwidth}{!}{%
            \begin{tabular}{|l|l|l|l|}
                \hline
                Level & Category & Availability & Capacity \\ \hline
                L4 & Priv. Servers \& Desktops & Medium & Medium \\ \hline
                L5 &Laptops & Low & Medium \\ \hline
                L6 &Mobile devices & Low & Low \\ \hline
                L7 &Actuators \& Sensors & Varied & Low \\ \hline   
        \end{tabular}}
    \end{minipage} 
    \label{tab:taxonomy_edge}
\end{table}

Table \ref{tab:taxonomy_edge} shows the categories of edge devices, we assign levels to categories as a function of the distance from the cloud infrastructure. Coincidently, the levels are correlated to the number of devices and their computational power, where higher levels tend to have more devices that are closer to the origin of the data and have lower computational power.

\textbf{Levels 0 and 1} \textit{cloud and edge DCs} offer pools of computational and storage resources, that can dynamically scale to support the operation of edge-enabled applications (e.g. serving as an optimization reference point for edge devices). Both of these have high availability and large amounts of storage and computational power, as such, there is no limitations on which type of computations these devices can perform.

\textbf{Level 2} is composed of \textit{5G cell towers}, which serve as access points for mobile devices, \textbf{Level 3} also consists of \textit{networking devices}, although with lower capacity than those in level 2. Devices in both levels have high availability, and they can still contribute to the applicational domain, however in a limited fashion (e.g. coordinate resource management, host a microservice, or just act as a gateway for mobile devices).

\textbf{Level 4} Consists of \textit{private servers and desktops}, it is the first level where devices belong to the user domain. Devices in this level have medium capacity and availability, and can perform a varied amount of tasks on behalf of devices in higher levels (e.g. compute on behalf of smartphones, act as logical gateways or just cache data). 

\textbf{Level 5} consists of \textit{laptops}, which are also on the user domain, and can perform a role similar to devices in level 4, although with lower availability and capacity. The main differentiating factor with devices in level 4 is that laptops are battery-powered, which means that energy consumption must also be taken into account whenever monitoring and computing on these devices. 

\textbf{Level 6} consists of \textit{tablets and mobile devices}, devices in this level act as producers and consumers of data and belong to the user domain. Because because of their low capacity, availability and short battery life, mobile devices are limited in how they can perform computations and contribute towards edge applications. Aside from caching user data, common usages are filtering or aggregation of data generated from devices in level 7. 

Finally, \textbf{level 7} consists of \textit{actuators, sensors and things}, these devices are the most limited in their capacity, and varied availability. \textit{Things} act both as data producers and consumers towards edge-enabled applications. They enable limited forms of computation in the form of aggregation and filtering.

\subsection{Discussion}

Intuitively, the lower the level the harder it is to employ devices towards applications. Devices in levels 6 and 7 are especially restricted due to having lower availability and computational power, however, these can still be used in specific scenarios (e.g. an application with very low computational overhead but real-time latency requirements). 

Devices in levels 0-5 are potential candidates towards building the resource monitoring system we intend to create. Given the low availability of devices in higher levels, they are not very suitable towards overlay that supports the resource monitoring system, as they could not contribute much towards the system and would incur churn. This can be circumvented by employing devices in  other levels as gateways for mobile devices and \textit{things}.
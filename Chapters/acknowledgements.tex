%!TEX root = ../template.tex
%%%%%%%%%%%%%%%%%%%%%%%%%%%%%%%%%%%%%%%%%%%%%%%%%%%%%%%%%%%%%%%%%%%%
%% acknowledgements.tex
%% NOVA thesis document file
%%
%% Text with acknowledgements
%%%%%%%%%%%%%%%%%%%%%%%%%%%%%%%%%%%%%%%%%%%%%%%%%%%%%%%%%%%%%%%%%%%%

\typeout{NT FILE acknowledgements.tex}

\begin{ntacknowledgements}

% Acknowledgments are personal text and should be a free expression of the author.

The development of the work presented in this document would not have been possible without the help of some people and institutions that deserve the dearest acknowledgements.

First, I would like to thank the mentorship provided by my advisor, João Leitão, that, through example, inspired me to become a harder working person, and that through his insight and patience guided me to create this work.

Second, I would like to thank Bruno Anjos for the contributions provided to the developed benchmarking application, as well as for providing great company and coming into the lab with a smile throughout a rather unusual year due to COVID. Secondly, I would like to extend my thanks to Pedro Akos, for being a great colleague that was always ready to step in and help, providing great insights into my work as well as providing tools that helped benchmark the developed work.

Then, I would like to thank the Department of Informatics of the NOVA University of Lisbon and the NOVA LINCS research centre for providing a framework with mentorship and tools that greatly helped develop an ever-increasing interest in the area.

I also would like to extend a personal thank you to my partner, Marta Carlos, for making this thesis possible through the always enthusiastic support provided, given with a bright smile, even during the grimmest days of this, rougher than usual, year.

Then, I would like to thank my friends and family for always supporting me and encouraging me to follow what makes me happy.

Finally, the work presented in this thesis was partially supported by FC\&T through NOVA LINCS (grant UIDB/04516/2020) and NG-STORAGE (PTDC/CCIINF/32038/2017). Experiments presented in this paper were carried out using the Grid’5000 testbed, supported by a scientific interest group hosted by Inria and including CNRS, RENATER and several Universities as well as other organizations (see https://www.grid5000.fr).

% However, without any intention of conditioning the form or content of this text, I would like to add that it usually starts with academic thanks (instructors, etc.); then institutional thanks (Research Center, Department, Faculty, University, FCT / MEC scholarships, etc.) and, finally, the personal ones (friends, family, etc.).

% But I insist that there are no fixed rules for this text, and it must, above all, express what the author feels.

\end{ntacknowledgements}
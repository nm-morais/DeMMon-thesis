%!TEX root = ../template.tex
%%%%%%%%%%%%%%%%%%%%%%%%%%%%%%%%%%%%%%%%%%%%%%%%%%%%%%%%%%%%%%%%%%%%
%% abstrac-en.tex
%% NOVA thesis document file
%%
%% Abstract in English([^%]*)
%%%%%%%%%%%%%%%%%%%%%%%%%%%%%%%%%%%%%%%%%%%%%%%%%%%%%%%%%%%%%%%%%%%%

\typeout{NT FILE abstrac-en.tex}

The centralized model proposed by the Cloud computing paradigm mismatches the decentralized nature of mobile and IoT applications, given the fact that most of data production and consumption is performed by devices outside of the data center, as the numbers of these devices grows, so do the infrastructure costs and additional delays for the responses obtained by the end users. The aforementioned limitations have led us into a post-cloud era where a new computing paradigm arose: Edge Computing. Edge Computing takes into account the broad spectrum of devices residing outside of the data center as potential executors for computations. However, as these devices tend to have heterogenous capacity and computational power, there is the need for them to effectively share resources and coordinate to accomplish tasks otherwise impossible for a single device. 
The study of the state-of-the-art has revealed that existing resource monitoring and management solutions are commonly tailored for homogenous devices deployed on a single stable environment (characteristics fulfilled by the cloud environments) and often require manual configuration, which makes them unsuited for dynamic edge environments. In this work, we address these limitations by presenting a novel Decentralized Management and Monitoring (``DeMMon'') system targeted for edge settings. DeMMon aims to federate multiple devices with varied capacity and to continuously collect and aggregate information regarding not only the devices' operation, but also the execution of deployed applicational components in a decentralized and on-demand manner. This framework aims to allow a new category of decentralized resource deployment systems that perform, through partial knowledge of the system and custom-tailored resource managemnt protocols, resource management decisions that allow edge-enabled applications, decomposed in components, to adapt to runtime environmental changes by either offloading tasks, replicating or migrating the aforementioned components, among other adaptations.

Our solution was tested in its information dissemination and aggregation capabilities across a set of realistic emulated scenarios of up to 750 nodes and with variable amounts of node failures. Results show our system's validity and show that it can outperform state-of-the-art baselines across certain scenarios. 


% Palavras-chave do resumo em Inglês
\begin{keywords}
  Edge Computing, Resource Management, Resource Monitoring, Resource Location, Topology Management
\end{keywords} 

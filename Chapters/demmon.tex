%!TEX root = ../template.tex
%%%%%%%%%%%%%%%%%%%%%%%%%%%%%%%%%%%%%%%%%%%%%%%%%%%%%%%%%%%%%%%%%%%%
%% chapter2.tex
%% NOVA thesis document file
%%
%% Chapter with the template manual
%%%%%%%%%%%%%%%%%%%%%%%%%%%%%%%%%%%%%%%%%%%%%%%%%%%%%%%%%%%%%%%%%%%%

\typeout{NT FILE demmon.tex}

\chapter{DeMMON}
\label{cha:demmon} 

DeMMon (Decentralized Management and Monitoring framework) is a monitoring framework that aims to tackle the needs of decentralized resource management tools. These tools, as previously mentioned, must perform resource management decisions, such as load balancing or QOS optimizations, supported by partial and localized knowledge of the system. It is the goal of this framework, through the on-demand decentralized collection, aggregation, and storage of metrics in the form of time-series, to provide this knowledge base. We now detail what we believe to be the most common requirements of such tools:

\begin{enumerate} \label{enum:demmon}

    \item \textbf{Locality, by interacting with a partial set of nodes from the system}, optimized according to a certain proximity heuristic. This set is crucial such that a certain node has others to interact with to perform the aforementioned localized resource management decisions. In our framework, we chose latency as the heuristic for the proximity heuristic. The reasons for this choice were that not only does it does not rely on external tools, such as traceroute or a reverse IP-to-geolocation service, nor does it require pre-configuration of geolocation, making it possible for all nodes' configurations to be similar (thus making the deployment of large quantities of nodes easier). \label{enum:demmon_1}
    
     \item \textbf{Storage and querying of metric values}. As it is impossible to know ahead of time what type of information resource management systems and the functions to aggregate that information would otherwise require, we also believe that it is a requirement to \textbf{be as flexible as possible regarding metrics types and aggregation functions}. Furthermore, by allowing resource management systems to create custom-tailored metric formats tailored for their own needs, we believe it may even promote higher efficiency, as this feature may prevent inefficient workarounds from metric type restrictions. \label{enum:demmon_4}
    
    \item \label{enum:demmon_2} Ensure there are ways to \textbf{obtain the globally aggregate value of a metric distributed across one or more nodes in the system}, for example, the total number of nodes, service replicas, among others, without having to rely on a central component. This feature is important for resource management tools to, for example, maintain a (configurable) ratio of service replicas to nodes: by simultaneously collecting both the number of nodes in the system and the number of replicas, nodes can perform local decisions such as creating or decommissioning replicas, whenever the desired ratio of reaches a certain bound. Or alternatively, for example, for periodically collecting the number of nodes in the system to act as a configuration parameter for other systems. 
    
    \item Have a way to \textbf{obtain the aggregate value from a set of ``nearby'' nodes}. This feature is useful for decentralized resource management systems as it allows them to perform actions in a decentralized manner: by collecting the metrics relative to the usage of nearby nodes, each node may decide (e.g to improve a service's latency through proximity, to or reduce the load on a saturated service) to replicate or migrate service, motivated by this partial aggregate value. \label{enum:demmon_3}
    
    \item \textbf{Have a way to collect non-aggregated metric values from a set of ``nearby'' nodes}. Similar to item \ref{enum:demmon_3}, resource management frameworks may need to collect non-aggregated values to perform actions. In a service deployment context, it may want to collect the geographical positions of some nodes and deploy service replicas nearer to the current service clients' location. \label{enum:demmon_7}
    
    \item Provide ways to efficiently \textbf{propagate information} across nodes in the system. This is useful for resource management systems, as it prevents the overhead of establishing information propagation at the resource management layer. \label{enum:demmon_5}
    
    \item Ensure ways to \textbf{receive notifications based on issued alerts} that trigger whenever a supplied condition is met. This prevents clients of this system from resorting to periodically requesting/consulting information and performing the verifications themselves, saving unnecessary computation. By setting these alarms, resource management tools can, in turn, trigger resource management actions, for example, set an alarm that triggers if the mean of the CPU usage over the last N seconds reaches a certain threshold. When this alarm triggers, perform load-balancing or service migrations to spread the CPU load throughout nearby nodes. Furthermore, it is important to note that it is possible to create alerts on aggregated metric values. \label{enum:demmon_6}
    
\end{enumerate}

Having enumerated what we believe to be the requirements of such tools, we now provide a brief overview of the devised framework, which aims to fulfill these requirements. 

\section{Overview}
\label{sec:framework_overview}

The devised framework (illustrated in Figure \ref{fig:demmon-overview}) is coalesced by four main modules: the overlay network, the aggregation protocol, the API, and the monitoring module. In the following paragraphs, we describe each module's role within the framework and how they contribute to fulfilling the above-mentioned requirements.

\begin{figure}[htbp]
    \centering
    \includegraphics[width=\textwidth]{Chapters/Figures/DeMMon-overview.pdf}
    \caption{An overview of the architecture of DeMMon}
    \label{fig:demmon-overview}
\end{figure}
    
First, the \textbf{API} exposes the functionality of the framework, its main objectives are to (1) allow resource management solutions to collect metrics about nodes (or services they host) in the system; (2) allow those metrics to be queried through the use of a query language; (3) allow registering alarms which trigger based on conditions which evaluate the collected information. It is important to notice that the API is not the component tasked with gathering the information to perform these tasks. Instead, it exposes the results and mediates the interactions between the clients and the remaining modules.

Second, the \textbf{monitoring module} is tasked with storing metrics, resolving queries regarding stored metrics, removing expired metrics, periodically evaluating registered alarms, and triggering callbacks which the API then propagates to the client. This module satisfies points \ref{enum:demmon_4} and \ref{enum:demmon_6} of the aforementioned requirements.

The \textbf{overlay network} is responsible for building a latency-aware multi-tree-shaped network. Nodes in this network use latency, node capacity, and a set of logical rules to change their location either from one tree to another or within their tree until they have an optimized set of nodes (according to latency). The connections resulting from the operation of this protocol are the basis for the aggregation protocol. In addition, this module also offers limited horizon flood techniques, exposed through the API, fulfilling the points \ref{enum:demmon_1} and \ref{enum:demmon_5} of the requirements presented previously.

Finally, the \textbf{aggregation protocol} is a component that performs on-demand metric collection based on issued commands from the API. This component takes advantage of the overlay networks' established connections and hierarchical structure to perform efficient distributed aggregations. It allows three types of decentralized aggregation: (1) \textit{tree aggregation}, which consists of collecting metrics and merging them using the overlay protocols' trees, collecting a globally aggregated value in the tree roots (or a partial view of the system for nodes that are not the root of the overlay); (2) \textit{global aggregation}, where nodes also use their tree connections to efficiently collect a globally aggregated value (independently of being the root of the tree); and (3) \textit{neighbourhood aggregation}, where nodes collect values (non aggregated) of nearby nodes in term of hop proximity. These three mechanisms satisfy points \ref{enum:demmon_2}, \ref{enum:demmon_3} and \ref{enum:demmon_7} of the aforementioned requirements. 

In the following sections, we will begin by providing an brief explanation of the design and implementation of \textbf{GO-Babel} (section \ref{sec:GO-Babel}), a framework to build distributed systems' protocols (inspired in Babel~\cite{babel}), that we ported to Golang to ease the development of the overlay network. Following, we will provide a detailed explanation of the second contribution, composed by four modules. For each, we cover its' design and implementation, starting by the \textbf{overlay network} (section \ref{sec:overlay_network}), followed by \textbf{aggregation protocol} (section \ref{sec:mon_protocol}), and lastly, the \textbf{monitoring module} (section \ref{sec:mon_module}) and \textbf{API} (section \ref{sec:api}). 

\section{GO-Babel} 
\label{sec:GO-Babel}
%!TEX root = ../template.tex
%%%%%%%%%%%%%%%%%%%%%%%%%%%%%%%%%%%%%%%%%%%%%%%%%%%%%%%%%%%%%%%%%%%%
%% chapter2.tex
%% NOVA thesis document file
%%
%% Chapter with the template manual
%%%%%%%%%%%%%%%%%%%%%%%%%%%%%%%%%%%%%%%%%%%%%%%%%%%%%%%%%%%%%%%%%%%%

\typeout{NT FILE GO-Babel.tex}

\chapter{GO-Babel}
\label{cha:GO-Babel}

The first contribution of this masters dissertation is an event-based framework called GO-Babel. This framework is a port in Golang \todo{citation} of Babel \todo{citation} with a few additions focused on fault detection and latency probing. Babel \todo{citation} itself is based on Yggdrasil \todo{citation}, which in turn is inspired on \todo{cite and discover paper of original event-based framework}.

The decision to build this framework arose from the need to use Babel for building the distributed protocols and the decision to use Golang during this dissertation (due to its primitives for building concurrent systems). Given that there was no implementation of Babel in Golang, and the current Babel implementation lacked needed features such as a fault detector and a latency measurement tool, we implemented a new version in Golang with these additions.

\section{Overview}

In summary, this framework has the following main objectives:

\begin{enumerate}

    \item Abstract the networking layer, providing \textbf{channels}, which are essentially an abstraction over TCP connections, providing callbacks whenever outbound or inbound connections are established or terminated and whenever messages or sent or received from the respective operating system buffers.

    \item Execute protocols in a single-threaded environment and provide abstractions for timers, request-reply patterns, notifications, and ease channel management.

    \item Provide a layer of abstraction over node latency probing and fault detection.

\end{enumerate}

\begin{figure}[htbp]
    \centering
    \includegraphics[width=\textwidth]{Chapters/Figures/Go-Babel-Overview.pdf}
    \caption{An overview of the architecture of GO-Babel}
    \label{fig:go-babel-overview}
\end{figure}

In the figure \ref{fig:go-babel-overview} we may observe a high-level overview of the architecture of this framework, composed of five main components which communicate via callbacks. We now summarize each components' roles in the framework:

\begin{enumerate}

    \item Babel is the component tasked with initializing the protocols and all the other components according to issued configurations. It also acts as a mediator between the protocols and the remaining components.

    \item The stream manager is responsible for handling incoming and outgoing connections, connecting to new peers, and sending messages. Whenever the state of any connection changes, the stream manager delivers events to protocols with the connection status (e.g. if the connection established, connection failure, message sent/received, connection terminated, among others). It also provides operations for sending messages in temporary connections (either using TCP or UDP).

    \item The timer queue allows the creation and cancellation of timers and manages the lifecycle of timers issued by the protocols, delivering events to protocols whenever timers reach their expiry time. The timer queue also allows creating periodic timers, which trigger at the set periodicity until cancelled.

    \item The notification hub is responsible for handling notifications and notification subscriptions, propagating issued notifications to registered subscribers (protocols).

    \item The node watcher allows for protocols to measure node latency and detect failures via a PHI-Accrual fault detector.

\end{enumerate}

As previously mentioned, the  Node Watcher is the only new addition to the framework, and consequently, it is the component explained in further detail. The remaining components of this framework were implemented similarly to Babel \todo{insert citation} and can found in \todo{cite}.

\section{Node Watcher} \label{sec:GO-Babel}

The node watcher is a component that, if registered, will listen for probes in a custom port (specified in the configurations) and send a reply with a copy of the contents back to the original senders. These probes are sent (usually) via UDP and carry a timestamp used by the original sender to calculate the round-trip time to the target node.

The motivation to build this component was a lack of tools to measure latency in the original design of Babel. If, for example, a protocol were to measure the latency to a node without an active connection, it would need to establish a new TCP connection and use it to send the probes. In this case, both the fault detector and latency detector logic are in the protocol, which is sub-optimal since the same logic would have to be replicated by any protocol that wishes to optimize its active connections using latency as a heuristic. Alternatively, if a protocol measures latencies in a separate module asynchronously (making the code reusable), this would break the single-threaded nature of the execution of protocols in Babel, and protocols would have to deal with race conditions of altering the state concurrently. Due to this, we believe that encapsulating this logic in an optional component and expose it in a Babel-compatible interface is the preferred option, which was the one used.

The main interface for the Node watcher is composed of two functions, ``watch'' and ``unwatch''. When a node is ``watched'', the node watcher starts sending probes to the target node according to the issued configuration settings and instantiates a PHI-accrual fault detector \todo{insert citation} together with a rolling-average latency calculator for that node. When the node receives replies with copies of sent probes, it updates the corresponding rolling average calculator and fault detector. Conversely, when a node is ``unwatched'', the node watcher stops issuing the probes and deletes the fault detector and latency calculator.


When a protocol issues a command to watch a node, if the ``watched'' node fails to reply within a time frame, the Node Watcher falls back to TCP. This fallback aims to overcome cases where the watched node may be dropping UDP packets due to a constraint in its infrastructure. If the watched node also does not accept the TCP connection, the node watcher sends a notification to the issuing protocol.


In order to prevent protocols from having to set timers to check the nodes' latency calculator or fault detector, the node watcher allows the possibility of registering ``observer'' functions (or conditions), which return a boolean value based on the current node information. The node watcher then executes these functions periodically, and if one returns true, a notification gets sent to the issuing protocol. In order to prevent protocols from getting overloaded with notifications when a condition returns ``true'', these may configure a grace period, which the node watcher will wait for until re-evaluating the condition.

\section{Conclusion}

We believe Go-Babel is a valuable contribution as it eases the implementation of self-improving protocols which employ latency as an optimization heuristic. In addition, it provides a secondary fault detector which may be employed together with the TCP connections. Lastly, as the implementation is in Golang \todo{cite}, it allows easier integration with a range of packages already implemented in the language. \todo{Sinto que esta secçao nao devia existir?}


\section{Overlay network} 
\label{sec:overlay_network}

\input{Chapters/membership/pseudocode/jlt-pseudocode.tex}

In this section, we discuss the design of the overlay network, which aims to build and maintain a latency and capacity-aware tree-shaped network (capacity represents one, or a combination of, values that denote the node's computing and networking power). We begin by providing the considered system model, then follow with an overview of the mechanisms responsible for building and maintaining the tree. Lastly, we conclude the chapter with a summary and discussion of the protocol.

\subsection{System Model}

The assumed system model is assumed to be a distributed scenario composed of nodes connected to the internet set-up such that they can send and receive messages via the internet (with an external IP or port-forwarding). We also assume that nodes are spread throughout a large area and have varied capacity values.

Regarding the fault model, we assume that all but a small portion of nodes (also known as the landmarks, which in our model represent DCs) can fail, and when other nodes fail, they do so in a crash-fault manner, stopping all emissions and receptions of messages. We assume landmarks have additional fault tolerance given their privileged infrastructure, and additionally, we assume other such as replication \cite{} mechanisms could be employed to ensure that faulty landmarks get replaced in case of failure. 
  
Finally, all nodes must run the same software stack with similar configuration settings and landmark values, installed a priori.

\subsection{Overview}

As previously mentioned, the main objective of the created protocol is to establish a latency and capacity-aware multi-tree-shaped network, rooted on the previously mentioned landmarks. Our motivations for choosing the tree structure for the network are the following: (1) to map the cloud-edge environment, by rooting the trees on nodes running DCs in the cloud, and creating a hierarchical structure for other, less powerful, nodes to be coordinated from the roots \todo{isto e esticar?} (2) to be able to map the heterogeneity of each device in the environment: by biasing the placement of nodes in the tree such that nodes with higher capacity are placed higher in the tree, and nodes with lower capacity are biased towards lower levels of the tree, nodes are used more or less according to their capacity values; (3) the tree structure can be easily employed to perform efficient aggregations, by propagating and merging values recursively from the lower to the higher levels of the tree, which is the basis for the aggregation protocol presented in \todo{add ref}; and finally, (4) by leveraging on the tree structure, nodes can propagate information efficiently, given that, in a network composed of N nodes, broadcasts require only N-1 message transmissions to reach all nodes in the network. 

The tree structure the protocol aims to establish and maintain can be observed in figure \todo{criar imagem para ilustrar estrutura resultante}, which, as previously referenced, is composed of multiple trees, and these are connected through their respective landmarks. The nodes connected to the landmarks, (denoted their \textbf{children}), may or may not form themselves be the parent of their own children. Intuitively, the \textbf{grandparent} of a certain node is their parents' parent, and the descendants of a certain node are constituted by all its children, and childrens' children, recursively. All nodes who share the same parent (\textbf{siblings}) are connected among themselves, forming a \textbf{group}, whose size is biased (but not guaranteed) to be within two configurable upper and lower bounds. Therefore, all nodes have active connections to their parent, children and siblings, this group of nodes denotes a node's \textbf{active view}. Nodes also have have knowledge of other nodes in the network, acquired via periodic semi-random periodic walks, which we will describe ahead. 

The devised algorithm is composed by three main mechanisms: (1) the \textbf{join} mechanism, which aims to establish the initial tree structures, (2) the \textbf{active view maintenance}, responsible for bounding the number of connections for each node, and optimizing the connections of each node, (3)  and finally \textbf{passive view maintenance}, responsible for collecting information about peers which are not in the active view, which are used for both fault tolerance and connection optimizations.

\subsubsection{Join mechanism}

The Join mechanism is the mechanism responsible for choosing the initial parent connection, which performs a greedy depth-first search to find a suitable low latency node in the network with more than zero children. This mechanism is the first to be executed by all nodes in the system, with the pseudocode presented in algorithm \ref{alg:memb:join}. 

% JOIN -----

\begin{algorithm}{}
\caption{Join Protocol} \label{alg:memb:join}
    % \setstretch{0.85}
\begin{algorithmic}[1]
    \asdtypes
        \State Node : <lat, parentIP, nrChildren, replied, IP, ID, coords, version, children<IP,  nrChildren\>\>
    \asdend
    \asdstate \label{alg:memb:join:state}
        \State contactedNodes \Comment{collection of all successfully contacted nodes}
        \State nodesToContact \Comment{nodes being contacted}
        \State landmarks \Comment{landmark nodes}
        \State joinTimeouts \Comment{collection of contacted nodes -> timerIDs}
        \State bestPeerLastLevel : Node \Comment{the best peer contacted so far in the join process}
        \State joinReqTimeoutTid \Comment{ timerID for join messages}
        \State self : Node \Comment{ myself}
    \asdend

\asdupon[Init(landmarks : IP[], selfIP, isLandmark)] \label{alg:memb:join:init}
    \State landmarks \asdassign landmarks 
    \State joinTimeouts, prevBestP \asdassign \{\}, nil
    \IfThenElse{isLandmark}
    {addLandmarkUntilSuccess(landmarks) \label{alg:memb:join:add_land}} 
    {contactNodes(landmarks) \label{alg:memb:join:contact_landm}} 
\asdend


\asdupon[receive(Join<>,sender)] \label{alg:memb:join:recv_join}
    \State sendMessageSideChannel(JoinReply<self.parent, self.node, self.children>, sender) 
\asdend
    
\asdupon[receive JoinReply(<parentIP, node, children>, sender) \&\& measuredLatency(lat)]  \label{alg:memb:join:recv_join_reply}
        \If{\asdin{node.IP}{nodesToContact}} 
            \If{\asdin{parentIP}{Landmarks}}
                \State self.coordinates[getIdx(landmarks, sender)] = lat
            \EndIf
            \State nodesToContact[node.IP].lat \asdassign lat
            \State nodesToContact[node.IP].children \asdassign children
            \State nodesToContact[node.IP].parent \asdassign parentIP
            \State nodesToContact[node.IP].replied \asdassign true
            \State cancelTimer(joinTimeouts[sender])
            \State delete(joinTimeouts, sender)
        \Else
            \State nodesToContact.delete(node)
        \EndIf
\asdend

\asdupon[(forall n $\in$ nodesToContact -> n.replied)] \label{alg:memb:join:cond_go}
    \State contactedNodes.appendAll(nodesToContact)
    \For{node in sortedByLatency(nodesToContact)}
        \If{(\asdnotin{node.IP}{landmarks}) \&\& node.nrChildren == 0} \label{alg:memb:join:verif_children}
            \State continue \Comment{check if node has enough children}
        \EndIf
        \If{prevBestP != nil \&\& (prevBestP.lat $\le$ node.lat || prevBestP.nrChildren < config.minGroupSize)} \label{alg:memb:join:verif_vs_prev}
            \State joinAsChild(prevBestP)
        \Else
            \State prevBestP \asdassign node \label{alg:memb:join:advance}
            \State toContact \asdassign [\asdin{c}{prevBestP.children} -> c.nrChildren > 0]
            \State contactNodes([c.IP for c in toContact])
        \EndIf
        \State return
    \EndFor
    \IfThenElse{prevBestP != nil} 
    {joinAsChild(prevBestP)}  
    {abortJoinAndRetryLater()} \label{alg:memb:join:join_base_case}
    \State return
\asdend

\asdupon[JoinTimeoutTimer(node) || NodeMeasuringFailed(node)] \label{alg:memb:join:exclusions}
    \IfThenElse{(L in Landmarks)}{abortJoinAndRetryLater()}{delete(nodesToContact[L])} 
\asdend

\asdupon[JoinRequestTimer(p : Node)]
    \If {sender == prevBestP}
        \If{p.parentIP != nil}
            \State prevBestP \asdassign contactedNodes[p.parentIP]
            \State joinAsChild(prevBestP)
        \Else
            \State abortJoinAndRetryLater()
        \EndIf
    \EndIf
\asdend

\asdupon[receive(JoinRequest<>, sender)]
    \State childID \asdassign addChildren(sender) \Comment{new chilren is established, and an ID is generated for it}
    \State sendMessageSideChannel(JoinRequestReply<childID, self>, p.IP)
\asdend
    
\asdupon[receive(JoinRequestReply<myID, parent>, sender)]
    \If {sender == prevBestP} 
        \State parent \asdassign sender \Comment{Adds Parent is established, join complete}
        \State cancelTimer(joinReqTimeoutTid)
        \State self.ID \asdassign parent.ID + "/" + myID \Comment{Later used in shuffle mechanism}
    \EndIf
\asdend

\asdprocedure[joinAsChild(p : Node)]
    \State joinReqTimeoutTid \asdassign setupTimer(JoinRequestTimer<p>, config.JoinTimeout)
    \State sendMessageSideChannel(JoinRequest<>, p.IP)
\asdend

\asdprocedure[contactNodes(ips : IP[])]
    \State nodesToContact \asdassign \{\}
    \State toContact \asdassign [Node<0,nil,0,false,lIP,false,[]> for ip in ips]
    \For{n in toContact}
        \State nodesToContact[n] \asdassign n
        \State MeasureNode(n) 
        \State sendMessageSideChannel(JoinMessage<>, n)
        \State joinTimeouts[n] \asdassign \asdassign setupTimer(JoinTimeoutTimer(n), config.JoinTimeout)
    \EndFor
\asdend

\end{algorithmic}
\end{algorithm}


Its first step (line \ref{alg:memb:join:state}) is to initialize the state of the joining node, composed by: (1) a map of type Node containing all successfully contacted nodes so far the join process, (2) a collection of type Node and a set of timer ids for each contacted node, (4) the best node contacted so far in the join process, (5) a timer id for contacting the chosen node in the join process, and finally (5) a variable of type Node denoting the peer itself. The type ``Node'' is a collection of attributes regarding a node, composed of latency measured, its current parent, number of children, whether the node replied to the message, its IP, and an array of its childrens' IP and children number.

Then, each node joins the system, the procedures taken to join the tree differ consonant the node is a landmark or not. Given that landmarks are the roots of the trees, they have no parent in the resulting overlay, and consequently, in the join algorithm, these nodes attempt to repeatedly establish a connection with other landmarks by sending a special message. Landmarks that receive this message will send a reply and establish a connection back (line \ref{alg:memb:join:add_land}), a joining landmark node only stops sending messages to other landmarks when the respective reply is received.

Nodes that are not landmarks begin the process of choosing their initial parent, initiated by sending a JOIN message via a temporary TCP channel, measuring the latency, and issuing ``joinTimers'' for all tree roots (line \ref{alg:memb:join:contact_landm}), then the node awaits the responses from the contacted nodes, during this process, the joining node listens for any ``joinTimers'' which have triggered, or until any of the node measurements has been unsuccessful (meaning contacted nodes have exceeded their reply timeout), if this happens, in the case of the contacted node being a landmark, the joining node aborts the join process and waits a configurable amount of time until attempting to re-join the overlay again. If the timed-out node is not a landmark, then that node is excluded from the remaining of the join process, and the join process is resumed as normal (line \ref{alg:memb:join:exclusions}).

When a node receives a JOIN message, it sends a JOINREPLY message back to the original sender containing: its parent, itself, and its children (line \ref{alg:memb:join:recv_join}). When the joining node receives the joinReply, it checks to see if it is not from a timed out node, or if the node's parent is not the same anymore, if any of these conditions are observed, then the reply is discarded. 

Then, whenever the joining node has either: received the JOINREPLY messages from all the contacted nodes, and stored the information (line \ref{alg:memb:join:recv_join_reply}), or they have been timed-out via the ``joinTimers'', it evaluates all contacted nodes, attempting to find the contacted node with the lowest latency which is a suitable parent by performing the following verifications:

\begin{enumerate}
    \item Verify if the node already has any children or if the node is a landmark (and can become parent of the joining node) (line \ref{alg:memb:join:verif_children}).
    
    \item Verify if there was a node already contacted which was a suitable parent and had lower latency, if there was, the joining node sends a JOINREQUEST and sets up a ``JoinRequestTimer'' for that node, and stops the verification process. (line \ref{alg:memb:join:verif_vs_prev})

    \item Verify if the current node has both enough children, and has the lowest latency up to this point in the join process, then the joining node assigns it as its best node so far and starts a new recursive step by sending JOIN messages and measuring the children of that node which themselves have more than one children (line \ref{alg:memb:join:advance}). Note that if none the current nodes' children are suitable parents (i.e. have no children themselves), then the condition in line \ref{alg:memb:join:cond_go} is triggered and the joining node will request the current best node to be its parent.
\end{enumerate}

If none of the verified peers was suitable to start a new recursive step (line \ref{alg:memb:join:join_base_case}) (either had no children or all verified nodes had higher latency than a previously contacted node), then the node joining node sends a ``JoinRequest'' to that node and sets up a ``JoinRequestTimer'' for the best previously contacted node (any node which receives a ``JoinRequest'' message replies with a ``JoinRequestReply''). 

The join process is concluded with both the reception of a ``JoinRequestReply'' and the establishment of the connection between the sender and receiver of the message. If the ``JoinRequestTimer'' timer triggers while waiting for the response, the node will recursively fall back to the parent of the selected node or re-join the overlay later in case there is no parent available. 

\subsubsection{Active view maintenance}

The second mechanism of the devised membership algorithm, called active view maintenance, is the mechanism responsible for maintaining the size of the groups. It achieves this by choosing new parents to form new groups using latency and node capacity as heuristics for the choice. This mechanism is coordinated by each parent, and is only done when the group exceeds its size limit. The information necessary to feed this mechanism is transmitted periodically from each children to their parents. 

The pseudocode for this mechanism is presentend in algorithm \ref{alg:memb:active_view_maint}, and starts by defining the necessary state: the nodes' active view (parent, children, and siblings), and an auxiliary map of sets, which holds the latencies of each children to every other children. (lines \ref{alg:memb:active_view_maint:state_start}-\ref{alg:memb:active_view_maint:state_end}). 

The mechanism starts with the propagation of information from the parent to the children and vice-versa. As observable in lines \ref{alg:memb:active_view_maint:update}-\ref{alg:memb:active_view_maint:update_end}), each parent transmits to its children a list of its current siblings, and propagates to its parent the latency to each of its siblings. Then, when this information is received (lines \ref{alg:memb:active_view_maint:update_recv_par} and \ref{alg:memb:active_view_maint:update_recv_chi}), it is merged into their local states for later use.

The second part of this mechanism is also periodic and is responsible for maintaining the group sizes by creating new groups, or sending children to already created groups (line \ref{alg:memb:active_view_maint:update_eval}) when necessary, this mechanism is only executed if the number of children exceeds the configured maximum number of children per parent (in order to keep group sizes close to full). Each node starts by merging all of its received latency pairs into a single set, where the node with the highest capacity is the first node of each pair (lines \ref{alg:memb:active_view_maint:update_eval_merge_start}-\ref{alg:memb:active_view_maint:update_eval_merge_finish}). Then, it iterates the added edges set by ascending order of latency, making the following steps:

\begin{enumerate}
    \item If the number of current children minus the nodes already sent to a lower level is lower than the middle point between the maximum size of a group, then the node concludes the mechanism (line \ref{alg:memb:active_view_maint:check_done_1})
    
    \item if the latency of the node pair that is being observed is higher than the parents latency to it, meaning it raises the overall latency of the system, and the current node size is lower than the configured maximum, then the process is concluded. (line \ref{alg:memb:active_view_maint:check_done_2})
    
    \item If any of the nodes was already sent to lower levels of the tree, then the current edge is skipped (line \ref{alg:memb:active_view_maint:check_done_3})
    
    \item Then, if the node wigh higher capacity of the edge pair has no children yet, the lower capacity node is added to its ``possibleChildren'' set, when this set has the same size of the minimum configured group size (i.e. the hifgher capacity node has enough candidates to form a new group), then the 
\end{enumerate}


\begin{algorithm}
    \caption{Membership protocol (Active view Optimization)} \label{alg:memb:active_view_maint}
    % \setstretch{0.85}
    \begin{algorithmic}[1]
        \asdstate
            \State parent \Comment{defined in join} \label{alg:memb:active_view_maint:state_start}
            \State children \Comment{defined in join} 
            \State siblings  
            \State childrenLatencies : dict<string:dict<string:number>> \label{alg:memb:active_view_maint:state_end} \Comment{Holds the latencies of each children to every other children}
        \asdend

        \asdrepeateveryx{config.updatePeriodicity} \label{alg:memb:active_view_maint:update}
            \If{parent != nil}
                \State sLatencies \asdassign set()
                \For{sibling in siblings}
                    \State sLatencies.append(<sibling.IP,sibling.measuredLatency)
                \EndFor
                \State sendMessage(UpdateChildStatus<children, siblingLatencies>, parent)
            \EndIf
            \For{child in chidren}
                \State sendMessage(UpdateParentStatus<self, chidren \\ child>)
            \EndFor
        \asdend \label{alg:memb:active_view_maint:update_end}

        \asdupon[receive(UpdateParentStatus<parent, children>, sender)] 
        \label{alg:memb:active_view_maint:update_recv_par}
            \If{sender == parent.IP}
                \State parent \asdassign parent
                \State self.ID \asdassign parent.ID + myID
                \State grandParent \asdassign grandParent
                \State siblings \asdassign siblings
            \EndIf
        \asdend

        \asdupon[receive(UpdateChildStatus<child, childSiblingLatencies>, sender)]\label{alg:memb:active_view_maint:update_recv_chi}
            \If{children[sender] != nil}
                \State children[sender]\asdassign child
                \State childrenLatencies[sender] \asdassign childSiblingLatencies
            \EndIf
        \asdend

        \asdrepeateveryx{config.evalGroupSize} \label{alg:memb:active_view_maint:update_eval}
            \If{len(children) <= config.maxGroupSize}
                \State return
            \EndIf
            \State childrenLatValues \asdassign set()
            \For{c1 in children} \label{alg:memb:active_view_maint:update_eval_merge_start}
                \For{<c2, lat> in childrenLatencies[c]}
                    \If{lat - c1.measuredLatency > d.config.maxLatDowngrade}
                        \State continue
                    \EndIf
                    \State isDowngrade \asdassign lat > c1.measuredLatency
                    \IfThenElse{c1.cap > c2.cap}
                    {childrenLatValues.add(<c1,c2,lat,isDowngrade>)}
                    {childrenLatValues.add(<c2,c1,lat,isDowngrade>)}
                \EndFor
            \EndFor \label{alg:memb:active_view_maint:update_eval_merge_finish}
            \State kickedNodes, newParents \asdassign set(),set()
            \State pChildren \asdassign dict<string,set<Node{>}{>} \Comment{set of potential children for each children}
            \State sortByLatency(childrenLatValues)
            \State idealGroupSize \asdassign config.maxSize - config.MinGroupSize
            \For{<c1,c2,lat,isDowngrade> in childrenLatValues}
                \If{len(children) - len(kickedNodes) <= idealGroupSize} \label{alg:memb:active_view_maint:check_done_1}
                    \State break
                \EndIf
                \If{len(children) - len(kickedNodes) <= config.maxSize \&\& isDowngrade} \label{alg:memb:active_view_maint:check_done_2}
                    \State break
                \EndIf
                \If{\asdin{c1}{kickedNodes} || \asdin{c2}{kickedNodes} ||  \asdin{lowerCapC}{newParents}} \label{alg:memb:active_view_maint:check_done_3}
                    \State continue
                \EndIf
                \If{c1.nrChildren == 0 \&\& newParents[c1] == nil}
                    \State pChildren[c1].append(c2)
                    \If{len(pChildren) >= config.MinGroupSize}
                        \If{len(children) - len(kickedNodes) - len(pChildren) > idealGroupSize || len(pChildren) >= config.maxSize}
                            \For{potentialChild in pChildren[c1]}
                                \State newParents <- newParents + c1
                                \State send(OptimizationPropose<c1>, potentialChild)
                                \State c1.nrChildren++
                                \State kickedNodes <- kickedNodes + potentialChild
                            \EndFor
                            \For{<nIP,pontentialChildrenTmp> in pChildren}
                                \State pontentialChildrenTmp.deleteAll(pChildren[c1])
                            \EndFor
                            \State pChildren[c1] \asdassign set<Node>
                            \State continue
                        \EndIf
                    \EndIf
                \Else
                    \State kickedNodes <- kickedNodes + c2
                    \State send(OptimizationPropose<higherCapNode>, lowerCapNode)
                \EndIf    
            \EndFor
        \asdend

        \asdupon[receive(OptimizationPropose<newParent>, sender)]
            \If{sender == parent}
                \State send(OptimizationProposeRequest<sender>, newParent)
            \EndIf
        \asdend

        \asdupon[receive(OptimizationProposeRequest<p>, sender)]
            \If{ p == parent \&\& sender in siblings} \Comment{ parent issuing the message is the same parent that i have}
                \State send(OptimizationProposeRequestReply<true>, sender)
            \Else
                \State sendSideChannel(OptimizationProposeRequestReply<false>, sender)
            \EndIf
        \asdend

        \asdupon[receive(OptimizationProposeRequestReply<reply>, sender)]
            \If{reply}
                \State sendMessageAndDisconnectFrom(DisconnectMessage<>, parent)
                \State addParent(sender)
            \EndIf
        \asdend

    \end{algorithmic}
\end{algorithm}


\subsubsection{Passive view maintenance}
\begin{algorithm}
\label{alg:memb:passive_view_maint}
\caption{Membership protocol (Passive view maintenance)}
\begin{algorithmic}[1]
    
    \asdstate
        \State pView : set<Node> \label{alg:memb:passive_view_maint:state}
    \asdend

    \asdrepeateveryx{config.RandWalkPeriodicity} \label{alg:memb:passive_view_maint:walk_trig}
        \State sample \asdassign getRandSample([pView + allNeighs + children + parent + siblings], config.NrPeersToMergeRandWalk)
        \State target \asdassign getRand(excludeDescendatsOf(ascNeighs, self.ID))
        \State sendMessage(RandomWalk<sample + self, config.RandWalkTTL, self.ID, self.IP>, target)
    \asdend

    \asdupon[receive( RandomWalk<sample, ttl, nID, orig>, sender)] \label{alg:memb:passive_view_maint:walk_rec}
        \State nrNodesToRemove \asdassign config.NrPeersToMergeRandWalk
        \If{config.RandWalkTTL - ttl < config.NrStepsToIgnore}:
            \State nrNodesToRemove \asdassign 0
        \EndIf
        \State updateNodesToHigherVersion(sample, pView) \label{alg:memb:passive_view_maint:walk_rec_merge_start}
        \State ascNeighs \asdassign set(parent + siblings)
        \State allNeighs \asdassign set(allNeighs + ascNeighs + children)
        \State toAdd \asdassign getRandSample(excludeDescendantsOf(pView + allNeighs / sample,self.ID), config.NrPeersToMergeRandWalk)
        \State toRemoveFromSample \asdassign getRandSample(sample, nrNodesToRemove)
        \State sample \asdassign sample / toRemoveFromSample
        \State pView \asdassign excludeDescendantsOf(toRemoveFromSample + pView, self.ID)  
        \State pView \asdassign pView / allNeighs
        \State pView \asdassign pView[:config.MaxEViewSize]
        \State sample  \asdassign trimSetToSize(sample + toAdd + self, config.MaxRndWalkSampleSize) \label{alg:memb:passive_view_maint:walk_rec_merge_end}
        \State target \asdassign getRand(excludeDescendantsOf(allNeighs, nID)
        \If{target == nil || ttl == 0} \label{alg:memb:passive_view_maint:walk_rec_send}
            \State sendMessageSideChannel(RandomWalkReply<sample>, orig)
        \Else
            \State sendMessage(RandomWalk<sample, ttl-1, nID, orig>, getRandom(ascNeighs))
        \EndIf \label{alg:memb:passive_view_maint:walk_rec_send_end}
    \asdend

    \asdupon[receive( RandomWalkReply<sample>, sender)]: \label{alg:memb:passive_view_maint:walk_reply_recv_start}
        \State sample \asdassign excludeDescendantsOf(sample, self.ID)
        \State updateNodesToHigherVersion(sample, pView)
        \State sample \asdassign excludeNodesInActiveView(sample)
        \State pView \asdassign trimSetToSize(pView + sample, config.MaxEViewSize) \label{alg:memb:passive_view_maint:walk_reply_recv_end}
    \asdend


    \asdrepeateveryx{config.OportunisticOptimizationTimeout} \label{alg:memb:passive_view_maint:eval_nodes}
        \State toMeasureRand \asdassign getRandSample(pView, len(pView)) // shuffle sample
        \State toMeasureBiased \asdassign sortByEuclideanDist(pView / toMeasureRand)

        \State measuredNr \asdassign 0
        \For{i=0; i < len(toMeasureRand) \&\& measuredNr < config.ToMeasureRand ; i++}
            \If{canBecomeChildrenOf(p)} \label{alg:memb:passive_view_maint:opt_verification_1}
                \State measuredNr++
                \State measurePeer(p)
            \EndIf
        \EndFor
        \State measuredNr \asdassign 0
        \For{i=0; i < len(toMeasureRand) \&\& measuredNr < config.toMeasureBiased ; i++}
            \If{canBecomeChildrenOf(p)} \label{alg:memb:passive_view_maint:opt_verification_2}
                \State measuredNr++
                \State measurePeer(p)
            \EndIf
        \EndFor
    \asdend

    \asdupon[peerMeasured(p, latency)] \label{alg:memb:passive_view_maint:peer_measured}
        \State latencyImprovement := parent.measuredLatency - Latency
        \If{latencyImprovement >= config.MinLatencyForImprovement}
            \State sendMessageSideChannel(OportunisticImprovementReq<self>,p)
        \EndIf
    \asdend

    \asdupon[receive(OportunisticImprovementReq<p>,sender)] \label{alg:memb:passive_view_maint:oport_msg_recv}
        \If{isDescendent(p.ID,self)}
            \State sendMessageSideChannel(OportunisticImprovementReqReply<false>,sender)
        \Else
            \State addChildren(sender)
            \State sendMessageSideChannel(OportunisticImprovementReqReply<true>,sender)
        \EndIf
    \asdend

    \asdupon[receive(OportunisticImprovementReqReply<answer>,sender)] \label{alg:memb:passive_view_maint:op_msg_reply_recv}
        \If {answer} 
            \State disconnectFromCurrentParent(parent)
            \State addParent(sender)
        \EndIf
    \asdend

    \asdprocedure[canBecomeChildrenOf(c, parent)]
        \If{(c.nrChildren > 0 \&\& parent.ID.level() >= c.ID.level())}
            \State return false
        \EndIf
        \State return parent.nrChildren > 0 \&\& !isDescendentOf(parent.ID, c) \&\& !isDescendentOf(c, parent.ID)
    \asdend

    \asdprocedure[isDescendentOf(nodeID, PotentialDescID)]
        \State return PotentialDescID.Contains(nodeID)
    \asdend

\end{algorithmic}
\end{algorithm}

\subsection{Summary}

\section{Monitoring protocol}

\subsection{Overview}

\subsection{Aggregation mechanisms}

\subsubsection{Single root aggregation}

\subsubsection{Multi root aggregation}

\subsubsection{Neighborhood aggregation}

\subsection{Summary}

\section{API}

\subsection{System Model}

\subsection{Overview}

\subsection{Showcase}

\section{Aggregation protocol}
\label{sec:mon_protocol}

With a membership protocol capable of coordinating nodes into building a tree overlay network, a new range of options open for both the dissemination and aggregation of information. In this section, we discuss the devised decentralized aggregation protocol, which leverages on the tree structure to enable the on-demand collection of information (or metrics) in a decentralized manner. While in this work we present the protocol leveraging the overlay protocol defined in \ref{sec:overlay_network}, it is important to notice that the protocol is agnostic to which overlay protocol is executing, as long as it has the following characteristics: (1) it forms one or more network-shaped trees whose roots are interconnected, nodes in this tree must be connected to their parents, children and siblings (active view) in a bidirectional manner, and must provide callbacks for each node that is added or removed from the nodes's active view.

This protocol provides three different types of decentralized aggregation techniques, inspired from the study of the state of the art: (1) tree aggregation, (2) neighbourhood aggregation, and finally, (3) global aggregation, which we now clarify in further detail, starting by tree aggregation.

\subsection{Tree aggregation} \label{sec:mon_protocol:tree_agg}

Tree aggregation is the mechanism responsible for collecting metrics and merging them using the tree, collecting an aggregated value for all nodes which are descendants of the node performing this mechanism (also denoted the \textbf{root of the aggregation tree}). It is important to notice that this mechanism is executable by all nodes in the mechanism, and if two different nodes are aggregating the same values with the same parameters, and a node is descendent of the other, the descendant node, will, when possible, embed its tree into the ascendants' (thus not performing the mechanism individually for nodes with overlapping trees, and reusing the values of the already existing tree).

The pseudocode for this mechanism (as defined in \ref{alg:mon:tree_agg}) begins by defining the necessary state (line \ref{alg:mon:tree_agg:state}) for its execution, starting by the active view, composed of the parent, children, and siblings of the node, this state is maintained by the overlay protocol and changes to it are propagated through notifications (this process is omitted from the pseudocode). Then, it declares three maps, the first contains the necessary metadata for each aggregation tree, values of this map contain: (1) the height of the tree, (2) the merge function, (3) the query to generate local values, (4) the periodicity to export values (5) a boolean value representing if the value should be exported locally, (6) a boolean representing if the parent is also in the tree (and the node must propagate values to it or not), and finally (7) the ID of the tree from the parent's perspective (or nil, if the parent is not in the tree).

\begin{algorithm}
\caption{Tree aggregation} \label{alg:mon:tree_agg}
\begin{algorithmic}[1]

    \asdstate \label{alg:mon:tree_agg:state}
        \State parent,children,siblings \Comment{Defined by the overlay protocol}
        \State tIds \asdassign map()
        \State lastSeen \asdassign map()
        \State childValues \asdassign map()
    \asdend

    \asdupon[StartTreeAggregationRequest(tHeight, mergeF, query, periodicity ,outmName)] 
    \label{alg:mon:tree_agg:start_req}
        \State tId \asdassign hash(tHeight + mergeF + query + periodicity + outmName) \label{alg:mon:tree_agg:start_req_start}
        \If{tId in tIds} 
            \State <tHeight, mergeF, query, periodicity, outmName, timerId, isLocal, isParentSub, ptId> \asdassign tIds[tId]
            \State tIds[tId] \asdassign <tHeight, mergeF, query, periodicity, outmName, timerId, true, isParentSub, ptId>
        \Else:
        \State timerId \asdassign registerPeriodicTimer(ExportTreeAggTimer(tId), periodicity)
        \State tIds[tId] \asdassign <tHeight, mergeF, query, periodicity, outmName, timerId, true, false, nil>
        \EndIf\label{alg:mon:tree_agg:start_req_end}
    \asdend

    \asdupon[ExportTreeAggTimer(tId)] \label{alg:mon:tree_agg:export_trigger}
        \State <tHeight, mergeF, query, periodicity, outmName, timerId, isLocal, isParentSub, ptId> \asdassign tIds[tId]
        \If{isParentSub \&\& timeSince(tIdLastSeen[tId]) > config.treeAggExpiration}
            \If{!isLocal }
                \State tIds.delete(tId)
                \State lastSeen.delete(tId)
                \State cancelTimer(timerId)
                \State return
            \Else
                \State tIds[tId] \asdassign <tHeight, mergeF, query, periodicity, outmName, timerId, isLocal, false, nil>
            \EndIf
        \EndIf
        \State removeOldChildrenValues(childValues[tId])
        \State res \asdassign aggregateValues(mergeF, resolveQuery(query), childValues[tId])
        \If{isLocal}
            \State storeLocalVal(res, outmName)
        \EndIf 
        \If{isParentSub}
            \State sendMessage(PropagateTAggValues<ptId, res>, parent)
        \EndIf
    \asdend

    \asdupon[receive(PropagateTAggValues<tId, res>, sender)] \label{alg:mon:tree_agg:recv_propag_vals}
        \If{tId in tIds and sender in children}
            \If{tId not in childValues}
                \State childValues[tId] = map()
            \EndIf
            \State childValues[tId][sender] = res, time.Now()
        \EndIf
    \asdend

    \asdrepeateveryx{config.PropagateTAggTimeout seconds} \label{alg:mon:tree_agg:propag}
        \State toSendArr \asdassign set 
        \For{tId in tIds}
            \State <tHeight, mergeF, query, periodicity, outmName, timerId, isLocal, isParentSub, ptId> \asdassign tIds[tId]
            \If{isLocal}
                \State toSendArr.append(<max(tHeight -1, -1), mergeF, query, periodicity ,outmName, tId>)
            \EndIf
        \EndFor
        \For{c in chilren}        
            \State sendMessage(RefreshTreeAggFunc<toSendArr>, c)
        \EndFor
    \asdend

    \asdupon[receive(RefreshTreeAggFunc<tAggs>, sender)] \label{alg:mon:tree_agg:propag_recv}
        \If{parent == sender}
            \State toSendArr \asdassign set
            \For{<tHeight, mergeF, query, periodicity, outmName, ptId> in tAggs}
                \State tId \asdassign hash(tHeight + mergeF + query + periodicity + outmName)
                \If{id in tIds}
                    \State <tHeight, mergeF, query, periodicity, outmName, timerId, isLocal, isParentSub, ptId> \asdassign tIds[tId]
                    \State lastSeen[id] \asdassign time.Now()
                    \State tIds[tId] \asdassign <tHeight, mergeF, query, periodicity, outmName, timerId, isLocal, true, ptId>
                    \If{!isLocal \&\& <max(tHeight -1, -1) == -1 || <max(tHeight -1, -1) > 0}
                        \State toSendArr.append(<max(tHeight -1, -1), mergeF, query, periodicity ,outmName, timerId, tId>)
                    \EndIf
                \Else
                    \State toSendArr.append(<max(tHeight -1, -1), mergeF, query, periodicity ,outmName, timerId, tId>)
                    \State tIds[tId] \asdassign <tHeight, mergeF, query, periodicity, outmName, timerId, false, true, ptId>
                    \State registerPeriodicTimer(HandleTreeAggTimer(tId), periodicity)
                \EndIf
            \EndFor
            \For{c in chilren}        
                \State sendMessage(RefreshTreeAggFunc<toSendArr>, c)
            \EndFor
        \EndIf
    \asdend

    \end{algorithmic}
\end{algorithm}
    

Tree aggregation is started by the API sending a request to the protocol signalling the node should begin the procedure (line \ref{alg:mon:tree_agg:start_req}), this request contains: the maximum height of the tree, the merge function, the query to obtain the local value, the periodicity to execute the mechanism, and lastly, the resulting metric name. When a node receives this request, it generates a tree ID by hashing the concatenation of the tree height, the merge function, the query, the mechanism periodicity and finally the resulting metric name. This hashing is done so aggregation trees with different roots are embedded into each another, for example, if a certain node wishes to create an aggregation tree with height 5 rooted on itself, then, if a children of his (1 level lower) also creates an aggregation tree with height 4, the two trees will be embedded (as their ids will match), meaning the parent's aggregation results will be used to feed the locally requested values. It is important to notice that if the tree height is -1, then it is treated as if it had infinite height.

Before adding the tree to its local tree map, the node checks if there already is a tree with the same ID (meaning it has the same tree height, merge function, query, periodicity and resulting metric name) in the ``tIds'' map, if there is, then it sets the flag signalling the value should be exported locally, otherwise it adds a new entry to the map and sets up a periodic timer for that aggregation tree (lines \ref{alg:mon:tree_agg:start_req_start} to \ref{alg:mon:tree_agg:start_req_end}).

Whenever this timer triggers, (line \ref{alg:mon:tree_agg:export_trigger}), the node checks if the aggregation tree has expired (i.e. if the parent stopped refreshing the aggregation tree), if it it has, then the node cancels the timer and deletes the tree metadata, if it has not expired, then the node evaluates the query (this procedure will be explained in section \ref{sec:mon_module}), obtaining its local value, then it merges all the values sent by its children with its own using the merge function, producing the final aggregated result. Afterwards, if configured to do so, it will store the value locally, and if the parent is also subscribed to the aggregation tree, it propagates the obtained value to it. Whenever a node receives this propagation of values from another node (line \ref{alg:mon:tree_agg:recv_propag_vals}), it checks that it has the corresponding aggregation tree in its local set of trees and that the message was sent by its children, discarding it if it is not, and stores the received values into a map of maps, containing all children values for each aggregation tree.

Finally, each node periodically (using configured intervals) broadcasts to its children a message containing the aggregation trees it is the root of and that have height higher than 0 (line \ref{alg:mon:tree_agg:propag}). Whenever this message is received (line \ref{alg:mon:tree_agg:propag_recv}), the child merges all the received aggregation trees with registered ones, marking the parent as a subscriber for each previously present tree, and propagates the aggregation trees from which he is not the root of to its children.

\todo{talvez meter uma conclusão aqui?}

\subsection{Neighborhood aggregation}

Neighborhood aggregation is the mechanism responsible for collecting metrics from neighboring nodes to the node performing this mechanism, in essence, every node executing this mechanism creates a tree rooted upon itself by broadcasting messages with a configurable hop-based range periodically. Then, nodes who receive that message (essentially becoming federated in the tree) rebroadcast it to other peers and start to periodically propagate these values towards the root of each tree using the reverse path established by the broadcast message. Nodes in overlapping trees (i.e. propagating values derived from the same query) deduplicate the mechanisms of the propagation of broadcast messages and metric emission (by employing hashing, similarly to \ref{sec:mon_protocol:tree_agg}) in order to reduce the number of propagated messages.



\subsection{Global aggregation}

\subsection{Summary}


\section{Monitoring module}
\label{sec:mon_module}


As previously mentioned, the \textbf{monitoring module} is tasked with storing metrics, resolving queries regarding stored metrics, removing expired metrics, periodically evaluating registered alarms, and triggering callbacks which the API then propagates to the client. It is important to remember that the focus of this work is to provide a usable proof-of-concept of a decentralized monitoring framework targeted for decentralized resource management solutions. Consequently, the focus of this module is to provide just enough abstractions for a proof of concept, and aspects such as the storage of metrics in disk or the efficiency of the query language are engineering challenges that are orthogonal to the conducted work. 

This module is composed by three different components (illustrated by fig. \ref{fig:mon_module_overview}): the \textbf{query engine}, the \textbf{time-series database} (TSDB)), and the \textbf{alert manager}, whose roles in the system we now briefly explain:

\begin{figure}[htbp]
    \centering
    \includegraphics[width=\textwidth]{Chapters/mon_module/images/Monitoring_module.pdf}
    \caption{An overview of the monitoring module}
    \label{fig:mon_module_overview}
\end{figure}

\begin{enumerate}
    
    \item The \textbf{time-series database}, which allows the insertion and retrieval of time-series data. This time-series database makes use of an in-memory index to efficiently retrieve and insert data into the corresponding time-series.

    \item The \textbf{query engine} is the component tasked with resolving queries to the time-series database, in sum, it holds a set of sandboxes that evaluate user-provisioned queries that extract and apply aggregation functions to metrics present in the time-series database.
    
    \item Finally, the \textbf{alert manager} manages the alarms issued by the API, these alarms contain a condition that, when issued and triggered, should emit a notification to the issuing client. In sum, this component periodically verifies the issued queries using the \textbf{query engine} and propagates an event to the client whenever the condition is verified.
     
\end{enumerate}

In order to ease the explanation of these components, it is important to first describe what is the structure of the time-series data used in this framerwork, which has a similar structure to the metric types of InfluxDB \todo{cite}. 

\subsubsection{Metric  structure}

In DeMMon, a metric is composed by four elements: first, the \textbf{name}, which is a string containing the name of information that is being stored, it should be a human-readable name which is self-describing (e.g. ``CPU-USAGE''), second, the metric \textbf{tags}, that are a set of string pairs which denote attributes related to the metric that is stored, such as the hostname or cluster name of the node that emitted it, next, we have the \textbf{value}, which contains the data associated with the observed metric, and finally, we have the  \textbf{timestamp}, which contains the time at which the observation was taken. A typical example of a metric in the devised framework would be: name: ``CPU-Usage''; tags: <host:nodeX>, value: 0.3, timestamp: ``1609960731''.

It is important to mention that, in order to remain as flexible as possible, the metric values do not have a defined type. In this system, clients may use custom types (as long as they are marshallable using the JSON package provided by the golang package) \todo{cite}. This allows the devised framework to represent a multitude of different information types, such as histograms, strings, string maps, among others. For example, a decentralized service management system aimed at deploying service replicas in close proximity to the clients, may use, for example, a histogram of geographical locations, with pre-determined classes. This way, this system would have a data structure which would ease the process of finding a node in a certain geographical area.

Provided with the metric structure, we now explain how these are stored in the system. 

\subsubsection{Time-series database}

In DeMMon, time-series are sequences taken at successive equally spaced points in time, which in this system, are stored only in-memory. In this system, these sequences are indexed as a function of their name, periodicity, and tags.

In order for a certain metric (composed by: name, tags and periodicity) to be inserted into the database, a \textbf{bucket} must first be created, this is essentially a component that holds all time-series data with a certain name, periodicity and capacity. The periodicity of a bucket denotes the interval at which the sequences of points are spaced (in time), and the capacity denotes the number of points stored in each sequence, for example, a time-series with a 5-second periodicity and a capacity of 12 holds all points from the last minute. This allows the system to pre-allocate the memory necessary (using an array) for each time-series upon its creation.

Within a bucket, metrics are stored in a map and indexed by their tags, this is done by creating a key which creates the same key for each similar tag set, independent of its order: whenever a metric is inserted, its tags pairs are sorted alphabetically (by their key), and are concatenated into a single string, producing the resulting metric key. Then, using the metric key, the metric value is inserted into the corresponding time-series (a new time-series is created for that tag set if there was not one prevously in the system).

Time-series advance time in an on-demand manner, which means that any time-series, before returning values for a read or write request, verifies if its oldest value has a timestamp outside of the time-series window (as time has passed since the last check). If it has, the time series iterates its points from the oldest to the newest point, and removes all points which have exceeded its time window. It is important to mention that series concurrency is maintained using locking mechanisms, where operations which do not affect the state of the time series are executed concurrently, and operations which would otherwise change the time-series state are executed sequentially. 

\subsubsection{Query engine}

The query engine is a sub-component of the monitoring module, and it is responsible for evaluating the supplied text-based queries, transforming them into sets of instructions, and determining the final query result by executing the instructions. Keeping in mind the fact that the focus of this work is not the performance of the metric storage or querying modules and that it is still a focal point of this work to be as flexible as possible in the query language, we opted for using javascript-based interpreters to perform this work. This means that queries are essentially javascript code, meaning that users have infinite control on the behavior of their queries, provided these dont exceed the query timeout.

In order to provide this functionality we opted for using the package Otto \todo{insert citation}, this package provides access to javascript ``virtual machines'' that essentially take a string containing a javascript script, and build an AST from the parsed code, then this AST is executed and the result is returned by the VM, which allows a practically infinite range of query options. In order to allow users to access the time-series stored in the TSDB without directly doing so (as the user-defined queries could potentially alter internal state of time time-series), the query engine provides every Otto virtual machine access to the following functions, which return time-series from the database:

\begin{enumerate}

    \item SelectLast(Bucket\_Name, <Tag\_set\_regex>), this function returns the last point for every time-series which are in the supplied bucket that match the provided tag set regex. The way the tag set regex matching works is: for every time series present in the specified bucket, if all of the tag keys in the supplied regex match all of the time series tags, then the time series is returned. An example of the usage of this function woud be, for example: ``selectLast(CPU\_USAGE, <host:.*, cluster:cluster1>)''
    
    \item Select(Bucket\_Name, <Tag\_set\_regex) this function behaves similarly to selectLast, however it returns all points in all matched time series.
    
    \item SelectRange(Bucket\_Name, <Tag\_set\_regex>, startDate, endDate), this function behaves similarly to select and selectLast, however its arguments take an additional time range, and instead of only returning the last value it returns all points inside the supplied range.
    
\end{enumerate}

With these 3 functions, clients can select either a partial set of points or the totality of points from every time series stored in the system, these time series are then usable in the javascript code.  

We believe covers the most common use cases for metric selection. These metrics, upon selection, can then be aggregated in any way the user specifies in the query (since they are composed of user-defined code). In order to ease the design of queries and prevent developers from rewriting the same aggregation functions, the query engine also provides some aggregation primitives which can be applied to one or more timeseries such as: Max, Min and Average.

After the selection and aggregation of metrics, the resulting values are returned by placing them in a variable denoted ``result'' (this can be omitted if the query simply returns a set of values). Any query executed in deMMon can only result in one of two options: a single time series or an array of time series. Given this, in order to allow the creation of new time series during the query process, there are two additional functions available to the virtual machines: the first is called ``NewTimeSeries'', which creates a new time series, this function takes as argumments the name, tags and values which will integrate the time series; second, we have the function called ``NewObservable'' which takes the observed value and a timestamp and  to create a new metric point which can be added to time series.

 With this, we now provide some examples of possible queries along with a brief description of what they do:

\begin{enumerate}
    \item ``Avg(SelectLast(CPU\_USAGE, <host:.*, cluster:cluster1>))'' this query selects the metrics with the name ``CPU\_USAGE'' for all hosts which belong to cluster with name ``cluster1'' and returns the average of all the points.
    
    \item ``SelectLast(Nr\_services, <tenant:tenant10>, startDate, endDate)'' this query returns the timeseries for the metric called ``Nr\_services'' for the tenant with name ``tentant10'' during the provided time range.

    \item ``SelectLast(Nr\_replicas, <tenant:tenant10,service:service10>)'' this query returns the timeseries for the metric called ``Nr\_replicas'' for the tenant with name ``tentant10'' and service named ``service10'' during the provided time range.
    
    \item \todo{meter mais um exemplo de uma query custom}
    
\end{enumerate}

With this, clients are able to, through the API, obtain and manipulate data from the time series database using text-based queries. Furthermore, as the type of the value of each metric is not enforced, clients may store their metrics in custom data structures tailored for their use-cases.

\subsubsection{Alarm manager}

The alert manager is the last component of the monitoring module, it is responsible for managing the alarms issued to the monitoring module. Alarms are essentially sets of parameters which contain, among others, a condition to observe (e.g. the percentage of CPU usage). This component is essentially responsible for periodically verifying this condition and issuing notifications to the client whenever the condition is observed. Alarms are paramount to prevent applications from having to periodically access the API to verify the condition themselves, effectively saving bandwidth.

In deMMon, an alarm is composed by the following components: 

\begin{enumerate}
    \item \textbf{Query} - This denotes the query to perform periodically, this query must return a boolean value.
    
    \item \textbf{Periodicity} - The periodicity denotes how often the query is evaluated, and how often notifications are sent to the client that issued the alarm
    
    \item \textbf{Backoff time} - The backoff time is a duration decreases the rate at which the monitoring module emits notifications, which would otherwise happen at the alarm periodicity every time the alarm is verified (e.g. if the alarm periodicity is low).
    
    \item The \textbf{watch list} is a set which, for every item, contains both a name and a set of tag filters. Whenever the alarm manager receivers an alarm containing a watchList, in addition to performing the verification at the specified periodicity, it also performs the verification whenever any time series matching the watch list is changed. The rates at which the alarm is verified in this manner also respects the backoff time.
    
    \item The \textbf{CheckPeriodic} is a boolean variable denoting if the alarm should be verified periodically. When false, the alarm manager does not check the metrics at every \textbf{Periodicity} seconds, effectively saving CPU time. This option is meant to be used together with the watch list, for example, for checking a parameter which is rarely altered. It is important to mention that if this parameter is set to false, then time-based effects such as the expiration of either time series or points are not observed (as the alarm is not checked periodically).
\end{enumerate}

The monitoring module, whenever it receives a new alarm, essentially adds it to a priority queue containing all the alarms which uses the time of reception of the alarms plus their periodicity as their key to the queue. Using this data structure, alarms are sorted by the time at which they need to be verified. Then, the monitoring module continuosly obtains and removes the first item of the queue, containing the next alarm to verify out of all issued alarms. After this, monitoring module waits until it is time of verification of that alarm (i.e. the time of reception of the alarm plus its periodicity), then verifies it, and adds it to the queue with a new timestamp corresponding to the current time plus the alarms' periodicity.

Whenever the alarm is verified and the result of its query returns ``true'', the alarm manager verifies if it has emitted a notification for that alarm in the last ``Backoff time'', and issues a notification to the client if it hasn't.



\section{API}
\label{sec:api}

The API is the last module of the devised framework. As previously mentioned, the purpose of this module is to expose the functionality of the remaining components of the framework by mediating the interactions between the clients and the remaining modules via well-defined operations. In this section, we provide a brief overview of the API implementation and detail its most relevant operations. 

\subsection{Overview}

This API is coalesced by a message-based protocol performed via WebSockets \todo{citation?}. The choice of using a message-based protocol is motivated by the fact that, contrary to traditional HTTP APIs, messages enable clients to receive events sent by the DeMMon servers. This feature is essential for both alerting (as triggered alarms need to be propagated to their issuing clients) and for issuing events such as active view changes to subscribed clients. Consequently, in this API, the client must first establish a connection with the server in order to perform operations, this is done via an HTTP server, which contains a single endpoint, used for establishing the WebSockets connection. In order to test the provided API and the capabilities of the framework, we also devised a client which performs the operations, available on \todo{put repo in}.

When connected, clients and servers exchange JSON formatted messages which may contain messages related to the behaviour of two types of operations: the first is a \textbf{request}, which is similar to an HTTP request, where the client creates a request and assigns it an ID which it sends to the deMMon server via a WebSockets message. When the server receives the message, it processes it and sends back to the client a reply message containing the ID and the reply contents using the same established connection. Requests are used, for example, for querying metrics. The second type of operation is a \textbf{subscription}, which initially performs similarly to a request, hovever, in this operation the server, posterior to the initial request, may send sporadic messages to the client containing events related to the issued subscription. This type of operation is used, for example, for both clients wanting to receive active view changes or for clients installing alarms and then receiving updates to changes in these alarms.

With this, we now provide a brief overview of what we believe to be the more relevant operations exposed by the deMMon API.

\subsubsection{API operations}

\begin{enumerate}
    \item \textbf{Install or remove buckets} These operations, as their name indicates, insert or remove buckets from the time series database. As previously mentioned in section \todo{ref}, buckets are containers for all time-series data with a certain name, periodicity and capacity. Whenever a ``create bucket'' operation is issued for a bucket with a name that collides with a pre-existing bucket (with a different periodicity), an error message is returned to the client.
    
    \item \textbf{Retrieve and insert metric values}. These two operations, performed via requests, add or extract values from the time series database. The insertion of values is performed using messages the metric values. When these messages are received, the values are inserted directly in the corresponding time series. The retrieval of metric values is also performed via a request containing a query in the devised query language. This query is passed to the monitoring module (specifically the metric engine) for processing. When the query has finished being processed by the metric engine, the result is sent back to the client via a message.
    
    \item \textbf{Subscription to active view updates}. This operation, as the name denotes, is performed via a subscription, where the initial reply contains the current view of the server. Then, whenever there is a change in the view of the deMMon server, it sends a message containing the change type (if either a new connection was established to a node, or if a previous connection disappeared).
    
    \item \textbf{Install continuous query}. This operation allows the installation of a continuous query. Continous queries are operations which contain multiple parameters, and are essentially queries that are evaluated at specified periodicities, and whose returning values are inserted into the time series database under a specified name. This operation is useful for applications that wish to, for example, resample their data to longer periodicities, or wish to calculate a certain aggregated value at specific intervals. 

    \item \textbf{Broadcast messages and subscribe to broadcast message receptions}. As the name indicates, these two operations refer to issuing and receiving broadcast messages. The emission of new broadcast messages is done via a request containing the message contents, the message TTL and the message ID (a text-based field). Whenever this request is received, the API sends a request to the overlay protocol, which in turn propagates the message contents via their active connections (until the TTL is 0). Broadcast messages are propagated and forwarded to peers in the active in a similar manner to the subscription messages (described in \todo{quote broadcast aggregation}). Finally, nodes can perform \textbf{Subscriptions to broadcast message receptions}, which are subscription operations for clients that wish to receive all messages with a certain ID that pass through the server's overlay protocol. 
    
    \item \textbf{Install Alarm}. This operation is coalesced by a subscription containing the parameters described in section \todo{citar seccao alarms}, whenever a client issues this operation, the API assigns a new ID to the alarm and adds it to the alarm manager (described in section \todo{ref}), where it begins being periodically verified. After this, if in any verification the alarm fires, the alarm manager notifies the API which in turn sends a message to the client with the firing alarm's ID.

    \item \textbf{Install and removal of neighborhood aggregation set}. These operations manage the operation of the neighborhood aggregation algorithm defined in section \todo{put ref to neigh agg}. These are performed via requests, in case of installation, these contain contain the parameters for the request mentioned in \todo{another ref to place}. Whenever the api receives a request for the creation of a neighborhood aggregation, it assigns an ID to it, and sends a reply to the client with the generated ID. Conversely, when the client no longer wishes to collect these values, it issues a removal request containing originally assigned ID of the aggregation set.

    \item \textbf{Install and removal of global aggregation function}. These operation initiate and stop the global aggregation procedure (described in subsection \todo{ref section global agg}), the behavior of this operation, in regard to the interaction betweeen the client and the server, is similar to the neighborhood aggregation set. However, nodes receiving this request become roots of their own global aggregation trees.

    \item \textbf{Install and remove tree aggregation function}. This is the last detailed operation of the deMMOn API, which in terms of interaction between the client and the server, behaves similarly to both the neighborhood and global aggregation requests. Whenever the API receives a request to begin global aggregation, it forwards it to the aggregation protocol, which begins the mechanism described in \todo{insert ref}.
\end{enumerate}

\todo{falar do cliente e do exporter}

\subsubsection{Summary}

In this section, we have presented an overview of the capabilities of the DeMMon API. We began by providing a brief overview of the interaction paradigm (message-based) and the reasons behind this choice. Next, we detailed the technologies used to realize this interaction paradigm and enumerated what we believe to be its most relevant operations. For each enumerated operation, we provided a brief explanation of its behaviour regarding both the effect on the remaining components and the model of interaction between the client and the server. 

\section{Summary}

In this chapter, we covered the implementation of the DeMMon framework, a decentralized management and monitoring framework targeted for the operation of decentralized resource management systems. We began by covering what we believe to be the requirements of this solution (beggining of chapter \ref{cha:demmon}). Then, we detailed the implementation and design of Go-Babel (Section \ref{sec:GO-Babel}), which we used to develop the overlay network. Following, we provided a brief overview of the four modules which compose this framework, beginning with the overlay network (Section \ref{sec:overlay_network}), which is responsible for creating and maintaining a multi-tree shaped network, optimized using latencies and node capacity. Following, in Section \ref{sec:mon_protocol} we covered the aggregation protocol, which provides multiple primitives for collecting and aggregating metrics in a decentralized and efficient manner, using in-transit aggregation, from a partial (or complete) set of nodes in the tree-shaped network. Next, we covered the monitoring module (Section \ref{sec:mon_module}), which is the module responsible for enabling the storage and retrieval of metrics, parsing and processing queries and managing alarm lifecycles. Finally, we finished the DeMMon implementation by covering the API (Section \ref{sec:api}), which essentially is the module responsible for mediating, via a WebSockets interface, the aforementioned interactions between the external clients and the other modules.
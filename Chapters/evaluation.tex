%!TEX root = ../template.tex
%%%%%%%%%%%%%%%%%%%%%%%%%%%%%%%%%%%%%%%%%%%%%%%%%%%%%%%%%%%%%%%%%%%%
%% evaluation.tex
%% NOVA thesis document file
%%
%% Chapter with lots of dummy text
%%%%%%%%%%%%%%%%%%%%%%%%%%%%%%%%%%%%%%%%%%%%%%%%%%%%%%%%%%%%%%%%%%%%

\typeout{NT FILE evaluation.tex}

\chapter{Evaluation}
\label{cha:evaluation}

In this chapter, it is our goal to demonstrate the applicability of the devised solution through the comparison of multiple aspects of the framework against popular baseline solutions from the literature. To this end, section \ref{sec:exp_setting_conf} covers the experimental setting and configuration in which these comparisons were conducted, namely the execution environment, the specifications of the nodes executing the tests, among other aspects of the experimental setting. Following, in section \ref{sec:overlay_proto_eval}, we provide the obtained results from the conducted experiments testing the applicability of the devised protocol against state-of-the-art baselines. We compare these baselines with our protocol both in their capacity to create and establish the network and in their capacity to perform information dissemination. Lastly, section \ref{sec:agg_proto_eval} covers the experimental evaluation of the implemented aggregation protocol, notably, which solutions composed the baseline for comparison, which experiments were carried, and the obtained results.

\section{Experimental Setting} \label{sec:exp_setting_conf}

To conduct the experimental evaluation of the devised solution against state-of-the-art baselines, instead of resorting to simulation, we implemented those solutions and tested them in an emulated network that aims to be as similar as possible to real-world scenarios. Provided that scalability is one of the components we aimed to test, and there is a limited pool of individual machines in our testbed to conduct the experiments, we resorted to using containerization. Containers allowed us to execute multiple independent processes in a single physical node while still being an isolated environment that allowed the manipulation of the networking conditions of each process. 

As containers are running in different machines, without any additional software, a container from a machine would not be able to communicate with containers executing in a different machine. To overcome this, we made use of Docker \cite{docker} containers and employed a tool called docker swarm \cite{docker-swarm}. This tool allows users to coordinate sets of nodes running Docker. Nodes in a swarm, among many other features, may perform Multi-host networking, which consists in integrating the containers executing among nodes running into a unified network. In this network, containers are automatically assigned IP addresses and can communicate with each other, regardless of the physycal machine each container is executing on. To bootstrap the experimental scenario, we developed a set of scripts in both BASH, Python and GO to create, orchestrate, and decommission containers to run the experiments.

\subsection{Node capacity and connection delays}

As previously mentioned, in order to emulate a real-world scenario where nodes have limited capacity and their connections have delays, there was the need to apply these constraints. Furthermore, it is important to set up these delays realistically. To do so, we used data from real-world readings of real-world scenarios obtained from WonderNetwork \cite{wondernetwork}, which consists in a network composed of 252 nodes spread across 88 countries in 6 continents. This network provides, in addition to node metadata (city, country, among others), a set of latency measurements from each node to every other in the network (including themselves), which we used to setup the latency delays between the containers. As there was the need to test the framework with larger network sizes (of up to 750 nodes), the data points from this network were multiplied by five times. 

Then, as the obtained data from this network did not contain bandwidth information for each node, we used the metadata provided by WonderNetwork, namely the country, to assign bandwidth values according to the list of bandwidth per country provided by speedtest.net \cite{speedtest_global_index}. Provided the purpose of this framework is to perform on cloud-edge scenarios, composed by nodes inside and outside of the data-centre (DC), where nodes outside the DC have lower networking capacity comparatively to nodes running inside the DC, we divided each data point by 12x (to represent the edge nodes running outside of the DC), and divided the first N nodes by 2.5x, (corresponding to the number of data centres).

% Provided with the networking capacity and the latency matrix for all connection pairs, there was the need to both

To limit the networking capacity and inject latency in the containers executing the protocols for the experiments, we used a tool called Traffic Control (TC) \cite{tc_man_page}. This tool is a traffic shaping tool that performs shaping, scheduling, policing and dropping of network packets through the configuration of the kernel packet scheduler. 

% This tool sets up queuing systems and mechanisms by which packets are received and transmitted. 
% then queue (also denominated qdisc) or class specific policies decide which (and whether) packets to accept at what rate on the input of an interface and determining which packets to transmit in what order at what rate on the output of an interface. 

In our case, we used this tool to limit both the available inbound/outbound bandwidth on the container interfaces and to inject delays in all connection pairs. In order to limit the available bandwidth, we used hierarchical token buckets (HTB), which are classful queueing disciplines that employ a complex token-borrowing system to ensure the shaping of traffic according to a (configurable) rate. HTB requires programmers to set up a hierarchical class structure, where child classes, attached to a queue (or qdisc), manipulate packet order and apply rate limiting policies according to configuration.

For our benchmark, we made use of rate-limiting policies, which employ a token borrowing mechanism that functions in the following manner: whenever a certain child class reaches the maximum of its rate, it borrows tokens (up to its \textbf{ceiling} value) from the parent class (if there is a parent class, and that class has available tokens). If the parent class is also limited, then the sum of its child classes will be limited to the parent rate. 

For our experimental configuration, each container creates two default qdiscs, attached to the inbound and outbound networking interfaces. Then, two HTBs are attached to these qdiscs and set up with the respective inbound and outbound bandwidth rate. After this, for both the outbound and inbound classes, two child classes are installed: one intended for latency measurements and keepalive traffic (specific UDP traffic on a pre-configured port); and the other for the remaining traffic. The inbound and outbound classes responsible for measurement traffic are assigned a fixed rate of 500kb, and the inbound and outbound default traffic classes are assigned the remaining rate for the container (according to speedtest.net) minus the 500kb rate for the measurements class. Then, for all the outbound classes (measurement and default traffic), we set up another set of HTB classes for each other container with a very low rate of 6kb and ceiling rate corresponding to the parents' class. This setup forces child classes to borrow tokens from the parent class and be limited by its bandwidth rate.

For each of these leaf classes, we attached a netem qdisc. This qdisc applies a delay to each packet according to the latency measurements taken from WonderNetwork. To route packets from one qdisc to the other, we use filters: in the case of the measurement traffic, the filtering was performed via installing a high-priority filter that verifies the source and destination ports of the packets and sends it to the measurement classes. The remaining traffic is forwarded to the default traffic class via a lower priority filter with no restrictions. After this, the routing from these two outbound classes to the leaf classes is performed via filters observing the destination IP address of the packets and redirecting them to their corresponding netem qdiscs.  The objective of separating the traffic into two distinct classes with their own bandwidth values is to prevent cases where the applicational traffic is high (i.e. testing information dissemination) and the delay caused by the high usage of the data channels would interfere with the measurement packets, leading to incorrect latency measurements and consequent instability during experiments for both DeMMon and the baseline overlay protocols that optimize the overlay. 

Experiments presented in this work were carried out using the Grid'5000 testbed, supported by a scientific interest group hosted by Inria and including CNRS, RENATER and several Universities as well as other organizations (see https://www.grid5000.fr). The hardware from this testbed used to carry the experiments consisted in sets of 10 physical nodes for the experiments with 50 and 250 logical nodes and sets of 30 physical nodes for experiments with both 500 and 750 logical nodes. Each of these physical machines is equipped with 2 x Intel Xeon E5-2630 v3 and 128 GiB of RAM and is executing Linux Debian version 4.19.104-2 and Docker version 20.10.7. The results were obtained through logging the relevant aspects of the experiment to disk and then processing the obtained logs to extract the intended information posterior to the end of the experiments.

Provided with the experimental setup, we now explain the steps taken and the results obtained for the DeMMon framework evaluation.

\section{Overlay Protocol} \label{sec:overlay_proto_eval} 
In this section, we present the results obtained from the multiple conducted experiments of the overlay protocol against state-of-the-art baselines. These experiments aimed at testing: (1) the cost of establishing/maintaining the overlay networks for each protocol; (2) testing the established networks' efficiency (according to latency); and (3) finally, testing DeMMons' message dissemination capabilities against that same set of baseline protocols. In this last test, the baseline membership protocols are paired with two distinct dissemination protocols to perform the message dissemination. We now begin by providing a brief discussion of the protocols and parameters used for conducting the experiments.

\subsection{Baselines and configuration parameters}

The chosen protocols to perform the overlay protocol network comparison were \textbf{Hyparview} \cite{Hyparview}, \textbf{X-Bot} \cite{x-bot}, \textbf{Cyclon} \cite{voulgaris2005Cyclon} and \textbf{T-Man} \cite{t-man}. We now provide a brief description of each (further discussion is provided in Section \ref{sec:topology_management}).

\textbf{Hyparview} is a protocol that builds a non-structured overlay network using a fixed-sized view materialized by active bidirectional TCP connections (these connections also serve as fault detectors). 

The second baseline protocol is \textbf{X-Bot} \cite{x-bot}, which is a protocol that essentially behaves similarly to Hyparview in terms of establishing the initial network structure but takes iterative steps to optimize the overlay network (according to a configurable heuristic). These steps are performed via gossip mechanisms to improve the nodes' active connections' costs. In X-Bot, nodes perform optimizations in such a way that maintains the in and out-degree established initially. 

The third implemented baseline protocol was \textbf{Cyclon} \cite{voulgaris2005Cyclon}, which is an overlay protocol that materializes a network composed of asymmetric links via periodic exchanges of node pointers with a configurable age. 

The last implemented baseline was \textbf{T-Man} \cite{t-man}, which is a protocol that iteratively builds on an existing set of nodes to build a new, more optimized set of nodes. These optimizations are performed iteratively by each node in the system such that a configurable cost function (defined a priori) gets minimized. In order to feed the initial node sample for T-man to optimize, we employed the Cyclon protocol, and consequently, the evaluation results for this protocol are labelled as ``Cyclon T-Man''. All of the described baseline protocols were, for comparativeness, implemented using the GO-Babel framework (described in Section \ref{sec:Node-Watcher}). As Go-Babel only provides unidirectional connections, protocols that require bidirectional connections, namely Hyparview, X-Bot and DeMMon, were enriched with added mechanisms to ensure that the bidirectional connections were established for each node contained in nodes' active views.

The utilized parameters for the different protocols were tuned to attempt to perform a fair comparison. These are displayed in table \ref{table:proto_test_params}, we now describe each of the columns. The first is called ``VSizeMax'', and represents the maximum size of the active view, which in most protocols was set to 5, except for Cyclon, where it was set to 7 as it is the only protocol without a secondary backup view. In the case of DeMMon, ``VSizeMax'' represents the maximum number of children per node. 

The second column, named ``VSizeMin'', corresponds to the minimum number of children for each node in DeMMon. The third parameter, titled ``PVSizeMax'', corresponds to the maximum size of the passive view, which is set to 25 for DeMMon, Hyparview and X-BOT, and set as 7 for the case of Cyclon T-Man. In the case of T-man, this parameter corresponds to the size of the Cyclon view (running to provide its initial view). The next parameter, labelled ``Shuffle'', corresponds to the periodicity of each protocols' shuffle mechanism. This parameter is set to 5 seconds for all protocols. Following, we have the ``PWRL'' parameter, which corresponds to the TTL of the random walks (for each protocol that has a random walk mechanism). The last-mentioned parameter is the ``$\delta$T'' parameter, which corresponds to the minimum latency improvement for both X-Bot to perform active view exchanges and for DeMMon to make opportunistic improvements. Some parameters such as timeouts and the duration of some periodic procedures were omitted, however, all timeouts, e.g. timeouts for dialling nodes, receiving message responses, among others, are set to 5 seconds. Furthermore, all periodic mechanisms are executed with a frequency lower than 15 seconds.

\begin{table} \label{table:proto_test_params}
    \caption{Membership evaluation: protocol configuration parameters}
    \scalebox{1}{
    \centering
    \resizebox{\textwidth}{!}{%
    \begin{tabular}{llllllllllll}
                & VSizeMax & VSizeMin & PVSizeMax & Shuffle $\delta$T (s) & PRWL & ARWL & ka & kp & improvement  $\delta$T (ms) & UN & PSL \\
                Hyparview    & 5        & -        & 25        & 5                     & 6    & 3    & 2  & 3  & -                            & -  & -   \\
                X-Bot        & 5        & -        & 25        & 5                     & 6    & 3    & 2  & 3  & 50                           & 1  & 2   \\
                Cyclon       & 7        & -        & -         & 5                     & -    & -    & -  & -  & -                            & -  & -   \\
                Cyclon T-Man & 5        & -        & 7         & 5                     & -    & -    & -  & -  & -                            & -  & -   \\
                DeMMon       & 5        & 2        & 25        & 5                     & 6    & -    & -  & -  & 50                           & -  &    
    \end{tabular}%
    }
    }
\end{table}

\subsection{Overlay construction and maintenance}

The first conducted experiment, aimed at evaluating how protocols establish and maintain the overlay network, consists of an experiment where different numbers of nodes join the system and remain for 25 minutes. In this experiment, we evaluate the properties of the built overlay networks (costs, degree distribution, among other properties) and how fast the protocol converges towards an optimized network. Finally, to compare the scalability, performance and fault-tolerance at multiple scales, we performed the previously mentioned experiment using network sizes of 50, 250, 500 and 750 nodes along with two failure rates of 0 and 50\%.

\begin{figure}
    \centering
    \includegraphics[width=\linewidth]{Chapters/evaluation/figures/membership/membership_lat_over_time_0_failures.pdf}
    \caption{Average latency per node in established networks}
    \label{fig:overlay_proto_res_net_building:0_failures_lat}
\end{figure}


\begin{figure}
    \centering
    \includegraphics[width=\linewidth]{Chapters/evaluation/figures/membership/membership_total_lat_over_time_0_failures.pdf}
    \caption{Total network cost (in latency)}
    \label{fig:overlay_proto_res_net_building:0_failures_lat_total}
\end{figure}

In the graphs displayed in Figures \ref{fig:overlay_proto_res_net_building:0_failures_lat} and \ref{fig:overlay_proto_res_net_building:0_failures_lat_total}, we may observe the results pertaining to the average latency of a connection in the overlay and the total cost of the established overlay networks for the experiment with no failures, respectively. For both of these graphs, we show the results obtained from both the baseline protocols and the DeMMon protocol. In the case of DeMMon, we make a distinction between two latency values, the first (represented by a blue, continuous, line) represents the results relative to all connections of all nodes, the second value (represented by a blue dashed line) represents the cost of the ``vertical'' connections of the DeMMon tree (the parent and children of each node), essentially excluding the siblings of each node from the results. We made this distinction for two reasons: first, as the DeMMon protocol only performs optimizations to improve the parent connection, we believe it is important to see the correlation between improving the parent connections to the sibling latencies. The second reason to make this distinction is that these connections are more used when compared with the sibling connections, namely for network maintenance, information dissemination and in-transit aggregation. The results displayed in these graphs (\ref{fig:overlay_proto_res_net_building:0_failures_lat} and \ref{fig:overlay_proto_res_net_building:0_failures_lat_total}) show that both Hyparview and Cyclon converge to a similar average latency value, which corresponds to the average of all connections of the latency matrix. This is expected, as these protocols do not attempt to perform optimizations in regard to the network latency. The results also show that the devised protocol is the fastest to converge to its lowest latency value and that X-Bot is the slowest, not converging to a final value in a test of 25 minutes. We believe this occurs because X-Bots' overlay improvements are performed using 7 messages, contrasting heavily with DeMMons' 2 required messages, and T-Mans' 0 required messages. While the total and average latency of the DeMMon overlay is not the lowest in any of the displayed results, when comparing only the parent and children connections, DeMMon reaches a total latency cost lower than any other tested protocol. This is important given that, as previously mentioned, these connections are the ones most used when performing overlay improvements and maintenance, information dissemination and in-transit aggregation. 

\begin{figure}
    \centering
    \includegraphics[width=\linewidth]{Chapters/evaluation/figures/membership/membership_inDegree_0_failures.pdf}
    \caption{Node in-degree}
    \label{fig:overlay_proto_res_net_building:0_failures_inDegree}
\end{figure}

It is important to mention that, while T-Man is the protocol that reaches the lowest overall and average latency in the conducted tests, it does so disregarding the fact that nodes may become disconnected from overlay, which as we will observe further, prevents this protocol from being suitable for reliable message dissemination. This may be observed in fig. \ref{fig:overlay_proto_res_net_building:0_failures_inDegree}, which shows the in-degree (the number of incoming connections for each node) for all nodes participating in the network, these results pertain to the last observed configuration of the network before the experiment finished. They show that T-Man, at multiple node counts, possesses nodes with 0 incoming connections, which are effectively isolated from the network. While still analyzing the in-degree results, we observe that both X-Bot and Hyparview have a fixed number of incoming connections, which stems from the use of bidirectional connections, while Cyclon has varied numbers of incoming connections ranging from 10 to 1, which occurs due to the shuffle mechanisms of the active connections. In the case of DeMMon, the values range from 2 to 10 incoming connections, which is expected given the configuration parameters of a minimum number of children of 2, and a maximum number of children of 5.


\begin{figure}
    \centering
    \includegraphics[width=\linewidth]{Chapters/evaluation/figures/membership/membership_bw_over_time_0_failures.pdf}
    \caption{Protocol bandwidth cost}
    \label{fig:overlay_proto_res_net_building:0_failures_BWUsage}
\end{figure}

Finally, still regarding the experiments without node failures, we show, in Figure \ref{fig:overlay_proto_res_net_building:0_failures_BWUsage}, the average network cost (in kbit/5s) incurred by each node running the experiments. This graph shows that DeMMons' overlay protocol, on average, spends more bandwidth to build and maintain the network structure, we believe this is because DeMMon exchanges more information periodically with peers in the active view to maintain and improve the network structure when compared to the other protocols. Conversely, the protocol which uses the least amount of bandwidth is Cyclon, as its shuffle mechanism is relatively inexpensive, and the protocol has no other mechanisms that incur networking costs. Although protocols have varied networking costs, we believe that even DeMMon, which uses the most bandwidth, is relatively inexpensive when compared with the bandwidth standards at the time of writing this work.

\begin{figure}
    \centering
    \includegraphics[width=\linewidth]{Chapters/evaluation/figures/membership/membership_inDegree_50_failures.pdf}
    \caption{Node in-degree (50\% failures)}
    \label{fig:overlay_proto_res_net_building:50_failures_inDegree}
\end{figure}

Provided with the result analysis for the experiments with no failures, we now provide the results for the in-degree distribution of the protocol in a scenario with failures. This experiment attempts to test the fault tolerance of the protocols by first establishing the network, and during the middle of the experiment, induce a failure of 50\% of the nodes. The objective of this experiment was to test if any node became isolated from the network after the failures. Results from this experiment may be observed in Figure \ref{fig:overlay_proto_res_net_building:50_failures_inDegree}, where it is observable that, for all tested protocols except T-Man, no nodes became isolated, allowing us to conclude that both the devised protocol and the tested baselines can recover from faults effectively.

As previously mentioned, the applicability of our solution was tested in two different aspects: the first was the process of building and maintaining the overlay network, which was covered in the previous paragraphs. The second evaluated aspect is information dissemination (via message broadcasting), which we will now cover in the following subsection.

\subsection{Information dissemination} \label{results:inf_diss}

The second set of conducted experiments, as mentioned previously, intends to test the applicability of the devised membership protocol in an information dissemination scenario. To do so, we tested it against the same set of baseline protocols used in the previous experiments enriched with two message dissemination protocols: the first is a simple flood protocol, where if a node wishes to broadcast a message, it sends that message to every peer in its active view, then, nodes that receive this message, propagate it to every neighbour if they have not done so previously (excluding the sender). The second used dissemination protocol was PlumTree \cite{plumTree}, which is a dissemination protocol that builds a dissemination tree based on the paths taken by the broadcast messages.

The reasoning behind this choice of dissemination protocols was to provide a more comprehensive comparison of DeMMon with the remaining protocols. As the simple flood generates redundant messages when compared to dissemination primitives tree structures, we included a dissemination protocol that (similarly to DeMMon) also employs a tree for the dissemination of messages. It is important to mention that, when testing the PlumTree protocol, in order to establish the initial tree structure, a single node first starts the dissemination of its messages a minute earlier than other nodes. For both of these comparisons, DeMMon is set up with a dissemination protocol similar to the simple flood protocol, however only using its vertical connections (parent and children).

Similarly to the first set of experiments, we conducted multiple tests with 50, 250, 500 and 750 nodes during 15 minute periods. For all these system sizes, we also tested failure rates of 0 and 50\%. For each of these combinations, we varied the number of messages each node emitted until all protocols reach their saturation point. While doing the tests, we extracted the following metrics: (1) the reliability of the messages, i.e. what is the average percentage of nodes that receive the emitted broadcast messages ; (2) the maximum message throughput reached by every protocol in a 30-second window, (3) the average latency taken by messages until they reach other nodes, and (4) the bandwidth usage of each of the protocols. 

\begin{figure}[htbp]
    \centering
    \includegraphics[width=\linewidth]{Chapters/evaluation/figures/flood/flood_0.0_failures_reliability.pdf}
    \caption{Average message reliability in simple flood scenario (0\% failures)}
    \label{fig:overlay_proto_res_msg_diss:0_failures_reliability_flood}
\end{figure}

\begin{figure}[htbp]
    \centering
    \includegraphics[width=\linewidth]{Chapters/evaluation/figures/flood/plumTree_0.0_failures_reliability.pdf}
    \caption{Average message reliability in PlumTree scenario (0\% failures)}
    \label{fig:overlay_proto_res_msg_diss:0_failures_reliability_plumTree}
\end{figure}

Figures \ref{fig:overlay_proto_res_msg_diss:0_failures_reliability_flood} and \ref{fig:overlay_proto_res_msg_diss:0_failures_reliability_plumTree} show the obtained results regarding the message reliability during the experiments for both the simple flood and PlumTree experiments with 0 failures. As we may observe, in general, the saturation point for all protocols using PlumTree tends to be earlier (in terms of emitted messages per node) than the simple flood protocol. We believe this occurs because the PlumTree protocols' tree becomes unstable whenever certain nodes become a bottleneck to the messages being propagated using the tree (because their bandwidth capacity is exceeded). Whenever this occurs, as certain messages get delayed, the tree structure becomes unstable (as the order of delivery of messages is what defines the dissemination tree structure). Whenever this occurs, the tree repair procedure is triggered, however as multiple nodes emit new messages and other nodes can become saturated while performing this mechanism, the tree structure may never reconverge until all messages are delivered (and nodes stop being saturated). Until this occurs, the protocol essentially becomes a push-pull gossip protocol, which has lower performance in our experiments in terms of reliability because message delivery requires 2 messages, incurring additional networking costs (and the tests end before the protocol has delivered all emitted messages).

In addition to the previously mentioned reasons, in an occasion where a node has received an IHAVE message for a certain message and at that moment happens to have available upload bandwidth, but its download capacity is all taken up by incoming traffic, this node will periodically emit GRAFT messages to the sender of the IHAVE message, which, in turn, will reply with the broadcast messages that will only be received after a large time frame. During this time frame, there may be multiple redundant GRAFT and IHAVE messages being emitted, which results in the system possibly becoming even more saturated, which causes the tree to become even more unstable. The devised overlay protocol, although it also uses a tree structure, its tree is not defined by the propagations of broadcast messages and consequently is not as susceptible to instability in conditions where the network is saturated, consequently achieving higher reliability in higher message counts. 

We may observe that both the Cyclon and T-Man tend to perform worse in general regard to reliability when compared to DeMMon, Hyparview and DeMMon, which we believe, in the case of T-Man, to occur because there are nodes with 0 incoming connections and consequently do not receive any broadcast messages from other nodes. In the case of Cyclon, we believe the lower reliability value is attributed to the use of UDP as its communication medium, which means that whenever the data channels become saturated, many of the broadcast messages are lost, contrary to DeMMon, Hyparview and X-BOT, that use TCP and consequently do not drop messages in congestion periods. Finally, we believe that both Cyclon and T-MAN, when paired with PlumTree, also have lower reliability because this protocol requires bidirectional connections to perform optimally, which are not guaranteed in either of these protocols. 

In regard to the simple flood experiment (fig. \ref{fig:overlay_proto_res_msg_diss:0_failures_reliability_flood}), DeMMon tends to perform exceptionally well with fewer node counts, particularly with 50 nodes. We believe this may be due to the height of the DeMMon tree being lower, as when the tree height is smaller, the number of descendants for each node is fewer, which in turn means that when a certain node becomes saturated, fewer nodes are impacted by it. In higher node counts, DeMMon performs in line with both Hyparview and X-Bot. We believe this happens because the tradeoffs of having a tree (a single node possibly becoming a bottleneck for many other nodes in the system) tend to impact the system the same amount that sending multiple redundant messages does. 

\begin{figure}[htbp]
    \centering
    \includegraphics[width=\linewidth]{Chapters/evaluation/figures/flood/flood_50.0_failures_reliability.pdf}
    \caption{Average message reliability in simple flood scenario (50\% failures)}
    \label{fig:overlay_proto_res_msg_diss:50_failures_reliability_flood}
\end{figure}

\begin{figure}[htbp]
    \centering
    \includegraphics[width=\linewidth]{Chapters/evaluation/figures/flood/plumTree_50.0_failures_reliability.pdf}
    \caption{Average message reliability in PlumTree scenario (50\% failures)}
    \label{fig:overlay_proto_res_msg_diss:50_failures_reliability_plumTree}
\end{figure}

In the case of scenarios with induced failures (Figures \ref{fig:overlay_proto_res_msg_diss:50_failures_reliability_flood} and \ref{fig:overlay_proto_res_msg_diss:50_failures_reliability_plumTree}), we observe a similar trend in regard to the PlumTree experiments, with DeMMon achieving higher reliability values. However, in the simple flood experiments, we observe that DeMMon achieves a lower reliability value when under congestion, we believe this occurs because as the failures are occurring, if the nodes are saturated, the failure recovery mechanisms may take a long time frame to execute, and during this period nodes are disconnected from the remaining overlay and consequently do not receive or send message to or from any node which is not their descendant, leading to a lower reliability value.

Provided the results from the combination of the baseline protocols with PlumTree consistently performs worse in terms of reliability (when the network is saturated) when compared to employing only a simple flood protocol, we now focus on the comparison between DeMMon and the baseline protocols executing the simple flood protocol, however for completeness, all obtained results are available in annex \ref{annex_1}.

\begin{figure}[htbp]
    \centering
    \includegraphics[width=\linewidth]{Chapters/evaluation/figures/flood/flood_0.0_failures_msg_lat.pdf}
    \caption{Average message latency (in ms) in simple flood scenario (0\% failures)}
    \label{fig:overlay_proto_res_msg_diss:0_failures_latency_flood}
\end{figure}

In Figure \ref{fig:overlay_proto_res_msg_diss:0_failures_latency_flood}, we may observe the obtained results from collecting the latency between the emission and reception of broadcast messages for each node. The first takeaway from these results is that all protocols plateau at the same latency value, we believe this is due to the fact the the test times are limited to 15 minutes, and whenever the system is saturated, all messages tend to take a similarly long time to be delivered, those which are not delivered are only reflected in the previously discussed reliability graphs (Figures \ref{fig:overlay_proto_res_msg_diss:0_failures_reliability_flood}, \ref{fig:overlay_proto_res_msg_diss:0_failures_reliability_plumTree}, \ref{fig:overlay_proto_res_msg_diss:50_failures_reliability_flood}, and \ref{fig:overlay_proto_res_msg_diss:50_failures_reliability_plumTree}). However, for lower message counts, the latency results show that DeMMon tends to achieve lower latency values when compared with the baseline protocols on certain workloads (e.g. low numbers of messages emitted on both the 50 and 750 node graphs) where we believe the simple flood protocol becomes saturated due to the number of redundant messages sent.

\begin{figure}[htbp]
    \centering
    \includegraphics[width=\linewidth]{Chapters/evaluation/figures/flood/flood_0.0_failures_throughput.pdf}
    \caption{Maximum message throughput during experiment (30 second window) in simple flood scenario (0\% failures)}
    \label{fig:overlay_proto_res_msg_diss:0_failures_throughput_flood}
\end{figure}

In Figure \ref{fig:overlay_proto_res_msg_diss:0_failures_throughput_flood}, we may observe the obtained throughput across the message dissemination experiments for the simple flood protocol with 0 failures. As we can observe, in lower node counts (i.e. 50 nodes), the throughput achieved by DeMMon surpasses the throughput achieved by the remaining protocols, which also explains the higher values of reliability achieved by DeMMon in these node counts (see fig. \ref{fig:overlay_proto_res_msg_diss:50_failures_reliability_flood}). However, at higher node counts, all protocols tend to plateau at the same throughput, which we believe to be attributed to the fact that, as previously mentioned, the tradeoffs of using a tree (a single node possibly becoming a bottleneck for many other nodes in the system) tends to impact the system the same amount that sending multiple redundant messages does.

\begin{figure}[htbp]
    \centering
    \includegraphics[width=\linewidth]{Chapters/evaluation/figures/flood/Message latency comparison.jpg}
    \caption{Message latency distribution in scenario with low network saturation}
    \label{fig:overlay_proto_res_msg_diss:0_failures_msg_lat_histogram}
\end{figure}


\begin{figure}[htbp]
    \centering
    \includegraphics[width=\linewidth]{Chapters/evaluation/figures/flood/Message hop distribution.jpg}
    \caption{Message hop distribution in scenario with low network saturation}
    \label{fig:overlay_proto_res_msg_diss:0_failures_msg_hop_histogram}
\end{figure}

In Figure \ref{fig:overlay_proto_res_msg_diss:0_failures_msg_lat_histogram} we compare the baseline protocols with DeMMon in regard to the message latency. These results show the averaged latency distribution for the tests conducted with 250 nodes and one message emitted per node. On the left graph, we may observe the results obtained by the execution of PlumTree with the baseline protocols, while on the right graph, we have the results for the simple flood tests. As we can observe, in general, the message latency obtained by combining simple flood with the baseline protocols tends to be lower in latency when compared with protocols that employ shared trees to disseminate the messages, such as PlumTree and DeMMon. We believe this can be explained by the fact that, by employing a single shared tree to disseminate the messages, as the messages must take specific routes in order to reach all nodes with decreased message redundancy, messages have to take more hops to get to their destination, and consequently achieve higher latency values. This behaviour is observable in Figure \ref{fig:overlay_proto_res_msg_diss:0_failures_msg_hop_histogram}, which shows the hop distribution of the delivered messages in the same scenario of 250 nodes and on message emitted per node.

It is important to mention that, while Cyclon with T-Man achieves lower latency values in both tests, it does so at the cost of reliability, making it less applicable for a reliable broadcasting solution (as observed previously in the results displayed in fig. \ref{fig:overlay_proto_res_msg_diss:0_failures_reliability_flood} and \ref{fig:overlay_proto_res_msg_diss:0_failures_reliability_plumTree}.


\subsection{Summary}

In this section, we covered the obtained results from the experimental evaluation of the devised membership protocol against multiple popular baseline protocols obtained from the study of the state-of-the-art. Two main aspects of the devised protocol were tested (at multiple scales): the first aspect was the ability to establish and maintain the overlay connections, where obtained results show that the devised protocol is consistently one of the fastest protocols to converge to a final topology. Furthermore, in regard to the latency values of the vertical connections of the established tree (excluding connections between nodes sharing the same parent in the tree, which are less used in general), DeMMon also achieves both the lowest average and total latency cost.

The second tested aspect of the devised protocol was their message dissemination capacity, where the devised protocol was evaluated against the previously mentioned benchmarks paired with two flood protocols: a simple flood and the PlumTree protocol. We conducted tests at both multiple scales and multiple failure rates and observed that while DeMMon tends to perform particularly well in regard to throughput at lower scales (50 nodes) when compared with any other tested protocol, while at larger node counts, its throughput tends to plateau at around the values as both X-Bot and Hyparview when paired with a simple flood. We also observed that, while tree topologies (both DeMMon and PlumTree) incur lower message redundancy, the use of a single shared tree for scenarios with multiple senders causes higher delays in messages when compared to simple flood alternatives, as messages take more hops to reach their destination.

To conclude, we believe the devised overlay performs competitively with popular state-of-the-art solutions for both creating and establishing an overlay network and for performing information dissemination. Conducted tests suggest that DeMMon performs better in saturation tests at lower node counts, indicating it as the most performant solution for these scenarios. However, for scenarios where message latency is a concern, results show that any tree approach (including DeMMon), although incurs lower networking costs, performs worse when compared to simple flood protocols.

% We now discuss the remaining obtained results from testing the baseline protocols executing a simple flood protocol against DeMMon, beggining with the obtained latency values for the scenario with no failures, detailed inFigure\ref{}. 

% \newgeometry{margin=1cm}
%     \begin{landscape}
%         \thispagestyle{empty}
%         \begin{figure}
%             \centering
%             \includegraphics[width=\linewidth]{Chapters/evaluation/figures/membership_0_failures.pdf}
%             \caption{Overlay protocol comparison with failures 0 failures for network sizes of 50, 250, 500 and 750 nodes.}
%             \label{fig:overlay_proto_res_net_building:0_failures}
%         \end{figure}
%     \end{landscape}
% \restoregeometry

% \newgeometry{margin=1cm}
%     \begin{landscape}
%         \thispagestyle{empty}
%         \begin{figure}
%             \centering
%             \includegraphics[width=\linewidth]{Chapters/evaluation/figures/membership_50_failures.pdf}
%             \caption{Overlay protocol comparison with failures 50\% failures for network sizes of 50, 250, 500 and 750 nodes.}
%             \label{ig:overlay_proto_res_net_building:50_failures}
%         \end{figure}
%     \end{landscape}
% \restoregeometry


\section{Aggregation Protocol} \label{sec:agg_proto_eval} 

In this section, we present and analyze the obtained results from the experimental evaluation of the devised decentralized aggregation protocol when compared with a popular monitoring solution from the state-of-the-art: named Prometheus \cite{prometheus}. We begin by providing the experimental setting and configuration settings used across the conducted experiments, then we present and discuss the obtained results from these experiments, and finish the Section by providing a summary along with the drawn conclusions from the evaluation of our solution in its monitoring and aggregation capacity.

The experimental setting in which the evaluation of our aggregation protocol was conducted on is the same as the one defined in \ref{sec:exp_setting_conf}, where each solution is tested using containers to multiplex the physical nodes, isolate the running processes, and apply both bandwidth capacity constraints and latency delays between nodes.

As previously mentioned in Section \ref{sec:mon_protocol}, the devised aggregation protocol offers three decentralized information collection primitives: neighbourhood, tree and global aggregation. In this section, we provide the obtained results regarding the comparing the tree and global aggregation features with comparable setups running Prometheus. Neighbourhood aggregation results are not shown as Prometheus does not provide a comparable feature. For all the conducted experiments, we tested the systems by collecting a certain aggregated value, calculated through the aggregation of a variable number of metrics, emitted at configurable intervals by dummy applications running in all the nodes of the system. The main criteria used to test the applicability of our solution was its error over time: obtained by comparing the aggregated value obtained by each node against their ``supposed'' value, according to the following formula: 

\[ Error(t) =  \frac{|\sum localVal_i(t) - aggVal(t)|}{\sum localVal_i(t)}\]

, where $localVal_i$ corresponds to the emitted value of each node locally, $\sum localVal_i$ corresponds to the ``real'' value, and $aggVal$ corresponds to the obtained aggregated value during the experiment. In addition to the error over time, we collected other metrics to measure the performance of our solutions, such as the consumption of networking and computing resources. All tests were conducted with network sizes of 750 logical nodes, and for each experiment, we varied the number of metrics emitted by the dummy applications. Finally, for each of these experiment combinations, we conducted tests with failure rates of 0 and 50\% of the nodes in the system, excluding the configurated tree roots.

The designed features were compared against Prometheus configured in two, distinct, tree-shaped setups: the first setup, which we named \textbf{centralized Prometheus}, corresponds to the most typical configuration of a Prometheus server, where a single server collects and aggregates the metrics correspondent to all the nodes in the system. The second experimental setup, named \textbf{Prometheus tree}, corresponds to a more sophisticated setup where instead of having a single aggregating node, there is an intermediate layer of nodes aggregating the metric values of the leaf nodes. This intermediate layer, in turn, is aggregated by the root node (effectively splitting the load among the aggregator nodes). The root node, in this configuration, makes use of federation to scrape the partially aggregated value from the Prometheus servers in the intermediate layer.

In the following results, there are two parameters displayed for the Prometheus results, the first, denoted by $i$, corresponds to the size of the intermediate layer, the second parameter, denoted by $o$, corresponds to the number of servers to aggregate for each node in the intermediate layer (essentially, $o$ corresponds to maxium number of leaf nodes in the promehtue). In addition, for both of the centralized and tree configurations, we also test setup a variation where every node in the system is an aggregator node (with the name \textbf{aggregator leaves}), which aggregates the metrics provided by their local dummy application, and only export the aggregated value. It is important to mention that only the first two setups (centralized and tree) are, to our knowledge, the most representative of common Prometheus configurations, however, we include the aggregator leaves scenarios to study the impact in terms of network cost of performing in-transit aggregation by every node which emits metrics when compared to performing the aggregation process of metrics extracted from multiple nodes on a single node. An illustration of these setups can be found in figure \ref{fig:sec:eval_prom_setups}.
In the following results, there are two parameters displayed for the Prometheus results, the first, denoted by $i$, corresponds to the size of the intermediate layer, the second parameter, denoted by $o$, corresponds to the number of servers to aggregate for each node in the intermediate layer (essentially, $o$ corresponds to maxium number of leaf nodes per  Prometheus server in the intermediate layer).

In addition, for both of the centralized and tree configurations, we also test setup a variation where every node in the system is an aggregator node (with the name \textbf{aggregator leaves}). In this setup, every node aggregates the metrics provided by their local dummy application, and only exports the aggregated value to the Prometheus server. It is important to mention that only the first two setups (centralized and tree) are, to our beliefs, the most representative of common Prometheus configurations.  However, we include the aggregator leaves scenarios to study the impact in terms of network cost of performing in-transit aggregation by every node which emits metrics when compared to performing the aggregation process of metrics extracted from multiple in a centralized manner. An illustration of these setups can be found in figure \ref{fig:sec:eval_prom_setups}.

\begin{figure}
    \centering
    \includegraphics[width=\linewidth]{Chapters/evaluation/figures/aggregation/Prometheus setups.pdf}
    \caption{Exemplification (at smaller scale) the of tested prometheus setups}
    \label{fig:sec:eval_prom_setups}
\end{figure}

\subsection{Tree aggregation}

For the \textbf{tree aggregation} evaluation, we configured DeMMon with a single tree aggregation function, which triggers the algorithm defined in Section \ref{sec:mon_protocol} that, in sum, collects an aggregated value of the metrics of its descendants in the DeMMon tree. This feature was designed for decentralized resource management applications that follow the DeMMon hierarchical structure to perform decentralized resource management decisions. For example, a certain application that wishes to maintain a certain ratio of two service replicas (because one depends on the other), it can do so by having each node monitor its descendants and perform resource management actions (possibly coordinated with other nodes) to replenish or decommission a service replica such that the desired ratio is maintained.

To test this feature, we set up both Prometheus and DeMMon collecting an aggregated value of the whole system in a single node, providing an aggregated view of the system (our case, we used the sum to produce the aggregated value), and collected the error of the obtained aggregated value against the correct value over time. In addition to the error, we also collected the total network cost over the duration of the experiments. 

\begin{figure}
    \centering
    \includegraphics[width=\linewidth]{Chapters/evaluation/figures/aggregation/Error_over_time_global_tree.jpg}
    \caption{Error over time obtained in tree aggregation (centralized scenario)}
    \label{fig:sec:mon_eval_tree_centralized}
\end{figure}

We begin this comparison by discussing the obtained results regarding the centralized version of Prometheus against DeMMon (fig. \ref{fig:sec:mon_eval_tree_centralized}), where we may observe that both DeMMon and Prometheus reach the 0\% error values across all conducted tests, which means the systems are working correctly. Furthermore, we may observe that, in the regular Prometheus setup (non-aggregator leaves), as the number of metrics increases, Prometheus cannot obtain the metrics to calculate the aggregated value, we believe this occurs because the root node exceeds its allocated bandwidth. This contrasts with the Prometheus ``aggregator leaves'' results, which obtain 0\% error value across all conducted experiments, this happens because every node is aggregating their emitted metrics and only propagating an aggregated value, which does not saturate the system bandwidth. It is important to notice that as the number of series increases, the DeMMon error tends to fluctuate between 0 and low error values, we believe this occurs due to the DeMMon nodes saturating their CPU when parsing the metrics. Although we realize this may be a limitation of the developed system, we argue this limitation is an engineering problem that may be easily addressed by employing a more efficient metric transmission and parsing protocol (similar to Prometheus' or InfluxDBs' \cite{influxdb_data_elements}).

\begin{figure}
    \centering
    \includegraphics[width=\linewidth]{Chapters/evaluation/figures/aggregation/Error_over_time_global_tree.jpg}
    \caption{Error over time obtained in tree aggregation (tree scenario)}
    \label{fig:sec:mon_eval_tree_tree}
\end{figure}

As the centralized Prometheus setup saturates at higher metric counts, we now compare the performance of our solution against a more scalable Prometheus setup, the \textbf{Prometheus tree} setup. The results of this comparison may be observed in figure \ref{fig:sec:mon_eval_tree_tree}, which contains the error over time obtained for the experiments with 0 and 50\% failure rates. As we can observe, Prometheus (in the non-aggregator leaves scenario) now splits the load of aggregating the metrics throughout multiple nodes and consequently can obtain the correct aggregated value. This configuration, however, is plagued by multiple unrecoverable points of failure, this is shown in the scenarios with 50\% failures (bottom half of the graph), where Prometheus setups do not recover from the induced failures, as servers require manual intervention to change their configuration.

\begin{figure}
    \centering
    \includegraphics[width=\linewidth]{Chapters/evaluation/figures/aggregation/network_cost_tree.jpg}
    \caption{Average network cost incurred during tree aggregation experiments}
    \label{fig:sec:mon_eval_tree_net_cost}
\end{figure}

Finally, in figure \ref{fig:sec:mon_eval_tree_net_cost}, we may observe the average network cost for the previously shown experiments. In this graph, we can observe that both the ``aggregator leaves'' incur a costant cost number, which is also the lowest obtained result when compared with other setups. Which is expected because these setups perform aggregation of their metrics locally before emitting them towards the root node. In the case of DeMMon, the incurred networking cost is also constant, as DeMMon also performs local aggregation before emitting the results. In the case of what we believe to be the most representative Prometheus setups, the network costs tend to increase linearly with the number of metrics, which is less desirable when compared to a constant networking cost.

\subsection{Global aggregation}

Global aggregation in DeMMon, (as further explained in Section \ref{sec:mon_protocol:global_agg}), is a feature where each node in the system calculates the result of an aggregation in a decentralized manner. To test the applicability of this feature, we also employed Prometheus as a baseline comparison, also configured with the setups described in the beginning of this section. However, for comparativeness, we configured each Prometheus server to node periodically query the root node for obtaining the aggregated value (effectively providing the same results as DeMMon).

\begin{figure}
    \centering
    \includegraphics[width=\linewidth]{Chapters/evaluation/figures/aggregation/Error_over_time_global_0_failures_centralized.jpg}
    \caption{Error over time obtained in global aggregation (centralized scenario)}
    \label{fig:sec:mon_eval_global_centralized}
\end{figure}

\begin{figure}
    \centering
    \includegraphics[width=\linewidth]{Chapters/evaluation/figures/aggregation/Error_over_time_global_tree.jpg}
    \caption{Error over time obtained in global aggregation (tree scenario)}
    \label{fig:sec:mon_eval_global_tree}
\end{figure}

The experiments conducted for this feature are similar, in terms of duration and failure rate, to the ones conducted for tree aggregation. Their results can be observed in figures \ref{fig:sec:mon_eval_global_centralized} and \ref{fig:sec:mon_eval_global_tree}. In these, we observe a similar pattern to the one observed by the tree aggregation results, namely: the centralized Prometheus configurations cannot scale when the number of metrics emitted per node increases, and the tree Prometheus configurations, while they resolve the scaling problem, are subjected to multiple points of failure that, in the case of a failure, require manual configuration to recover. 

Finally, in these results, we may observe that DeMMon can correctly obtain the aggregated value even when the number of metrics increases and in the presence of failures, making it a more versatile option for scenarios where failures are common, such as those found at the edge of the network.

\begin{figure}
    \centering
    \includegraphics[width=\linewidth]{Chapters/evaluation/figures/aggregation/network_cost_global.jpg}
    \caption{Average network cost incurred during global aggregation}
    \label{fig:sec:mon_eval_global_net_cost}
\end{figure}

The final obtained result, showing the average network cost incurred during these experiments, can be observed in figure \ref{fig:sec:mon_eval_global_net_cost}, where similarly to tree aggregation, both DeMMon and the two Prometheus configurations with ``aggregator leaves'' obtain constant network costs during the experiments (because these setups perform local aggregation before emitting their metrics). Furthermore, these Prometheus setups incurs in less networking costs given they does not have to maintain the overlay network, unlike DeMMon. 

The remaining configurations (Prometheus tree and Prometheus centralized), as they do not perform in-transit aggregation of the metrics, incur networking costs that scale with the number of metrics emitted by the nodes. Consequently, from the standpoint of scalability, this means its scalability will be limited by the number of series emitted per node, that  contrary to DeMMon or configurations with ``aggregator leaves'' ( which as previously mentioned, are not representative of most Prometheus setups), which obtain linear costs with the number of emitted metrics.

% \subsection{Neighborhood aggregation} 

% \todo{neighborhood aggregation}

\section{Summary}

In this chapter, we studied, through experimentation, the applicability of the devised decentralized aggregation and information dissemination framework, named DeMMon. We began by providing the system model in which we tested our solution (section \ref{sec:exp_setting_conf}), which aims to emulate a realistic cloud-edge scenario composed of nodes with heterogenous networking capacity and distributed among multiple places of the globe. Following, in section \ref{\label{sec:overlay_proto_eval}, we provide the obtained results from the experimental evaluation of the developed membership protocol. In this section, we tested the applicability of this protocol in its capacity to build a latency-optimized network by comparing it with implementations of state-of-the-art baselines in realistic scenarios with multiple node counts and failures. We showed that the implemented protocol is fault-tolerant (within its system model), and although its latency total is not the lowest, if only accounting for its most heavily used connections (the parent connections in the tree), then its total cost is the lowest. While still in this section, we evaluated the information dissemination capabilities of the devised protocol against the same baseline protocols, paired with two dissemination protocols: a simple flood protocol and PlumTree. Overall, we showed that our protocol is a competitive alternative in this regard, as in our tests, it performed better than PlumTree when paired with any baseline membership protocols in all conditions and performed better in lower node counts across all conducted scenarios.

Finally, we concluded the chapter with section \ref{sec:agg_proto_eval}, where we validated the implemented monitoring primitives and compared the obtained results against multiple configurations of a popular baseline solution: Prometheus. We showed that the devised decentralized monitoring solution obtains results comparable to those obtained by Prometheus, and that it provides a higher degree of fault-tolerance, as it does not require manual configuration to recover from failures. 
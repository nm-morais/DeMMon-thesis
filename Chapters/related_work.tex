\chapter{Related Work}
\label{cha:related_work}

The goal of this chapter is to present the related work studied that is associated with our objectives. We begin by identifying the four high-level requirements of a resource sharing platform, as denoted in figure \ref{fig:proposed_architecture}:

\begin{enumerate}

    \item \textit{Topology Management} consists in the study of how to organize multiple devices in a logical network such that they can cooperatively solve tasks. Efficiently managing the topology is an essential building block for achieving efficient operation of the remaining components.

    \item \textit{Resource Location and Discovery} focuses on how to efficiently index and locate resources in the aforementioned logical network. For example, in the context of resource sharing, resource discovery is paramount towards locating nearby devices which have enough (free) computing and networking capabilities to perform a certain task, or host a certain application component or service.

    \item \textit{Resource Monitoring} studies which metrics to track per device, and how to efficiently compress those metrics through aggregation to reduce the size of the collected data, as well as how to propagate that data towards the components that need it to operate.

    \item \textit{Resource Management} addresses how to efficiently manage system resources and schedule jobs across existing resources such that: (1) the system remains load-balanced; (2) operations can operate efficiently; (3) jobs have data locality; and (4) resources are not wasted. While the work conducted in this thesis is tailored toward supporting this goal, this thesis does not aim at devising a complete scheduling solution, as that is a complete research line on its own. However, for completeness, we also discuss this aspect here.

\end{enumerate}

\begin{figure}
    \centering
    \includegraphics[width=0.55\linewidth]{Figures/proposed_architecture.pdf}
    \caption{High-level architecture for a resource sharing platform}
    \label{fig:proposed_architecture}
\end{figure}

Considering the identified high-level components of such a system, in the following sections we cover the taxonomy of devices which compose the edge environment, and discuss how they can be employed towards the design of the proposed solution (Section \ref{sec:edge_computing}). Next, we study execution environments for applications, namely virtual machines and containers, discuss their performance impact as well as their strengths and limitations towards supporting edge-enabled applications (Section \ref{sec:runtime_environments}).

Following, we study how to federate devices in an efficient abstraction layer  that establishes an efficient topology (Section \ref{sec:topology_management}), and address how peers can efficiently index and search for the resources they need (e.g. services, peers, computing power, among others) in the aforementioned abstraction layer, which in turn enables the delegation of particular application components (Section \ref{sec:res_location}). This is important given the fact that edge devices are typically resource constrained, and a computing task which would otherwise require a single cloud device, may require multiple edge devices to be accomplished in an efficient way.

Next, we cover tools to collect metrics from the aforementioned execution environments that are relevant towards performing efficient resource allocations. We analyze how to aggregate those metrics in a decentralized manner, and discuss relevant resource monitoring systems in the literature, for each, we address its limitations and advantages for the edge environment (Section \ref{sec:res_monitoring}). Lastly, we cover the taxonomy of resource management solutions, and present popular systems in the literature that share aspects with the solution we aim at developing (Section \ref{sec:res_management}).

\section{Edge Environment} \label{sec:edge_computing} 
In this chapter we provide context about related work towards decentralizing computation from the Cloud. First, we define how we characterize Edge Computing and discuss how it is related to edge-related paradigms. Following, we study the taxonomy of the environment and focus on which computations each device can perform.

\subsection{Edge Computing}

As previously mentioned, edge computing calls for the processing of data in the edge of the network, specifically servicing IoT devices and performing computations on behalf of cloud services \cite{7488250}.

As previously mentioned, edge computing has the potential of enabling novel edge-enabled applications along with providing many improvements to existing systems:

\begin{itemize}

    \item \textit{Reducing the amount of traffic}, if the workload of a client can be computed in the cloud, then the latency can be greatly improved vs when doing the work in the cloud. For example, in the case of a smart home which computes the average home temperature in all divisions, almost all the data collected by the sensors can be aggregated (averaged) in a home gateway, effectively summarizing the data from a set of values to one single value. Reducing the data size improves the transmission reliability and saves bandwidth for other edge devices.
    
    \item \textit{Latency} is one of the most important aspects towards the performance of applications / services, especially time-critical applications (e.g. traffic monitoring) have special latency needs. Latency consists in the time it takes for a packet to travel the network from the origin endpoint to the target endpoint. Edge computing solves this by offloading the computation towards the nearest edge device, for example a cellphone offloading facial recognition to the nearest 5G cell tower.

    \item Facilitating new approaches of \textit{load-balancing}, given that there is a massive increase of devices in the edge of the network, if all devices cooperate towards a common goal, they can achieve tasks similar to those of more complex systems (e.g. sensor networks performing sql queries \cite{Madden2002}). For example, edge computing solves service load-balancing by offloading its computation towards the nearest availablse neighbor which has more capacity, or by offloading a portion of its tasks.
    
    \item \textit{Minimizing energy consumption}, edge computing has the potential of performing some of the aforementioned tasks, however, instead of optimizing towards latency or traffic locality, devices take into consideration their energy percentage. For example, a device which has very low latency but also low battery in a time-critical service can chose to offload any unnecessary computations towards neighbors, as a cost-saving measure. Additionally, given that devices are closer to their targets, communications are faster and consequently consume less energy.
    
\end{itemize}


Many approaches have already leveraged on some form of Edge computing in the past. \textbf{Cloudlets} \cite{10.1145/2307849.2307858} are an extension of the cloud computing paradigm beyond the DC, which consists of deploying resource rich computers near the vicinity of users that provide cloud functionality.

A limitation of cloudlets is that because they are specialized computers, they cannot not guarantee low-latency ubiquitous service provision, and consequently cannot satisfy QoS of large hotspots of users. Cloudlets have become a trending subject, and have been employed towards resource management, Big Data analytics, security, among others.

\textbf{Content Distribution networks} \cite{peng2004cdn} (CDNs) emerged to address the overwhelming utilization of network bandwidth and server capacity that arose with bandwidth-intensive content (e.g. streaming HD video). In short, CDNs consist of specialized high bandwidth servers strategically located at the edge of the network, these servers replicate content from a certain origin and serve it at reduced latencies, effectively decentralizing the content delivery. 

Additionally, many paradigms have emerged which propose to solve similar problems to the Edge Computing paradigm. \textbf{Fog Computing} \cite{bonomi2012fog} proposes to provide compute, storage and networking services between end devices and traditional cloud computing data centers, typically, but not exclusively located at the edge of the network. Fog computing is interchangeable with our vision of all devices acting as an "edge"\ contributing towards computing tasks.

\textbf{Osmotic Computing} \cite{villari2016osmotic} envisions the automatic deployment and management of inter-connected microservices on both edge and cloud infrastructures. Osmotic computing envisions edge devices employing an orchestration technique similar to the process of "osmosis". Translated, this consists in dynamically detecting and resolving resource contention via coordinated microservice deployments, furthermore, this paradigm is focused towards ensuring and maintaining quality of service.

\textbf{Multi-access edge computing} \cite{mobile_edge_cloud} (MEC) formerly known as mobile-edge cloud computing, is a network architecture that proposes to provide fast-interactive responses for mobile applications. It solves this by employing the network edge (e.g. base stations and access points) to provide compute resources for latency-critical mobile applications. MEC is a subset of our edge computing vision, although with a higher focus on communications technology and how to offload the computation from mobile to the cloud and not vice-versa. 

\subsubsection{Edge Environment Taxonomy} \label{subsec:edge_taxonomy}

Similar to \cite{Leitao2018}, we classify edge device according to 3 main attributes: \textbf{capacity} refers to computational, storage and connectivity capabilities of the device,  \textbf{availability} consists in the probability of a device being reachable, and finally, \textbf{domain} characterizes the way in which a device may be employed towards applications. In short, if the device can support the whole \textit{applicational domain} or only the activities of a single user (\textit{user domain}) .

\begin{table}[!htb]
    \caption{Taxonomy of the edge environment}
    \begin{minipage}{.45\linewidth}
        \centering
        \resizebox{\columnwidth}{!}{%
        \begin{tabular}{|l|l|l|l|}
            \hline
            Level & Category & Availability & Capacity \\ \hline
            L0 & Cloud Data Centers & High & High \\ \hline
            L1 & ISP, Edge \& Private DCs & High & High \\ \hline
            L2 & 5G Towers & High & Medium \\ \hline
            L3 & Networking devices & High & Low \\ \hline
        \end{tabular}}
    \end{minipage} %
    \begin{minipage}{.45\linewidth}
        \centering
        \resizebox{\columnwidth}{!}{%
            \begin{tabular}{|l|l|l|l|}
                \hline
                Level & Category & Availability & Capacity \\ \hline
                L4 & Priv. Servers \& Desktops & Medium & Medium \\ \hline
                L5 &Laptops & Low & Medium \\ \hline
                L6 &Mobile devices & Low & Low \\ \hline
                L7 &Actuators \& Sensors & Varied & Low \\ \hline   
        \end{tabular}}
    \end{minipage} 
    \label{tab:taxonomy_edge}
\end{table}

Table \ref{tab:taxonomy_edge} shows the categories of edge devices, we assign levels to categories as a function of the distance from the cloud infrastructure. Coincidently, the levels are correlated to the number of devices and their computational power, where higher levels tend to have more devices that are closer to the origin of the data and have lower computational power.

\textbf{Levels 0 and 1} \textit{cloud and edge DCs} offer pools of computational and storage resources, that can dynamically scale to support the operation of edge-enabled applications. Both of these options have high availability and large amounts of storage and computational power, as such, there is no limitations on which type of computations these devices can perform.

\textbf{Level 2} is composed of \textit{5G cell towers}, which serve as access points for mobile devices, \textbf{Level 3} also consists of \textit{networking devices}, although with lower capacity than those in level 2. Devices in both levels have high availability, and they can contribute to the applicational domain, however in a limited fashion (e.g. coordinate resource management, host a microservice, or just act as a gateway for mobile devices). Devices in this level can easily improve the management of the network, for example, by manipulating data flows.

\textbf{Level 4} Consists of \textit{private servers and desktops}, it is the first level where devices belong to the user domain. Devices in this level have medium capacity and availability, and can perform a varied amount of tasks on behalf of devices in higher levels (e.g. compute on behalf of smartphones, act as logical gateways or just cache data). 

\textbf{Level 5} consists of \textit{laptops}, which are also on the user domain, and can perform a role similar to devices in level 4, although with lower availability and capacity. The main differentiating factor with devices in level 4 is that laptops are battery-powered, which means that energy consumption must also be taken into account whenever monitoring and computing on these devices. 

\textbf{Level 6} consists of \textit{tablets and mobile devices}, devices in this level act as producers and consumers of data and belong to the user domain. Because because of their low capacity, availability and short battery life, mobile devices are limited in how they can perform computations and contribute towards edge applications. Aside from caching user data, common usages are filtering or aggregation of data generated from devices in level 7. 

Finally, \textbf{level 7} consists of \textit{actuators, sensors and things}, these devices are the most limited in their capacity, and varied availability. \textit{Things} act both as data producers and consumers towards edge-enabled applications. They enable limited forms of computation in the form of aggregation and filtering.

\subsection{Discussion}

Intuitively, the lower the level the harder it is to employ devices towards applications. Devices in levels 6 and 7 are especially restricted due to having lower availability and computational power, however, these can still be used in specific scenarios (e.g. an application with very low computational overhead but real-time latency requirements). 

Devices in levels 0-5 are potential candidates towards building the resource monitoring system we intend to create. Given the low availability of devices in higher levels, they are not very suitable, as they could not contribute much towards the system and would incur churn. This effect can be circumvented by employing devices in other levels as gateways for mobile devices and \textit{things}.

\section{Execution Environments} \label{sec:runtime_environments} 
% After studying the taxonomy of the edge environment, it is paramount to study how these devices can execute computations (e.g. hosting application components, monitoring tasks, among others) in a controlled environment. A major requirement of these environments is the ability to simultaneously execute multiple computations, and that these interfere as little as possible with each other, as well as with the core behavior of the system.

% A popular approach towards solving these challenges is to perform computations in loosely coupled independent components running some form of virtualization software, as it enables the co-deployment of components within the same physical machine. The main benefits of employing virtualization include hardware independence, isolation, secure user environments, and increased scalability. 

% The two most common types of virtualization used nowadays are containers and virtual machines (VMs), in this Section present a brief description of both technologies, and study their advantages / limitations towards supporting edge-enabled applications.

% \subsection{Virtual Machines}

% A VM provides a complete environment in which an operating system and many processes, possibly belonging to multiple users, can coexist. By using VMs, a single-host hardware platform can support multiple, isolated guest operating system environments simultaneously \cite{1430629}. 

% Virtual machines rely on a type of software called a \textit{hypervisor}, the role of the hypervisor is to abstract hardware to support the concurrent execution of full-fledged operating systems (e.g. Linux or Windows). Virtualizing the hardware layer ensures great isolation between virtual machines, meaning that a VM cannot directly interact with the host or the other VMs, which is highly desirable for both the virtualized applications and the host. 

% However, virtualizing the hardware and the device drivers incurs non-negligible overhead, and the large image sizes of operating systems required by virtual machines makes live migrations harder to accomplish, which we believe to be crucial in edge environments.

% \subsection{Containers}

% Containers (e.g., Docker \cite{docker}, Linux Containers \cite{lxc}, among others) can be considered as a lightweight alternative to hypervisor-based virtualization. When using containers, applications share an OS (and maybe binaries and libraries), and implement isolation of processes at the operating system level. As a result, these deployments are significantly smaller in size than hypervisor deployments, for comparison, a physical machine may store hundreds of containers versus a few tens of VMs \cite{7036275}.  

% In terms of performance, container-based virtualization can be compared to an OS running on bare-metal in terms of memory, CPU, and disk usage \cite{preeth2015evaluation}, and contrary to VMS, restarting a container doesn't require rebooting the OS \cite{7036275}, meaning that a small-sized computation task may be accomplished much faster. 

% Consequently, given their lightweight nature, it is possible to deploy container-based applications (e.g. microservices), which can perform fast migration across nodes in the edge environment (e.g. in order to improve quality of service (QoS) of applications). This flexibility towards the migration process is an effective tool to deal with many challenges such as load balancing, scaling, resource reallocation and fault tolerance. 

% \subsection{Discussion}

% Although VMs are widely present in the cloud infrastructure, they incur significant start up time (due to having to start-up an entire OS) and image sizes are larger when compared to containers (due to requiring a full OS image), which hinders the ability to perform quick migrations across different devices. The accumulation of these factors make VMs unsuited for devices with low capacity and availability, which are abundant in edge environments, consequently, we believe containers are the most appropriate solution when it comes to performing resource sharing in edge scenarios. 

\section{Topology Management} \label{sec:topology_management} 
% -------------------
% Topology Management
% -------------------



Provided with an overview of the taxonomy of the devices materializing edge environments, we now study the related work towards federating all these devices (that we also refer to as peers following the peer-to-peer (P2P) literature) in an abstraction layer (an overlay network) that allows intercommunication and efficient resource discovery \cite{leitaoPHDthesis}. This section provides context regarding the taxonomy of overlay networks, followed by a discussion of popular overlay network protocols and what we believe to be their strengths and limitations.

In a P2P system, peers contribute to the system with a portion of their resources to accomplish tasks otherwise unfeasible for an individual peer. Typically, this is achieved in a decentralized way, which means peers must establish neighbouring connections among themselves to enable information exchange which, in turn, enables progress towards the system goals. 

Participants in a P2P system may know all other peers in the system, which is typically referred to as \textbf{full membership} knowledge, this is a popular approach in Cloud systems. However, as the system scales to larger numbers of peers, concurrently entering and leaving the system (a phenomenon called churn \cite{stutzbach2006understanding}), this information becomes costly to maintain up-to-date. 

In order to circumvent the aforementioned problems, a common alternative is to have peers only maintain a view of a subset of all peers in the system, which is called \textbf{partial membership}. This information is maintained by some membership algorithm which restricts neighbouring relations among peers. Partial membership solutions are attractive because they offer similar functionality to full membership systems, while achieving more scalability and resiliency to churn. The closure of these neighbouring relations is what materializes an \textbf{overlay network}.

\subsection{Taxonomy of Overlay Networks}

Overlay networks are logical networks that operate at the applicational level. These rely on an existing network (commonly referred to as the \textit{underlay}) to establish neighbouring relations, where each participant typically only communicates directly with its overlay neighbours \cite{leitaoPHDthesis}. Overlays are commonly designed towards specific applicational needs, as such, their neighbouring relations may or may not follow some  established logic. As illustrated in Figure \ref{fig:overlay_networks}, there are two main categories of overlays: \textbf{structured} and \textbf{unstructured}:

\subsubsection*{Unstructured Overlays}

\begin{figure}
    \centering
    \includegraphics[width=0.60\linewidth]{Chapters/Figures/overlay_networks.pdf}
    \caption{Examples Overlay Networks}
    \label{fig:overlay_networks}
\end{figure}

Unstructured overlays usually impose little to no rules in neighbouring relations, peers may pick random peers to be their neighbours, or alternatively employ strategies to rank neighbours and selectively pick the best given a particular criteria that is typically entwined with the needs of applications. A key factor of unstructured overlays is their low maintenance cost, given that nodes can easily create neighbouring relations, which eases the process of replacing failed ones. Consequently, this is the type of overlay which offers better resilience to churn.

In figure \ref{fig:overlay_networks}, we illustrate three examples of unstructured overlay networks: (A) is a representation of an overlay network where the connections are unidirectional (e.g. Cyclon \cite{jelasity2007gossip}), in this type of overlay peers have no control over the status of incoming connections, consequently, a peer may become isolated from the network without realizing it, which is undesirable. 

Overlay (B) is similar to (A), however, neighbouring connections are bidirectional. This means that a peer with a given number of outgoing connections must also have the correspondent number of incoming connections, diminishing the risk of the peer becoming disconnected from the overlay (this is the approach taken by HyParView \cite{Hyparview} to achieve high reliability and fault-tolerance).

Lastly, (C) is a representation of an unstructured overlay where peers establish groups among themselves (such as Overnesia \cite{leitao2014overnesia}). Grouping multiple devices into a group can provide benefits such as (1) failures can be quickly identified and resolved by other members of the group; (2) nodes can replicate data within the group, leading to increased availability of that data; (3) groups can abstract groups of resources and internally manage their usage by, for example, offloading computational tasks within the group. 

\subsubsection*{Structured Overlays}

Structured overlays enforce stronger rules towards neighbour selection (generally based on identifiers of peers). As a result, the overlay generally converges to a certain topology known a \textit{priori} (e.g., a ring, tree, hypercube, among others). 

Figure \ref{fig:overlay_networks} also illustrates three kinds of structured overlay networks: (D) corresponds to a tree, which are widely used to perform broadcasts (e.g., PlumTree \cite{leitao2007epidemic}) because of the smaller message complexity required to deliver a message to all nodes, or to monitor the system state (if nodes in lower levels of the tree periodically send monitoring information to upper levels in the tree, in turn, the root of the node has a global view of the collected monitoring information (e.g., Astrolabe \cite{Renesse2003})). However, trees are very fragile in the presence of faults \cite{leitao2007epidemic}.

Overlay (E) corresponds to a typical overlay typically expected to support Distributed Hash Tables (DHTs). These are extremely popular due to their effective applicational-level routing capabilities. In a DHT, peers employ a global coordination mechanism which restricts their neighbouring relations such that can find any peer \textit{responsible} for any given key in a  limited number of steps (typically the logarithm of the system size). In this example (figure E), the topology consists of a ring (which is the strategy employed by Chord \cite{stoica2003chord}). It is important to mention however, that not all distributed hash tables rely on rings to perform effective routing. For example, in Kademlia \cite{maymounkov2002kademlia}, nodes organize as leaves across a binary tree.

Finally, the overlay denoted in (C) is similar to overlay (E), however each position of the DHT is made up of a virtual node composed of multiple physical nodes (which is the strategy employed by Rollerchain \cite{rollerchain}). Because of this, routing procedures are still limited acording to the logarithm of system size, and also have the potential to be load-balanced. Furthermore, as the failure of a physical node does necessarily mean the failure of a virtual node, churn effects are mitigated. 
 
\subsection{Overlay Network Metrics}

If we look at an overlay network where connections between nodes represent edges and nodes represent vertices in a graph, we obtain a graph from which we may extract direct metrics to estimate overlay performance \cite{leitaoPHDthesis}, we now enumerate some which we believe to be the most relevant to our goal:

\begin{enumerate}
    
    \item \textbf{Connectivity}. This property is usually measured as a percentage, corresponding to the largest portion of the system that is connected, intuitively, a connected graph is one where there is at least one path from each node to all other nodes in the system.
    
    \item \textbf{Degree Distribution}. The degree of a node consists in the number of arcs that are connected to it. In a directed graph, there is a distinction between \textbf{in-degree} and \textbf{out-degree} of a node, nodes with a high in-degree value have higher reachability, while nodes with 0 in-degree cannot be reached. The out-degree of a node represents a measure of the contribution of that node towards the maintenance of the overlay topology.
    
    \item \textbf{Average Shortest Path}. A path is composed by the edges of the graph that a message would have to cross to get from one node to other. The average shortest path consists in the average of all shorter paths between every pair of peers, to promote efficient communication patterns, is desirable that this value is as low as possible.
    
    \item \textbf{Clustering Coefficient}. The clustering coefficient provides a measure of the density of neighbouring relations across the neighbours of links between a given node. It consists in the number of a node's neighbours divided by the maximum number of links that could exist between those neighbours. A high value of clustering coefficient means that there is a higher amount of redundant communication among nodes.
    
    \item \textbf{Overlay Cost}. If we assume that a link in the overlay has a \textit{cost}, (e.g. derived from latency), then the overlay cost is the sum of all the costs of the links that form the overlay. 
    
\end{enumerate}

\subsection{Examples of Overlay Networks}

\textbf{T-MAN} \cite{jelasity2005t} is protocol to manage the topology of overlay networks, it is based on a gossiping scheme, and proposes to build a wide range of structured overlay networks (e.g., ring, mesh, tree, among others). To achieve this, T-MAN expects a cost function as an input to the protocol, then employed as a ranking method, applied iteratively by every node to compare the preference among possible neighbours. 

Nodes periodically exchange their neighbouring sets with peers in the system and keep the nodes which rank higher according to the ranking method. A limitation of T-Man is that it does not ensure the stability of the in-degree of nodes during the optimization of the overlay, and consequently, the overlay may not remain connected. 

\textbf{Management Overlay Network} \cite{liang2005mon} (MON) is an overlay network system aimed at facilitating the management of large distributed applications. This protocol builds on-demand overlay structures that allow users to execute instant management commands, such as query the current status of the application or push software updates to all the nodes. MON performs these procedures in an on-demand fashion such that it achieves a low maintenance cost when no commands are running.

This solution allows the on-demand construction of two types of Overlay Networks: trees and direct acyclic graphs. These overlays, in turn, can be employed towards aggregating monitoring data related to the status of the devices. Limitations from using MON are that the resulting overlays are susceptible to topology mismatch, and do not ensure connectivity. Furthermore, since the topologies are supposed to be short-lived, MON does not provide mechanisms for dealing with faults.

\textbf{Hyparview} \cite{Hyparview} (Hybrid Partial View) gets its name from maintaining two exclusive views: the \textit{active} and \textit{passive} view, which are distinguished by their size and maintenance strategy. 

The \textit{passive view} is a larger view which consists of a random set of peers in the system,  maintained by a periodic gossip protocol, where each peer sends a message to another random peer in their active view. This message contains a subset of the neighbours of the sending node and a time-to-live (TTL). The message is then forwarded randomly throughout the system until the TTL expires, updating the views of nodes it passes. In contrast, the \textit{active view} consists of a smaller view (around log(n) in size), created during the bootstrap of the protocol. Each peer in the view has an associated TCP connection, which is then used as a bidirectional connection medium and failure detector. Whenever a node from the active view is detected as failed, it is replaced with one in the passive view.

Hyparview is often used as a \textit{peer sampling service} for other protocols which rely on the connections from the active view to collaborate (e.g. PlumTree \cite{leitao2007epidemic}). It achieves high reliability even in the face of a high percentage of node failures. However, its resulting topology is flat, which we believe to not be ideal for the taxonomy of edge environments we are considering. Furthermore, it may suffer from topology mismatch: given the random nature of neighbouring connections, the resulting neighbouring connections may be very distant in the underlying network.

\textbf{X-BOT} \cite{leitao2012x} is a protocol that constructs an unstructured overlay network where neighbouring relations are biased considering one metric. This metric is provided by an \textit{oracle}, which is a component that exports a function, which accepts a pair of peers and attributes a cost to that neighbouring connection. This cost may take into consideration factors such as latency, ISP distribution, network stretch, among others. 

The rationale X-BOT is as follows: nodes maintain active and passive views similar to Hyparview \cite{Hyparview}. Then, nodes periodically trigger optimization rounds where they attempt to bias a portion of their connections according to the functions provided by the oracle. Although this protocol potentially addresses the previous concerns about the overlay topology mismatching the underlying network, it still proposes a flat topology, which, as previously mentioned, we believe is not adequate for the edge environment taxonomy. 

\textbf{Overnesia} \cite{leitao2014overnesia} is a protocol that establishes an overlay composed of fully connected groups of nodes, where all nodes within a group share the same identifier. Nodes join the system by sending a request to a bootstrap node, triggering a random walk. The requesting node joins the group where its random walk terminates (either because it finds an underpopulated group or because the TTL expires). 

Nodes enforce intra-group membership consistency through an anti-entropy mechanism where nodes within a group periodically exchange messages containing their view of the group. When a group detects that its size has become too large, it triggers a dividing procedure that splits the group in two halves. Conversely, when the group size has fallen below a certain threshold, nodes trigger a collapsing procedure. During it, each node takes the initiative to relocate itself to another group, resulting in the graceful collapse of the group. Finally, inter-group links are acquired by propagating random walks throughout the overlay.

As previously mentioned, establishing groups of nodes enables load-balancing, efficient dissemination of queries, and fault-tolerance. Furthermore, it allows the abstraction of a set of physical resources into a single, unified, logical resource. 

However, in systems with heterogeneous composition, system scalability limitations may arise as devices may have trouble maintaining the group view up-to-date and the active connections to all group members. Finally, the overlay may suffer from topology mismatch, as two nodes within the same group may be distant within the underlay.

\textbf{Chord} \cite{stoica2003chord} is a well known structured overlay network where the protocol builds and manages a ring topology, similar to overlay (E) in Figure \ref{fig:overlay_networks}. Each node is assigned an m-bit identifier, uniformly distributed across the id space, and takes steps to fill its \textit{finger table}. The finger table contains at most \(m\) entries, each $ith$ entry of this table corresponds to the first peer that succeeds a certain peer \(n\) by \(2^{ith}\) in the ring. This means that whenever the finger table is up-to-date, and the system is stable, lookups for any data piece only take logarithmic time to finish. 

Although Chord provides a good trade-off between bandwidth and lookup latency, it has its limitations: peers do not learn routing information from incoming requests, links have no correlation to latency or traffic locality, and the overlay is highly susceptible to churn \cite{dht_performance_churn}. Finally, the ring topology is flat, which means that lower capacity nodes in the ring may become a limitation instead of an asset in the context of routing procedures.

%Chord is the building block for many other solutions: Cyclone \cite{Artigas2005} is a hierarchical DHT inspired on Chord which constructs clusters by splitting the ID space into a PREFIX and SUFIX. The PREFIX provides intra-cluster identity, whereas the SUFIX lets nodes know their residence cluster.

%Hieras \cite{1240580} uses a binning scheme according the underlay topology to group peers into smaller rings. The lower the ring, the smaller the average link latency. Routing is done in lower rings to make use of the reduced latency and if the resource is not present in those rings, then the query is forwarded to higher rings.

%Crescendo \cite{Ganesan2004} splits the ID range into domains (similar to DNS), where nodes in leaf-domains form Chord rings, then nodes merge rings by applying rules such that rings in different domains can communicate. The resulting routing table and the routing procedures in Crescendo are similar to Chord.

\textbf{Pastry} \cite{rowstron2001pastry} is another well known DHT which assigns a 128-bit node identifier (nodeId) to each peer in the system. The nodeIds are randomly generated, and consequently, are uniformly distributed in the 128-bit nodeId space. Routing procedures are forwarded to nodes whose nodeId shares a prefix that is at least one bit closer to the key, if there are no nodes available, nodes route messages towards the numerically closest nodeId. This routing procedure takes O(log N) routing steps, where N is the number of Pastry nodes in the system. 

This protocol has been widely used as a building block for Pub-Sub applications such as Scribe \cite{10.1007/3-540-45546-9_3} and file storage systems like PAST \cite{990064}. However, limitations from using Pastry arise from the use of a numeric distance function towards the end of routing procedures, which creates discontinuities at some nodeId values and complicates attempts at formal analysis of worst-case behaviour, in addition to establishing a flat topology that mismatches the edge device taxonomy.

\textbf{Tapestry} \cite{tapestry} Is a DHT with similar behaviour to Pastry~\cite{rowstron2001pastry}. In this system, however, nodeIDs are represented using base b, where b is a parameter specified during configuration. In routing procedures, messages are incrementally forwarded to the destination digit by digit (e.g. ***8 -> **98 -> *598 -> 4598). Consequently, in stable conditions, routing procedures theoretically take $\log{b}{n}$ hops to reach their destination, where b is the base of the ID space. Because nodes assume that the preceding digits all match the current node's suffix, nodes in Tapestry only need to keep a constant size of $\log{N}$ entries at each route level, consequently, nodes contain entries for a fixed-sized neighbour map of size b.  

\textbf{Kademlia} \cite{maymounkov2002kademlia} is a DHT where nodes are considered leaves distributed across a binary tree. Peers route queries and locate data pieces by employing an XOR-based distance function which is symmetric and unidirectional. Each node in Kademlia is a router where its routing tables consist of shortcuts to peers whose XOR distance is between \(2^{i}\) by \(2^{i + 1}\) in the ID space, given the use of the XOR metric, "closer"\ nodes are those that share a longer common prefix.

The main benefits that Kademlia draws from this approach are: nodes learn routing information from receiving messages, there is a single routing algorithm for the whole routing process (unlike Pastry \cite{rowstron2001pastry}) which eases formal analysis of worst-case behavior. Finally, Kademlia exploits the fact that node failures are inversely related to uptime by prioritizing nodes that are already present in the routing table.

\textbf{Kelips} \cite{gupta2003kelips} is a group-based DHT which exploits increased memory usage and constant background communication to achieve reduced lookup time and message complexity. Kelips nodes are split in $k$ affinity groups split in the intervals [0,$k-1$] of the identifier space, thus, with $n$ nodes in the system, each affinity group contains $\frac{n}{k}$ peers. Within a group, nodes store a partial set of nodes contained in the same affinity group and a small set of nodes lying in foreign affinity groups. With this architecture, Kelips achieves O(1) time and message complexity in lookups, however, it has limited scalability when compared to previous DHTs, given the increased memory consumption (O($\sqrt{n}$).

\textbf{Rollerchain} \cite{rollerchain} is a protocol which establishes a group-based DHT by leveraging on techniques from both structured and unstructured  overlays (Chord and Overnesia). In short, the Overnesia protocol materializes an unstructured overlay composed by logical groups of physical peers who share the same identifier. Then, the peer with the lowest identifier within each logical group joins a Chord overlay, obtains the adresses of other virtual peers, and distributes them among group members.

Rollerchain has the potential to enable a type of replication which has higher robustness to churn events when compared to other other replication strategies, however, there are limitations to this approach: (1) the load is unbalanced within members of each group, as only one node is in charge of populating and balancing the inter-group links; (2) similar to Chord, nodes do not learn from incoming queries, which contrasts with other DHTs such as Pastry; (3) the protocol has a higher implementation complexity and maintenance cost when compared to a regular DHT.

\subsection{Discussion}

Unstructured overlays are an attractive option for federating large amounts of devices in heavily dynamic environments. They provide a low clustering coefficient, are flexible, and maintain good connectivity even in the face of churn. However, given their unstructured nature, they are limited in certain scenarios, for example, when trying to find a specific peer or resource in the system.

Conversely, distributed hash tables enable efficient routing procedures with very low message overhead, which makes them suitable for application-level routing. However, given their strict neighbouring rules, participating nodes cannot replace neighbours easily, which hinders the fault-tolerance of these types of topologies, in addition, given the fact that devices in edge environments have varied computational power and connectivity, they may become a limitation instead of an asset in the context of routing procedures. 

%Hierarchical DHTS consisting of DHTS contained within other DHTS (e.g. a ring within a ring) offer several advantages over a flat DHT: first, lookups take less hops and messages to reach the target, second, organizing nodes in disjoint groups allows traffic locality if groups of nodes are close within the underlay, and churn events within a group stay contained within that group. 

%However, many of these systems either employ more memory to accommodate the many levels of the hierarchy, or tradeoff reliability (by shortening the number of connections) for memory and communication efficiency. 

\section{Resource Location and Discovery} \label{sec:res_location} 
% resource management in the DC
% -------------------
% Resource Management
% -------------------

Resource location systems are one of the most common applications of the P2P paradigm \cite{leitaoPHDthesis}, in a resource location system, a participant provided with a resource descriptor is able to query other peers and obtain an answer to the location (or absence) of that resource in the system within a reasonable amount of time. To do so, resource location systems employ search strategies, which depend on : (1) the structure the an overlay network (structured or unstructured). (2) on the characteristics of the resources to search (e.g. if there are many copies of it or not), and (3) on the desired results (e.g. if a single copy of a resource satisfies the query, or multiple are required). 

In the context of resource management, if a peer wishes to offload computations to other peers, it must employ an efficient search strategy to find nearby available resources (e.g., storage capacity, computing power, among others) in order to offload computations. In this section, we cover resource location and discovery, starting with the taxonomy of querying techniques for P2P systems, followed by the study of how resources can be stored or indexed and looked up throughout the topologies studied in the previous section.

\subsection{Querying techniques}

Querying techniques consist of how peers describe the resources they need, these, according \cite{leitaoPHDthesis}, may be classified as: \textbf{(1)~Exact Match queries}, these specify the resource to search by the value of a unique attribute (i.e., an identifier, commonly the hash of the value of the resource); the second querying methodology type is \textbf{(2)~keyword queries}, that employ one or more keywords (or tags) combined with logical operators to describe resources (e.g. "pop", "rock", "pop and rock" ...); next, \textbf{(3)~range queries} retrieve all resources whose value is contained within a given interval (e.g. "movies with 100 to 300 minutes of duration"); finally, \textbf{(4)~arbitrary queries} aim to find a set of nodes or resources that satisfy one or more arbitrary conditions (e.g. looking for a set of resources encoded in a certain format).

Provided with a way of describing their resource needs, peers need strategies to index and retrieve the resources in the system, there are three popular techniques: \textbf{centralized}, \textbf{distributed over an unstructured overlay}, or \textbf{distributed over a structured overlay}.

\subsection{Centralized Resource Location}

\textbf{Centralized resource location} relies on one (or a group of) centralized peers that index all existing resources. This type of architecture greatly reduces the complexity of systems, as peers only need to contact a subset of nodes to locate resources. 

It is important to notice that in a centralized architecture, while the indexation of resources is centralized, the resource access may still be distributed (e.g. a centralized server provides the addresses of peers who have the files, and files are obtained in a pure P2P fashion), a system which employs this architecture with success is BitTorrent \cite{cohen2003incentives}.

Although centralized architectures are widely used nowadays, they lack the necessary scalability to index the large number of dynamic resources we intend to manage, and have limited fault tolerance to failures, making them unsuited for edge environments. 

%However, there are many ways that a hybrid architecture can be applied to Edge computing: since the failure rate of a single data center (DC) is low, if we assume a system composed by multiple DCs, they may act as a reliable failover for whenever edge devices are partitioned of fail. 

\subsection{Resource Location on Unstructured Overlays}

When employing an unstructured overlay for resource location, the resources are scattered throughout all peers in the system, consequently, peers need to employ distributed search strategies to find the intended resources. This is accomplished through disseminating messages containing these queries throughout the overlay. The dissemination of these messages can follow multiple strategies, we now cover there two popular approaches: \textbf{flooding} and \textbf{random walks} \cite{leitaoPHDthesis}. 

\textbf{Flooding} consists of peers eagerly forwarding queries to others in the system as soon as they receive them for the first time, the objective of flooding is to contact multiple distinct peers that may have the queried resource. One approach is \textbf{complete flooding}, which consists in contacting every node in the system, this guarantees that if the resource exists, it will be found. However, complete flooding is not scalable and incurs significant message redundancy. \textbf{Flooding with limited horizon} minimizes the message overhead by attaching a TTL to messages that limits the number of times a message can be retransmitted. However, there is a trade-off for efficiency: flooding with limited horizon does not guarantee that all resources will be found. 

\textbf{Random Walks} are a dissemination strategy that attempts to minimize the communication overhead that is associated with flooding. A random walk consists of a message with a TTL that is randomly forwarded one peer at a time throughout the network. Random walks may also attempt to bias their path towards peers that are more likely to have answers to the query \cite{1022239}, this technique is commonly reffered to in the literature as a \textbf{random guided walk}. A common approach to bias random walks is to use bloom filters \cite{5751342}, which are space-efficient probabilistic data structures that allow the creation of imprecise distributed indexes for resources.

First generation of decentralized resource location systems relied on unstructured overlays (such as Gnutella \cite{gnutella_gtk}) and employed simple broadcasts with limited horizon to query other peers in the system. However, as the size of the system grew, simple flooding techniques lacked the required scalability for satisfying the rising number of queries, which triggered the emergence of new techniques to reduce the number of messages per query, called \textbf{super-peers}. 

\textbf{Super-peers} are peers which are assigned special roles in the system (often chosen in function of their capacity or stability). In the case of resource location systems, super-peers disseminate queries throughout the system. This technique is at the core of solutions such as Gia \cite{Chawathe2003}, employed towards effectively reducing the number of peers that have to disseminate queries on the second version of Gnutella \cite{gnutella_gtk}. 

\textbf{SOSP-Net} \cite{garbacki2007optimizing} (Self-Organizing Super-Peer Network) proposes a resource location system composed by regular peers and super-peers that effectively employs feedback concerning previous queries to improve the overlay network. Weak peers maintain links to super-peers which are biased based on the success of previous queries, and super-peers bias the routing of queries by taking into account the semantic content of each query. 

However, even with super-peers, one problem that still remains in these systems is finding very rare resources, which requires flooding the entire overlay. To circumvent this, the third generation of resource location systems rely on Distributed Hash Tables to ensure that even rare resources in the system can be found within a limited number of communication steps.

\subsection{Resource Location on Distributed Hash Tables}

Resource location on structured overlays is often done by relying on the applicational routing capabilities of distributed Hash Tables (DHTs). In a DHT, peers use hash functions to generate node identifiers (IDS) often uniformly distributed over the ID space. Then, by employing the same hash function to generate resource IDs, and assigning a portion of the ID space to each node, peers are able to map resources to the responsible peers in a bounded number of steps, which makes them very suitable for (\textbf{exact match queries}) \cite{leitaoPHDthesis}. 

One particular type of DHT that is commonly employed in small-sized resource location systems is the One-Hop Distributed Hash Table (DHT), nodes in a one-hop DHT have full membership of the system, and can locally map resources to known peers, thus performing lookups in O(1) time and message complexity. Facebook's Cassandra \cite{lakshman2010cassandra} and Amazon's Dynamo \cite{decandia2007dynamo} are widely used implementations of one-hop DHTs. 

There are two popular techniques for storing resources in a DHT, the first approach is to store the resources locally and publish the location of the resource in the DHT. This way, the node responsible for the resource's key only stores the locations of other nodes in the system and the resource may be replicated among distinct nodes composing the system. The second technique consists of transferring the resource to the responsible node in the DHT, although fewer nodes must keep the same value. It is, however, important to mention that this way the resources are not replicated, provided that with consistent hashing, all nodes with the same resource will publish the resource in the same location of the DHT.

\subsection{Discussion}

As mentioned previously, we believe centralized resource location systems are unsuited for edge environments, given that as previously mentioned in section \ref{sec:context}, for our goal, centralizing the computation (for example in data-centers) will eventually lead to a bottleneck for the system scalability. Furthermore, these types of systems are plagued with a single point of failure, making them unsuitable for volatile environments. \todo {improve}

Unstructured resource location systems are attractive for systems that perform queries in search for resources with multiple copies or for range queries, however, this approach is inefficient when performing exact match queries, as a finding the exact resource in an unstructured resource location system requires flooding the entire system with messages.

Conversely, distributed hash tables are specially tailored towards exact match queries, but are less robust to churn and are subject to low-capacity nodes being a bottleneck in routing procedures. 

% In the context of the proposed solution, given that the resources we intend to manage are present in all nodes (e.g., computing power, memory, among others), we believe unstructured resource location is better suited for our needs. For example, if an edge device wishes to find nearby computing resources to offload a certain task, it may employ a random walk. On the other hand, if a peer wishes to find a larger set of computing resources to deploy multiple application components, it may employ flooding techniques. 

%Hybrid approaches between  simultaneously ensure load balancing properties and address  problems related to churn and 

%\subsection{Hybrid approaches}

%\textbf{Curiata} \& \textbf{Build One Get One Free}

% TODO falar de surrogate routing
%\textcolor{red}{surrogate routing}

%\paragraph{\textbf{Viceroy} }

%\paragraph{\textbf{Koala} }

% TODO falar de otimizacoes fixes observadas:
% lazyness a montar a rede (usar mensagens de servicos)
% manter peers antigos (churn inversamente proporcional a uptime)
% formar grupos para reduzir routing (increased background communication)
% ao usar prefix routing consegue-se logb(n) routing
% Xor-distance vs numeric distance (unidirectionality)
% Pedidos assincronos para fazer queries mais rapidas
% usar um algoritmo para dar "feed" ao outro (gossip + dht)




\section{Resource Monitoring} \label{sec:res_monitoring} 

\subsection{Device monitoring}

\subsection{Service monitoring}

\subsection{Relevant monitoring services}

\subsection{discussion}

%\section{Resource Management} \label{sec:res_management} 
In this section, we study resource management in the context of edge environments. Resource management consists in providing resources (e.g. computing power, memory, among others) to tenants (i.e. applications, frameworks, among others), such that these can perform their computations. In this section, we cover aspects of resource management solutions and study popular solutions in the literature.

\subsection{Resource Management Taxonomy}

A resource management system aims at controlling the distribution of resources among tenants. We may classify resource management architectures according to their \textit{control} and \textit{tenancy}.

\subsubsection{Tenancy}

The term tenancy in resource management refers to whether or not underlying hardware resources are shared among entities \cite{Hong2019}.

\textbf{Single tenancy} refers to an architecture in which a single instance of a software application and supporting infrastructure serves one customer. In single-tenancy architectures, a customer (tenant) has nearly full control over the customization of software and infrastructure.

\textbf{Multi-tenancy} consists of tenants sharing multiple resources across multiple processes and machines. This approach has clear advantages, as sharing the infrastructure leads to lower costs (e.g. electricity), and companies of all sizes like to share infrastructure in order to achieve lower operational costs.

However, providing performance guarantees and isolation in multi-tenant systems is extremely hard, resource management systems must avoid mismatching the resource allocation, as tenant-generated requests compete with each other and with the system generated tasks. Furthermore, tenant workload can change in unpredictable ways depending on the input workload, the workload of other tenants in the system, and the underlying topology.

\subsubsection{Control}

Control refers to how resource management systems allocates tasks to available resources, there are two alternatives towards performing resource allocations: either \textit{centralized} or \textit{decentralized}.

\textbf{Centralized control} consists in a centralized component with a global view of the state of the system making all decisions regarding resource allocations. Intuitively, given that a centralized component generates manages all the resources in the system, this component can easily enforce policies to achieve the desired performance guarantees or fairness goals by identifying and only throttling the tenants or system activities responsible for resource bottlenecks \cite{verma2015large}.

\textbf{Decentralized control} architectures are defined by having the decision-making process regarding resource allocations distributed across multiple components \cite{Hong2019}. This topic has yet not been subject to much research, although it is of extreme relevance towards edge environments. For example, if the system is globally distributed, it may take too long for a centralized controller to identify hotspots in a certain zone and load-balance them.

One of the key challenges in distributed resource management is ensuring that the components which perform resource assignments do not conflict with each other. Additionally, in a multi-tenant decentralized resource management system, tenants may request resources to different resource controllers in the system, and if they do not coordinate themselves, the application may be provisioned with too many (or too little) resources.

\subsection{Resource Management Systems}

\textbf{Mesos} \cite{hindman2011mesos} is a multi-tenant centralized resource sharing platform that attempts to provide fine-grained resource sharing within a data centre. The tenants for this platform are frameworks such as HDFS \cite{borthakur2008hdfs}, MapReduce \cite{dean2008mapreduce}, among others, which in turn support multiple applications running within a DC. In short, the Mesos resource sharing system consists of a \textit{master} process which manages \textit{slave} daemons running on each cluster node. In order to achieve fault-tolerance for the master component, Mesos employs Zookeeper \cite{hunt2010zookeeper} to maintain replicas, elect a new master, and transfer state to a new master in case the active master fails.

The master implements fine-grained sharing of resources across frameworks by employing \textit{resource offers}, which consist of lists containing free resources distributed among slaves. The master makes decisions about how many resources to offer to each framework, and the decision-making process is based on an arbitrary organizational policy, such as fair sharing or priority. Each framework that wishes to use Mesos must implement a \textit{scheduler} and an \textit{executor}. The scheduler registers with the Mesos master to receive resource offers, and the executor is the process that is launched on slave nodes to run the framework's tasks.

A limitation of the Mesos resource sharing platform is that it is not scalable, given the central component issuing resource allocations (the original authors mention the system scales up to 50000 slave daemons on 99 physical machines), which is not enough for an edge environment. Furthermore, the resource offer model forces frameworks to employ a specific programming model based on schedulers and executors, which we believe to be too restrictive.

\textbf{Yarn} (Yet Another Resource Negotiator) \cite{Vavilapalli2013ApacheHY} is a centralized multi-tenant resource sharing platform that decouples the programming model from the resource management infrastructure and delegates many scheduling functions to per-application components. The architecture of YARN is composed by: a per-cluster Resource Manager (RM), multiple Application Masters (AM), and Node Managers (NM). The Resource Manager (RM) tracks resource usage and node liveness, enforces allocation invariants and arbitrates contention among tenants.

Application Masters (AM) run arbitrary user code, their duties in the system consist of managing the lifecycle aspects, including dynamically increasing and decreasing resource consumption, managing the flow of execution, and handling faults. Node Managers (NM) are worker daemons, whose responsibilities consist of managing container dependencies, monitoring their execution, and providing a set of services for them.

AMs send resource requests to the RM, containing the number of containers to request, the resources per container, locality preferences, and a priority level within the application. These requests are designed to capture the needs of applications while at the same time removing application concerns (such as task dependencies) from the scheduler. Because the RM is in charge of processing and scheduling all task distributions for each request made by AMs, it is effectively a \textit{monolithic} scheduler. By consequence, there is a unique point of failure, which makes this system inadequate for large scale edge environments.

\textbf{Omega} \cite{41684} is a scheduler designed for grid computing systems composed by schedulers and workers. Each scheduler receives large amounts of jobs composed by either one or many tasks that have to be scheduled among workers. Contrary to YARN, which is monolithic, OMEGA uses multiple schedulers per cluster, each with a shared global view of the cluster state.

Schedulers make task placement decisions according to their view of the cluster state and their scheduling policy. If two or more schedulers attempt to schedule a task to the the same worker (i.e., generating a conflict), the worker first tries to accommodate both tasks, if it cant, it rejects the least important one.

One advantage of OMEGA in relation to MESOS is that MESOS resource attributions ``lock'' the resources to the corresponding framework,  which means that only one framework is examining a resource at a time. While it achieves higher throughput in allocation operations, its main limitations are that: (1) in case the grid becomes overloaded, resource allocations can potentially start interfering with each other; (2) scheduling policies are harder to ensure; and finally, (3) all schedulers must have global knowledge of the system.

\textbf{Edge NOde Resource Management} \cite{wang2017enorm} (ENORM) is framework aimed at employing edge resources towards applications by provisioning and auto-scaling edge node resources. ENORM proposes a three-tier architecture: (1) the Cloud tier, where application servers are hosted; (2) the middle tier, where the edge nodes are situated; and (3) the bottom tier, where user devices (e.g. smartphones, wearables, gadgets) are situated.

To enable the use of edge nodes, ENORM deploys a cloud server manager on each application server, which communicates with potential edge nodes, requesting computing services. Using these computing resources, it deploys partitioned servers on the edge nodes. Edge nodes are maintained in a global view.

ENORM authors tested the designed system using an online game inspired on Pokemon GO (iPokemon)\cite{pokemonGo}. The ENORM framework partitions the game server and sends user data to each edge node containing information regarding the users within that geographical location. Users from the relevant geographical zone then connect to the edge server and are serviced by a geographically closer edge node as if they were connected to the data centre. Limitations from this framework are the large size of the required information to perform the deployments, and similarly to previous solutions, the lack of fault-tolerance and scalability, from employing a centralized component to perform monitoring and management of resources.

\textbf{FogTorch} \cite{Brogi2017} is a service deployment framework aimed at determining eligible deployments for an application over a given Fog infrastructure, modeled by: (1) Cloud Data Centers, denoted by their location and software capabilities; (2) Fog Nodes, that consist of tuples containing: the location, hardware, the software capabilities, and the things directly reachable from the fog node; (3) Things, which are represented by a tuple denoting the thing (sensor or actuator) location and its type; (4) QoS profiles, that are sets of QoS profiles composed by the latency and bandwidth of a communication link. (5) Applications, which are composed of independent sets of components, each with a set of requirements regarding QoS profiles, hardware and software capabilities, and things. Then, authors model service deployments as restrictions over the system model and employ a greedy heuristic, which reduces the search space of devices constituting options for these service deployments.

FogTorch is also the base for \textbf{FogTorchPI} \cite{brogi2017best}, which is a solution that employs the system model of FogTorch, however instead of a greedy approach, it uses Monte Carlo simulations to calculate the best possible deployment configurations.

These solutions provide a comprehensive system model which models many different types of application requirements, however,similarly to FogTorch, it requires an updated global view of the system, which requires collecting a large amount of information to a central entity, limiting system scalability.

\subsection{Discussion}

Although resource management systems have been present for many years, these are often tailored towards small scale environments composed by homogenous devices in stable environments, which contrast with the edge of the network, where devices are extremely numerous, operate on a decentralized fashion, and are highly heterogenous.

We argue that a centralized controller is not the ideal solution for an edge environment, given the fact that as the number of devices in the system increases, so does the number of resources to track, and the harder it is for a centralized component to have an up-to-date global view of the system.

Due to their low capacity, devices at the edge of the network are very susceptible to workload changes, for example, a 5G tower that is hosting services cannot handle a drastic increase in the number of users it is serving. In this scenario, we argue that in order to maintain pre-established performance criteria, devices must autonomously make resource management decisions such as scaling an allocation horizontally or vertically in order to quickly meet the demands of users/tenants.

% \subsubsection{Scheduling}

% There are many approaches towards scheduling resources among edge nodes: \textbf{Brute force} proposes exhaustively exploring all potential targets combinations towards offloading tasks (including the Cloud, the Edge and other user devices) and picking the one which provides the minimum execution time. This technique intuitively is not scalable enough to be applied in practice.

% \textbf{Greedy} heuristics focus on minimizing the time it takes for the task to be completed on a mobile device. FogTorch \cite{Brogi2017} employs a greedy heuristic by which reduces the search space of devices that constitute options for service deployments. 

% \textbf{Simulated Annealing} employs a search space based on the utilization of edge and cloud nodes, total costs, and the completion time of the task to find the optimal solution.

% \subsubsection{Offloading}

%In this section we study offloading, which is technique used by edge-enabled applications to fully take advantage of edge nodes.

%Offloading is a technique in which a server, application and the associated data are transferred onto another node in the network \cite{Hong2019}. There are two variants for offloading: either from the user device to the edge, or from the cloud to the edge. 

%Offloading from user device to the edge enhances computing in mobile nodes by employing edge nodes which are usually only one or two hops away. While offloading from the Cloud to the edge has the potential to reduce bandwidth consumption and improve QoS of edge-enabled applications. 

%\textbf{Server offloading} is a technique in which servers are offloaded to the edge via either replication or partitioning. \textbf{Replication} consists in offloading the full server state(e.g. a database or an  application server), while \textbf{partitioning} consists in offloading only a portion of the server state. 

%The portion of the server to offload must take into account a set of parameters such as latency, functionality, energy efficiency, geographical distribution, among others. 



\section{Summary}

The purpose of this chapter was to provide a brief overview of the studied relevant works and techniques found in the literature regarding (1) the edge environment and execution environments for edge environments; (2) construction of overlay networks; (3) resource monitoring platforms, and (4) resource location systems, with emphasis on analyzing their applicability toward edge Environments. Firstly, we began by studying the devices that we believe compose these environments and debated the applicability of popular execution environments for edge-enabled applications, following we addressed popular architectures and implementations of both structured and unstructured overlay networks, and analyzed popular techniques in the literature used towards performing resource location and discovery in these networks. After this, we examined related work regarding collecting metrics in a decentralized manner.

In the next chapter we present the proposed solution that we named DEMMON, which draws inspiration from the study of the state of the art to enable the decentralized management and monitoring of resources in the edge of the network.
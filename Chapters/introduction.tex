%!TEX root = ../template.tex
%%%%%%%%%%%%%%%%%%%%%%%%%%%%%%%%%%%%%%%%%%%%%%%%%%%%%%%%%%%%%%%%%%%
%% chapter1.tex
%% NOVA thesis document file
%%
%% Chapter with introduction
%%%%%%%%%%%%%%%%%%%%%%%%%%%%%%%%%%%%%%%%%%%%%%%%%%%%%%%%%%%%%%%%%%%


\typeout{NT FILE introduction.tex}

\chapter{Introduction}
\label{cha:introduction}

\prependtographicspath{{Chapters/Figures/Covers/phd/}{Chapters/Figures/Covers/msc/}}


\section{Context} \label{sec:context}

Nowadays, the Cloud Computing paradigm is the standard for development, deployment, and management of services, most of the software systems present in our everyday life, such as Google Apps, Amazon, Twitter, among many others, are deployed on some form of cloud service. Cloud Computing refers to both the applications delivered as services over the Internet and the hardware and software systems in the data centers that provide those services \cite{10.1145/1721654.1721672}. It enabled the illusion of unlimited computing power, which revolutionized the way developers, companies, and users develop, maintain, and changed our expectations from these services.

However, the centralized model proposed by the Cloud Computing paradigm mismatches the needs of many types of applications such as latency-sensitive applications, interactive mobile applications, and IoT applications \cite{10.1145/3154815}. All these application domains are characterized by having data being generated and accessed (predominantely) by end-user devices. When the computation resides in the data center (DC), far from the source of the data, challenges may arise: from the physical space needed to contain all the infrastructure, the increasing amount of bandwidth needed to support the information exchange from the DC to the client as well as the latency in communication from the clients to the DC have directed us into a new computing paradigm: \textit{Edge Computing}.

Edge computing addresses the increasing need for enriching the interaction between cloud computing systems and interactive/collaborative web and mobile applications \cite{10.1145/242857.242867} by taking into consideration computing and networking resources which exist beyond the boundaries of DCs, closer to the edge of systems \cite{Leitao2018} \cite{7488250}. This paradigm also aims at enabling the creation of systems that could otherwise be unfeasible with Cloud Computing: Google's self-driving car generates 1 Gigabyte every second \cite{datafloq}, and a Boeing 787 produces data at a rate close to 5 gigabytes per second \cite{finnegan_2013}, which would be impossible to transport and process in real-time (e.g., towards self-driving) if the computations were to be carried exclusively in a DC.

By taking into consideration all the devices which are external to the DC, as these range from Edge Data Centers to 5G towers and mobile devices, we are faced with a huge increase in the number and diversity of computational devices, that contrary to the cloud, have a wide range of computational capacity, and limited and (potentially) unreliable data lines. Given this, we believe developing an efficient resource management/sharing platform that uses these devices toward generic computation is an open challenge for fully realizing in Edge Computing.

\section{Motivation}

Resource management/sharing platforms are extensibly used in Cloud systems (e.g. Mesos \cite{hindman2011mesos}, Yarn \cite{Vavilapalli2013ApacheHY}, Omega \cite{41684}, among others), whose high-level functionality consist of: (1) federating all the devices and tracking their state and utilization of computational and networking resources; (2) keeping track of resource demands which arise from different tenants; (3) performing resource allocations to satisfy the needs of such tenants; (4) adapting to dynamic workloads such that the system remains balanced and system policies as well as performance criteria are being met.

Most popular resource management and sharing platforms are tailored towards small numbers of homogenous resource-heavy devices, which rely on a centralized system component that performs resource allocations with global knowledge of the system. Although this system architecture heavily simplifies the management of the resources, we argue that such systems, as they are often plagued by a central point of failure and a single point of contention, have hindered scalability and fault-tolerance, making them unsuitable for the scale of Edge Computing systems. 

Instead, for achieving general-purpose computation in Edge systems, we argue in favour of decentralized management/sharing, composed of multiple components, organized in a flexible hierarchical way, and promoting load management decisions supported by partial and localized knowledge of the system. As building such a platform would not be trivial, and as we believe that in such a system, the accuracy and freshness of the information (which may be but is not exclusive to the execution of components or services) each component has, dictates how efficiently they manage resources, we focus on that particular task: \text{decentralized data collection and aggregation}. It is the goal of this work to develop a solution that provides efficient decentralized data collection and aggregation primitives over multiple nodes inserted in cloud-edge scenarios. It is our final goal, through the creation of this solution, to ease the creation of a new generation of fully decentralized resource management solutions, performing efficiently with just a partial and localized knowledge of the system.

% abstracts layer that provides decentralized data collection, aggregation and dissemination primitives, targeted for resource decentralized management/sharing systems.

% Data aggregation is an essential step towards general-purpose computations in Edge systems, as it allows information to be summarized. For devices with constrained data links and limited resources in resource management systems, being able to summarize data in transit is crucial, as it provides them with a partial view of the aggregated value, which in turn can be used in decentralized resource management decisions (e.g. load-balancing, improving QOS, among others).

\section{Contributions}

The contributions which arose from this dissertation are as follows:

\begin{enumerate}
    \item A distributed monitoring framework, built for decentralized resource management systems, composed of three main components: {
        \begin{enumerate}
            \item A novel overlay protocol which strives to build a logical multi-tree-shaped overlay network using both bandwidth and node latency as heuristics for connection establishment. This protocol is fully decentralized and fault-tolerant, with its configuration being only a set of static nodes: the roots of the trees, and each node's latency.
            
            \item A distribured aggregation protocol, which uses the connections created by the overlay protocol's tree structure to perform efficient on-demand in-transit aggregations.
            
            \item An API to consult information regarding the operation of the overlay protocol, and to issue or collect arbitrary information in the framework in the form of time-series data. This framework also allows other auxiliary functions such as applying periodic functions to information, or alerting based on provided information. 
            
            \item An experimental evaluation of the membership protocol against popular membership protocols in the state of the art, where their fault tolerance, ability to improve the network cost, and ability to perform information dissemination reliably is tested. 

            \item An experimental evaluation of the monitoring protocol against common prometheus configurations. Here, the accuracy of the collected monitoring values is collected over time, as well as information regarding the networking/processing cost of collecting the information.
            
        \end{enumerate}
        }
    \item A benchmark in the form of an edge-enabled application composed by multiple loosely coupled micro-services, tailored to benchmark resource management platforms, in this benchmark, geographical proximity leads to a significant improvement of QOS for the end-user, favouring resource management platforms which value placement of their services closer to the client.
    
\end{enumerate}

\section{Document structure}

\todo{elaborate this}
The remaining of this document is structured as follows: 

\textbf{Chapter} \ref{cha:related_work} studies related work that is related with the overall goal of this thesis work: we begin by analyzing similar paradigms to Edge Computing, the devices which compose these environments, and execution environments for edge-enabled applications. Following, we discuss strategies towards federating various devices in an abstraction layer, and study search strategies to find resources in the this layer, finally, we cover monitoring and management of system resources.

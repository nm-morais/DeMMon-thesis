%!TEX root = ../template.tex
%%%%%%%%%%%%%%%%%%%%%%%%%%%%%%%%%%%%%%%%%%%%%%%%%%%%%%%%%%%%%%%%%%%
%% chapter1.tex
%% NOVA thesis document file
%%
%% Chapter with introduction
%%%%%%%%%%%%%%%%%%%%%%%%%%%%%%%%%%%%%%%%%%%%%%%%%%%%%%%%%%%%%%%%%%%

\typeout{NT FILE introduction.tex}

\chapter{Introduction}
\label{cha:introduction}

\section{Motivation}

Nowadays, the Cloud Computing paradigm is the standard for development, deployment, and management of services, most of the software systems present in our everyday life, such as Google Apps, Amazon, Twitter, among many others, are deployed on some form of cloud service. Cloud Computing refers to both the applications delivered as services over the Internet and the hardware and software systems in the data centers that provide those services \cite{10.1145/1721654.1721672}. It enabled the illusion of unlimited computing power, which revolutionized the way developers, companies, and users develop, maintain, and even use services.

However, the centralized model proposed by the Cloud Computing paradigm mismatches the needs of many types of applications  such as: latency-sensitive applications, interactive mobile applications, and IoT applications \cite{10.1145/3154815}. All of these application domains are characterized by having data being generated and accessed (mostly) by end-user devices, consequently, when the computation resides in the data center (DC), far from the source of the data, challenges may arise: from the physical space needed to contain all the infrastructure, the increasing amount of bandwidth needed to support the information exchange from the DC to the client, the latency in communication from the clients to the DC as well as the security aspects that emerge from offloading data storage and computation to DCs operated by third parties have directed us into a new computing paradigm: \textit{Edge Computing}.

Edge computing addresses the increasing need for enriching the interaction between cloud computing systems and interactive / collaborative web and mobile applications \cite{10.1145/242857.242867}, by taking into consideration  computing and networking resources which exist beyond the boundaries of DCs and close to the edge of systems \cite{Leitao2018} \cite{7488250}. This paradigm aims at enabling the creation of systems which could otherwise be unfeasible with Cloud Computing: Google's self-driving car generates 1 Gigabyte every second \cite{datafloq}, and a Boeing 787 produces data at a rate close to 5 gigabytes per second \cite{finnegan_2013}, which would be impossible to transport and process in real-time (e.g., towards self-driving) if the computations were to be carried exclusively in a DC. 

By taking into consideration all the devices which are external to the DC, we are faced with a huge increase in the number and characteristics of computational devices ranging from Edge Data Centers to 5G towers and mobile devices. These devices, contrary to the cloud, have heterogenous computational capacity and potentially limited and unreliable data lines. Given this, developing an efficient resource management and sharing platform which enables the adequate and efficient use of these devices is an open challenge for fully realizing Edge Computing. 

\section{Context}

Resource management and sharing platforms are extensibly used in Cloud systems (e.g. Mesos \cite{hindman2011mesos}, Yarn \cite{Vavilapalli2013ApacheHY}, Omega \cite{41684}, among others), whose high-level functionality consist of: (1) federating all the devices and tracking their state and utilization of computational and networking resources; (2) keeping track of resource demands which arise from different tenants; (3) performing resource allocations to satisfy the needs of such tenants; (4) adapting to dynamic workloads such that the system remains balanced and system policies as well as performance criteria are being met.

Most popular resource management and sharing platforms are tailored towards small numbers of homogenous resource-heavy devices, which rely on a centralized system component that performs resource allocations with global knowledge of the system. Although this system architecture heavily simplifies the management of the resources, we argue that such systems are plagued by a central point of failure and a single point of contention which hinders the scalability of such such solutions, making them unsuitable for the scale of Edge Computing systems. Instead, we argue in favor of decentralized architectures for such resource management and sharing platforms, composed of multiple components, potentially organized in a flexible hierarchical way, and promoting load management decisions supported by partial and localized knowledge of the system.

Given this, it is paramount that devices cooperatively materialize a robust lightweight decentralized resource control system, which tracks resource demands and allocations. In such a system, the accuracy and freshness of the information each component has dictates how efficiently they manage resources such that the system remains balanced, and applications running on the infrastructure maintain their target quality of service. This system must federate devices such that they leverage on heterogeneity to build a hierarchical infrastructure which combines naturally with the device taxonomy, and adapts to the environment changes.

\section{Expected Contributions}

The expected contributions from the work to be conducted in this thesis, as will be further detailed in Chapter \ref{cha:planning}, arise from the aforementioned challenges. We plan to focus on creating a decentralized resource monitoring and management solution tailored for provisioning resources for edge-enabled applications. Given this, the expected contributions to arise from our work consist of: 

\begin{itemize}

    \item A novel decentralized overlay network protocol tailored towards federating large numbers of edge devices in a hierarchical way.
    
    \item A lightweight monitoring solution which relies on that overlay structure to aggregate information regarding applications’ operation and the load across edge resources.
    
    \item A decentralized resource management solution capable of satisfying resource allocations for multi-tenant resource sharing, based on the monitoring information captured by our platform
    
    \item An extensive experimental assessment of our solution that will demonstrate its correctness and compare its performance with regard to existing solutions in a realistic test bed. Parts of this experimental work will be based on simple mock ups of applications that will be implemented to exercise our platform.

    %\item Design an experimental scenario for the system and collect metrics to evaluate the performance and correctness of designed components. Additionally, we may design or adapt an existing edge-enabled application to test the solution in a real world scenario by scheduling application deployments and tracking metrics about the overlay and the quality of service of the applications.


    
\end{itemize}

\section{Document structure}

This remaining of this document is structured as follows:

\textbf{Chapter} \ref{cha:related_work} studies related work that is related with the overall goal of this thesis work: we begin by analyzing similar paradigms to Edge Computing, the devices which compose these environments, and execution environments for edge-enabled applications. Following, we discuss strategies towards federating various devices in an abstraction layer, and study search strategies to find resources in the this layer, finally, we cover monitoring and management of system resources.

\textbf{Chapter} \ref{cha:planning} further elaborates on the proposed contributions and the proposed work plan for the remainder of the thesis, including a more detailed explanation of our plans to experimentally evaluate our solution.

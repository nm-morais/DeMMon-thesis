%!TEX root = ../template.tex
%%%%%%%%%%%%%%%%%%%%%%%%%%%%%%%%%%%%%%%%%%%%%%%%%%%%%%%%%%%%%%%%%%%
%% chapter1.tex
%% NOVA thesis document file
%%
%% Chapter with introduction
%%%%%%%%%%%%%%%%%%%%%%%%%%%%%%%%%%%%%%%%%%%%%%%%%%%%%%%%%%%%%%%%%%%


\typeout{NT FILE introduction.tex}

\chapter{Introduction}
\label{cha:introduction}

\prependtographicspath{{Chapters/Figures/Covers/phd/}{Chapters/Figures/Covers/msc/}}


\section{Context} \label{sec:context}

Nowadays, the Cloud Computing paradigm is the standard for the development, deployment, and management of services for most software systems present in our everyday life. Google Apps, Amazon, Twitter, among many others, are deployed on some form of cloud infrastructure and benefit from cloud-based services. Cloud Computing refers to both the applications delivered as services over the Internet and the hardware and software systems in the data centers that provide those services \cite{10.1145/1721654.1721672}. This paradigm enables the illusion of unlimited computing power, which revolutionized the way companies and developers design, develop, maintain and manage their online applications, as well as the expectations that users have from them.

However, the centralized model proposed by the Cloud Computing paradigm mismatches the needs of many types of applications such as latency-sensitive applications, interactive mobile applications, and IoT applications \cite{10.1145/3154815}. All these application domains are characterized by having data being generated and accessed (predominantly) by end-user devices. When the computation resides in the data center (DC), far from the source of the data, challenges may arise: from the physical space needed to contain all the infrastructure, the increasing amount of bandwidth needed to support the information exchanges as well as the latency in communication from the clients to the DC. All of these challenges have directed us into a new computing paradigm: \textit{Edge Computing}.

Edge computing addresses the increasing need for enriching the interaction between cloud computing systems and interactive/collaborative web and mobile applications \cite{10.1145/242857.242867} by taking into consideration computing and networking resources which exist beyond the boundaries of DCs, closer to the edge of systems \cite{Leitao2018} \cite{7488250}. This paradigm also aims at enabling the creation of systems that could otherwise be unfeasible with Cloud Computing: Google's self-driving car generates 1 Gigabyte every second \cite{datafloq}, and a Boeing 787 produces data at a rate close to 5 gigabytes per second \cite{finnegan_2013}, which would be impossible to transport and process in real-time (e.g., towards self-driving) if the computations were to be carried exclusively in a DC.

By taking into consideration all the devices which are external to the DC, as these range from Edge Data Centers to 5G towers and mobile devices, we are faced with a huge increase in the number and diversity of computational devices, that contrary to the cloud, have a wide range of computational capacity, and limited and (potentially) unreliable connections. Given this, we believe developing an efficient resource management/sharing platform that uses these devices toward generic computation is an open challenge to fully accomplish in Edge Computing.

\section{Motivation}

Resource management/sharing platforms are extensibly used in Cloud systems (e.g. Mesos \cite{hindman2011mesos}, Yarn \cite{Vavilapalli2013ApacheHY}, Omega \cite{41684}, among others), whose high-level functionality consists of: (1) federating all the devices and tracking their state and utilization of computational and networking resources; (2) keeping track of resource demands which arise from different tenants; (3) performing resource allocations to satisfy the needs of such tenants; (4) adapting to dynamic workloads such that the system remains balanced and system policies as well as performance criteria can be ensured.

Most popular resource management and sharing platforms are tailored towards small numbers of homogenous resource-heavy devices, which rely on a centralized system component that performs resource allocations with a global knowledge of the system (including available computational resources, their usage, workloads received per each hosted application, e.t.c). Although this system architecture heavily simplifies the management of the resources, we argue that such systems, as they are often plagued by a central point of failure and a single point of contention, have hindered scalability and fault-tolerance, making them unsuitable for the more heterogeneous and larger infrastructure that can be leveraged by Edge Computing systems. 

Instead, to achieve general-purpose computation in Edge systems, we argue in favour of decentralized management/sharing systems, composed of multiple components, organized in a flexible hierarchical way, that perform resource management decisions supported by partial and localized knowledge of the system state. Because building such a platform from scratch is not trivial, and as we believe that in such a system, the accuracy and freshness of the information available to each component (which includes but is not exclusive to the execution of components or services), dictates how efficiently they manage resources, we focus on that particular task: \textbf{decentralized data collection and aggregation}.

Hence, the goal of this work is to propose a novel solution that provides efficient decentralized data collection and aggregation primitives over multiple nodes located in and outside of the DC. It is our end goal to ease the creation of a new generation of fully decentralized resource management solutions that employ partial and localized knowledge, paving the way to more decentralized and effective solutions to manage complex edge infrastructures, enabling to improve the performance of future edge-enabled applications.

% abstracts layer that provides decentralized data collection, aggregation and dissemination primitives, targeted for resource decentralized management/sharing systems.

% Data aggregation is an essential step towards general-purpose computations in Edge systems, as it allows information to be summarized. For devices with constrained data links and limited resources in resource management systems, being able to summarize data in transit is crucial, as it provides them with a partial view of the aggregated value, which in turn can be used in decentralized resource management decisions (e.g. load-balancing, improving QOS, among others).

\section{Contributions}

The contributions which arose from the conducted work are as follows:

\begin{enumerate}

    \item A smaller contribution that derived from the work conducted in the context of this thesis which consists in a port to Golang of Babel \cite{babel}, a framework for building distributed systems, used to build our solution. This contribution has some additions focused on latency measurement and fault detection.

    \item A distributed monitoring framework, built for decentralized resource management systems, composed of three main components: {
        \begin{enumerate}
            \item A novel overlay protocol which strives to build a logical multi-tree-shaped logical network using both bandwidth and node latency as heuristics for defining the topology of the network. This protocol is fully decentralized and fault-tolerant, with its configuration being only a set of static nodes: the roots of the trees.
            
            \item A distributed aggregation protocol, which uses the connections made available by the overlay protocol's tree structure to perform efficient on-demand and decentralized information collection.
            
            \item An API that allows resource management applications to insert, process, and retrieve time-series data to and from a DeMMon node. This API also offers operations that allow the collection and processing of information from other DeMMon nodes in the network. 
            % Furthermore, it also allows applications to inspect the operation of the overlay protocol.
            
            \item An experimental evaluation of the membership protocol against popular alternatives found in the state of the art, where their fault tolerance, the ability to improve the network cost, and their capacity to perform information dissemination reliably is studied. 

            \item An experimental evaluation of the monitoring protocol against different Prometheus \cite{prometheus} configurations. This evaluation focuses on the accuracy of the collected monitoring values over time, as well as the cost for networking/processing the information.
            
        \end{enumerate}
        }
    \item A proposed benchmark in the form of an edge-enabled application composed by multiple loosely coupled micro-services, tailored to evaluate the performance of resource management platforms. In this benchmark, geographical proximity leads to a significant improvement of QOS for the end-user, favouring resource management platforms that optimize placement of their services closer to the client (i.e. applications that take advantage of edge computing).
    
\end{enumerate}

\section{Document structure}

The remaining of this document is structured as follows: 

Chapter \ref{cha:related_work} studies related work that is relevant to the overall goal of the work presented in this thesis: we begin by analyzing similar paradigms to Edge Computing, the devices which compose these environments, and execution environments for edge-enabled applications. We also discuss strategies towards federating various devices in an abstraction layer and study search strategies to find resources in this layer. Finally, we cover monitoring and management of system resources.

Chapter \ref{cha:GO-Babel} covers the design and implementation of the initial contribution, that as previously mentioned, consists of a port to Golang of Babel \cite{babel}, which in turn was used to build our solution. 

Following, chapter \ref{cha:demmon} explains the implementation of the developed solution, beginning with the overlay protocol, followed by the aggregation protocol that uses the overlay protocols' connections, and lastly, we cover the monitoring module of our system that stores and serves information as time-series.

After, in chapter \ref{cha:benchmark}, we cover the design and implementation of the previously mentioned edge-enabled benchmarking application targeted for testing decentralized resource management platforms.

In chapter \ref{cha:evaluation}, we provide the results of our experimental evaluation regarding the devised overlay protocol and aggregation protocol.

Finally, in chapter \ref{cha:conclusions_future_work}, we draw conclusions from the conducted work and discuss the future work we intend to pursue to further improve our solution.
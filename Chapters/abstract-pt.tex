%!TEX root = ../template.tex
%%%%%%%%%%%%%%%%%%%%%%%%%%%%%%%%%%%%%%%%%%%%%%%%%%%%%%%%%%%%%%%%%%%%
%% abstrac-pt.tex
%% NOVA thesis document file
%%
%% Abstract in Portuguese
%%%%%%%%%%%%%%%%%%%%%%%%%%%%%%%%%%%%%%%%%%%%%%%%%%%%%%%%%%%%%%%%%%%%

\typeout{NT FILE abstrac-pt.tex}

Independentemente da língua em que a dissertação esteja redigida, é necessário um resumo na mesma língua do texto principal e outro resumo noutra língua. Pressupõe-se que as duas línguas em questão sejam o português e o inglês.

Os resumos devem aparecer primeiro na língua do texto principal e depois na outra língua. Por exemplo, se a dissertação for redigida em português, o resumo em português aparecerá primeiro, seguido do resumo em inglês (\emph{abstract}), seguido do texto principal em português. Se a dissertação for redigida em inglês, o resumo em inglês (\emph{abstract} aparecerá primeiro, seguido do resumo em português, seguido do texto principal em inglês.

Na versão \LaTeX\, o template NOVAthesis irá ordenar automaticamente os dois resumos tendo em consideração a língua do texto principal. É possível alterar este comportamento adicionando
\begin{verbatim}
    \abstractorder(<MAIN_LANG>):={<LANG_1>,...,<LANG_N>}
\end{verbatim}
\noindent à zona de customização no preâmbulo do documento, e.g.,
\begin{verbatim}
    \abstractorder(de):={de,en,it}
\end{verbatim}

Os resumos não devem ultrapassar uma página e, de forma genérica, devem responder às seguintes questões (é essencial adaptá-los às práticas habituais da sua área científica):

\begin{enumerate}
  \item Qual é o problema?
  \item Porque é que é um problema interessante/desafiante?
  \item Qual é a proposta de abordagem/solução?
  \item Quais são as consequências/resultados da solução proposta?
\end{enumerate}

% E agora vamos fazer um teste com uma quebra de linha no hífen a ver se a \LaTeX\ duplica o hífen na linha seguinte se usarmos \verb+"-+… em vez de \verb+-+.
%
% zzzz zzz zzzz zzz zzzz zzz zzzz zzz zzzz zzz zzzz zzz zzzz zzz zzzz zzz zzzz comentar"-lhe zzz zzzz zzz zzzz
%
% Sim!  Funciona! :)

% Palavras-chave do resumo em Português
\begin{keywords}
Palavra-chave 1, Palavra-chave 2, Palavra-chave 3,  \ldots
\end{keywords}
% to add an extra black line

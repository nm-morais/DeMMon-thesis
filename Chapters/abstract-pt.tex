%!TEX root = ../template.tex
%%%%%%%%%%%%%%%%%%%%%%%%%%%%%%%%%%%%%%%%%%%%%%%%%%%%%%%%%%%%%%%%%%%%
%% abstrac-pt.tex
%% NOVA thesis document file
%%
%% Abstract in Portuguese
%%%%%%%%%%%%%%%%%%%%%%%%%%%%%%%%%%%%%%%%%%%%%%%%%%%%%%%%%%%%%%%%%%%%

O modelo de computação centralizado proposto pelo paradigma da Computação na Nuvem diverge do modelo das aplicações para a Internet das Coisas e para aplicações móveis, dado que a maioria da produção e requisição de dados é feita por dispositivos que se encontram distantes dos centros de dados. Armazenar dados e executar computações predominantemente em centros de dados incorre em custos de infrastrutura adicionais, aumenta a latência para os utilizadores e para fornecedores de serviços, como também levanta questões sobre a privacidade e segurança dos dados.

Para mitigar as limitações previamente mencionadas, surgiu um novo paradigma: Computação na Periferia. Este paradigma propõe executar computações, e potencialmente armazenar dados, em dispositivos fora dos centros de dados. No entanto, à medida que nos distanciamos dos centros de dados, a capacidade de computação e armazenamento dos dispositivos tende a ser limitada. Tendo isto, surge a necessidade de partilhar recursos entre dispositivos na periferia, de modo a executar computações sofisticadas que outrora seriam impossíveis com um único dispositivo destes.

O estudo do estado da arte revelou que as soluções existentes para a gestão e localização de recursos são normalmente especializadas para ambientes na Nuvem, onde os dispositivos têm capacidade de computação e armazenamento semelhantes, algo que não é adequado para ambientes dinâmicos e heterogéneos como a periferia do sistema. 
Nesta dissertação, propõe-se a criação de uma solução para a gestão e monitorização de recursos na periferia. Esta solução não só pretende gerir grandes quantidades de dispositivos, como também recolher e agregar métricas sobre a operação e execução de componentes aplicacionais, de forma descentralizada.
%, relativas ao funcionamento de componentes de aplicações hospedadas nestes dispositivos.
Estas métricas, por sua vez, auxiliam a tomada de decisão relativa à migração, replicação ou delegação (de porções) dos componentes aplicacionais, permitindo assim a adaptação autonómica do sistema.


% Palavras-chave do resumo em Português
\begin{keywords}

    %Edge Computing, Cloud Computing, Resource Management, Resource Monitoring, Topology Management, Overlay Networks, Migration, Offloading
    
    Computação na periferia, Gestão de recursos, Monitorização, Localização de recursos, Gestão de topologias de redes

\end{keywords}
% to add an extra black line
